\begin{questao}
  Mostre que existem infinitos conjuntos de $1983$
  inteiros positivos consecutivos, cada um dos quais é divisível por
  algum número da forma $a^{1983}$, onde $a$ é um inteiro
  positivo.
\end{questao}

\begin{questao}
  Suponha que o colar $A$ tenha $14$ pérolas e o colar
  $B$, $19$ pérolas. Prove que, para todo inteiro ímpar $n \geq
  1$, existe uma maneira de numerar cada uma das $33$ pérolas com
  um inteiro da sequência

  $$ \{n,n+1,n+2,\ldots,n+32 \} $$

  de forma que cada número seja usado apenas uma vez, e pérolas
  adjacentes recebam números primos entre si.
\end{questao}

\begin{questao}
  O conjunto $S = \{1/r: r=1,2,3,\ldots\}$ contém progressões
  aritméticas de vários tamanhos. Por exemplo, $(1/20; 1/8; 1/5)$ é
  uma de tais progressões, de tamanho $3$ e razão $3/40$. Mais
  ainda, esta é uma progressão {\it maximal} em $S$ de tamanho
  $3$, pois ela não pode ser estendida à esquerda ou à direita
  ($-1/40,11/40$ não são elementos de $S$.

  Mostre que existe uma progressão maximal em $S$ de tamanho $m$
  para todo $m \geq 3$.
\end{questao}

\begin{questao}
  Seja $a_n$ a sequência definida como

  \begin{align*}
    a_1 & = 1999 \\
    a_n & = a_{n-1}+p(n)
  \end{align*}

  onde $p(n)$ é o menor divisor primo de $n$. Mostre que $a_n$
  contém infinitos múltiplos de $7$.  
\end{questao}

\begin{questao}
  \begin{itemize}
    \item Existem $14$ inteiros positivos consecutivos cada um dos
    quais divisível por um ou mais primos do intervalo $2 \leq p \leq
    11$?

    \item Existem $21$ inteiros positivos consecutivos cada um dos
    quais divisível por um ou mais primos do intervalo $2 \leq p \leq
    13$?
  \end{itemize}

\end{questao}

\begin{questao}
  Mostre que existem infinitos conjuntos de $1983$ inteiros
  positivos consecutivos, cada um dos quais é divisível por algum número
  da forma $a^{1983}$, onde $a$ é um inteiro positivo.

\end{questao}

\begin{questao}
  Um ponto do reticulado é dito visível se $MDC(x,y)=1$. É
  verdade que dado um inteiro positivo $n$, existe um ponto do
  reticulado cuja distância a todo ponto visível é maior que ou igual a
  $n$?

\end{questao}

\begin{questao}
  Existe um número ímpar $n > 1$ que não é da forma
  $2^k+p$ onde $k$ é um número natural e $p$ é um primo ou o
  número $1$?

\end{questao}

\begin{questao}
  Prove que existe um inteiro positivo $k$ tal que
  $k \cdot 2^n+1$ é composto para todo inteiro positivo $n$.

\end{questao}

\begin{questao}
  Prove que é impossível desenhar no plano cartesiano uma
  linha poligonal fechada satisfazendo estas condições:

  \begin{itemize}
    \item As coordenadas de cada vértice são números racionais;

    \item O comprimento de cada aresta é igual a $1$;

    \item O número de vértices é ímpar.
  \end{itemize}

\end{questao}

\begin{questao}
  Considere a sequência de inteiros positivos $a_n$ satisfazendo a
  condição $0 < a_{n+1}-a_n \leq 2001$ para todo $n \geq 1$. Prove
  que existe um número infinito de pares de inteiros positivos $p,q$
  tais que $p<q$ e $a_p$ é divisor de $a_q$.
\end{questao}

\begin{questao}
  Encontre o maior $N$ para o qual existem $N$ inteiros positivos
  consecutivos tais que a soma dos dígitos do primeiro inteiro é
  múltipla de $1$, a soma dos dígitos do segundo inteiro é múltipla
  de $2$, a soma dos dígitos do terceiro inteiro é múltipla de
  $3$, \ldots, a soma dos dígitos do $N$-ésimo inteiro é múltipla
  de $N$.
\end{questao}

%%% Local Variables:
%%% mode: latex
%%% coding: utf-8-unix
%%% fill-column: 80
%%% TeX-master: "MASTER"
%%% End:
