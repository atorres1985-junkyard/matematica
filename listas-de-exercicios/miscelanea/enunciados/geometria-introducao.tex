
\begin{questao}

  O ângulo entre o lado $AC$ do triângulo $ABC$ e a
  mediana que passa por $A$ é igual a $30^\prime$,  assim como o
  ângulo entre o lado $BC$ e a mediana que passa por $B$. Prove
  que o triângulo $ABC$ é equilátero.
\end{questao}

\begin{questao}
  Seja, $A$ e $B$ pontos distintos da reta $p$. O
  ponto mais distante de $p$ da semicircunferência de diâmetro
  $AB$ é $C$. Seja $P$ um ponto da semicircunferência distinto
  de $A$, $B$ e $C$. A reta que passa por $P$ e $C$ corta
  $p$ em $Q$ e a reta perpendicular a $p$ por $Q$ intercepta
  $AP$ em $R$. Prove que os segmentos $BQ$ e $QR$ têm a mesma
  medida.
\end{questao}

\begin{questao}
  As circunferências $\Gamma_1$ de centro $O_1$ e
  $\Gamma_2$ de centro $O_2$ são disjuntas e estão do mesmo lado
  em relação à reta $p$. $Gamma_1$ e  $Gamma_2$ tangenciam $p$
  nos pontos $A_1$ e $A_2$ respectivamente. O segmento $O_1O_2$
  corta $\Gamma_1$ e $Gamma_2$ em $B_1$ e $B_2$
  respectivamente. Prove que as retas $A_1B_2$ e $A_2B_1$ são
  perpendiculares.
\end{questao}

\begin{questao}
  São dados uma circunferência de centro $O$ e um ponto
  exterior à circunferência $A$. A reta $AO$ corta a
  circunferência em $B$ e $C$ e a reta tangente à circunferência
  por $A$ toca a circunferência em $D$. Seja $E$ um ponto
  arbitrário sobre a reta $BD$ de modo que $D$ está entre $B$ e
  $E$. O circuncírculo do triângulo $DCE$ intercepta a reta $AO$
  em $C$ e $F$ e a reta $AD$ em $D$ e $G$. Prove que as
  retas $BD$ e $FG$ são paralelas.
\end{questao}

\begin{questao}
  O prolongamento da bissetriz $\bar{AL}$ do triângulo
  acutângulo $ABC$ intercepta no ponto $N$ a circunferência
  circunscrita ao triângulo. A partir do ponto $L$ traçam-se as
  perpendiculares $\bar{LK}$ e $\bar{LM}$ aos lados
  $\bar{AB}$ e $\bar{AC}$ respectivamente. Prove que a
  área do triângulo $ABC$ é igual à área do quadrilátero $AKNM$.
\end{questao}

\begin{questao}
  Duas cordas $\bar{AB}$ e $\bar{CD}$ de um
  círculo cortam-se em um ponto $E$ interior ao círculo. $M$ é um
  ponto interior ao segmento $\bar{EB}$. A tangente no ponto
  $E$ ao círculo que contém os pontos $D$,$E$ e $M$ corta as
  retas $BC$ e $AC$ nos pontos $F$ e $G$
  respectivamente. Calcule $EF/EG$ em função de $t = AM/AB$.
\end{questao}

\begin{questao}
  Sejam $ABC$ um triângulo e $M$ um ponto
  interior. Mostre que pelo menos um dos ângulos
  $\angle MAB$, $\angle MBC$ e $\angle MCA$ é menor ou igual a
  $30^\circ$.
\end{questao}

\begin{questao}
  Seja $ABC$ um triângulo isósceles com $AB=AC$. Suponha
  que
  \begin{itemize}
    \item $M$ é o ponto médio de $BC$ e $O$ é o ponto da reta $AM$ tal que $OB$ é
    perpendicular a $AB$;

    \item $Q$ é um ponto arbitrário no segmento $BC$ distinto de $B$ e $C$;

    \item $E$ é um ponto da reta $AB$ e $F$, da reta $AC$ tais que $E,Q,F$ são
    distintos e colineares.
  \end{itemize}

  Prove que $OQ$ é perpendicular a $EF$ se, e somente se, $QE=QF$.
\end{questao}

\begin{questao}
  Sejam $A,B,C,D$ quatro pontos distintos de uma reta,
  nesta ordem. Os círculos com diâmetros $AC$ e $BD$
  interceptam-se nos pontos $X$ e $Y$. A reta $XY$ encontra
  $BC$ no ponto $Z$. Seja $P$ um ponto da reta $XY$ diferente
  de $Z$. A reta $CP$ intercepta o círculo de diâmetro $AC$ nos
  pontos $C$ e $M$ e a reta $BP$ intercepta o círculo de
  diâmetro $BD$ nos pontos $B$ e $N$.\\
  Prove que as retas $AM$, $DN$ e $XY$ são concorrentes.
\end{questao}

\begin{questao}
  O ângulo $A$ é o menor dos ângulos do triângulo
  $ABC$. Os pontos $B$ e $C$ dividem a circunferência
  circunscrita ao triângulo em dois arcos. Seja $U$ um ponto
  interior ao arco $BC$ que não contém $A$.\\
  As mediatrizes de $AB$ e $AC$ cortam a reta $AU$ em $V$ e
  $W$, respectivamente. As retas $BV$ e $CW$ cortam-se em
  $T$.\\
  Demonstre que $$AU = TB + TC$$.
\end{questao}

\begin{questao}
  Seja $I$ o incentro do triângulo $ABC$. A
  circunferência inscrita no triângulo $ABC$ é tangente aos lados
  $BC$, $CA$ e $AB$ nos pontos $K$, $L$ e $M$
  respectivamente. A reta que passa por $B$, paralela ao segmento
  $MK$, intercepta as retas $LM$ e $LK$ nos pontos $R$ e
  $S$, respectivamente. Prove que o ângulo $\angle RIS$ é agudo.
\end{questao}

\begin{questao}
  Duas circunferências $\Gamma_1$ e $\Gamma_2$
  interceptam-se em $M$ e $N$.\\
  Seja $\ell$ a tangente comum a  $\Gamma_1$ e
  $\Gamma_2$ mais próxima de $M$ do que de $N$. A reta
  $\ell$ é tangente a $\Gamma_1$ em $A$ e tangente a
  $\Gamma_2$ em $B$. A reta paralela a $\ell$ que passa
  por $M$ intercepta novamente a circunferência $\Gamma_1$ em
  $C$ e novamente a circunferência $\Gamma_2$ em $D$.\\
  As retas $CA$ e $DB$ interceptam-se em $E$; as retas $AN$ e
  $CD$ interceptam-se em $P$; as retas $BN$ e $CD$
  interceptam-se em $Q$.\\
  Prove que $EP=EQ$.
\end{questao}

\begin{questao}
  Dado um quadrado $ABCD$ cujos lados medem $2$, sejam
  $M$ e $N$ pontos sobre os lados $AB$ e $CD$,
  respectivamente. As retas $CM$ e $BN$ cortam-se em $P$,
  enquanto as retas $AN$ e $DM$ cortam-se em $Q$. Mostre que
  $PQ \geq 1$.
\end{questao}

\begin{questao}
  Seja $ABCD$ um paralelogramo de lados $AB,BC,CD,DA$ e
  centro $O$ tal que $\angle BAD < 90^\circ$ e $\angle AOB >
  90^\circ$. Consideremos $A_1$ e $B_1$ pontos das semi-retas
  $OA$ e $OB$, respectivamente, tais que $A_1B_1$ é paralelo a
  $AB$ e $\angle A_1B_1C = \angle ABC/2$. Demonstre que $A_1D$ é
  perpendicular a $B_1C$.
\end{questao}

\begin{questao}
  Seja $ABC$ e $k_1$ uma circunferência passando pelos
  pontos $A$ e $C$ tal que $k_1$ intercepta $AB$ e $BC$ uma
  segunda vez nos pontos $K$ e $N$ respectivamente, $k \not =
  N$. Seja $O$ o centro de $k_1$ e seja $k_2$ o circuncírculo
  de $KBN$, que intercepta o circuncírculo de $ABC$ em $M$, $M
  \not = B$.\\
  Prove que $OM \perp MB$.
\end{questao}

\begin{questao}
  Em um pentágono convexo $ABCDE$, os lados $BC$, $CD$
  e $DE$ são iguais. Além disso, cada diagonal é paralela a um
  lado. Prove que $ABCDE$ é um pentágono regular.
\end{questao}

\begin{questao}
  Sejam $A_1$, $B_1$ e $C_1$ os pontos de intersecção
  dos prolongamentos das medianas do triângulo $ABC$ com o
  circuncírculo de $ABC$. Prove que se $A_1B_1C_1$ é equilátero, então
  $ABC$ é equilátero.
\end{questao}

\begin{questao}
  Seja $ABC$ um triângulo, e sejam $D$ e $E$ os pontos
  de intersecção das bissetrizes interna e externa do ângulo $A$ com
  a reta $BC$. Seja $F \not = A$ o ponto em que a reta $AC$
  intercepta a circunferência com diâmetro $DE$. Finalmente, seja
  $G \not = A$ o ponto em que a tangente em A à circunferência
  circunscrita ao triângulo $ABF$ encontra a circunferência com
  diâmetro $DE$. Prove que $AF = AG$.
\end{questao}

%%% Local Variables:
%%% mode: latex
%%% coding: utf-8-unix
%%% fill-column: 80
%%% TeX-master: "MASTER"
%%% End:
