\begin{questao}
  Prove que a fração $\displaystyle \frac{21n+4}{14n+3}$ é
  irredutível para todo natural $n$.
\end{questao}

\begin{questao}
  Qual o maior inteiro positivo $n$ para o qual $n^3+100$
  é divisível por $n+10$?
\end{questao}

\begin{questao}
  Os números na sequência $101, 104, 109, 116, \ldots$ são
  gerados pela fórmula $a_n=100+n^2\, n=1,2,3,4,\ldots$. Sendo
  $d_n=(a_n,a_{n+1})$, qual é o valor máximo assumido por $d_n$?
\end{questao}

\begin{questao}
  Prove que existem exatamente três triângulos retângulos
  cujos lados têm por medida números inteiros e cuja área é
  numericamente igual ao dobro do perímetro.
\end{questao}

\begin{questao}
  Sejam $x,y$ números primos entre si com $xy>1$ e seja
  $n$ um número par positivo. Prove que $x^n+y^n$ não é divisível
  por $x+y$.
\end{questao}

\begin{questao}
  Ao converter a fração $m/n$ em um número decimal, onde
  $m,n$ são inteiros positivos e $n$ é menor do que $100$,
  Vladimir encontrou um quociente que continha, em uma determinada
  posição após a vírgula decimal, os dígitos $167$ nesta ordem. Prove
  que Vladimir cometeu um erro nesta divisão.
\end{questao}

\begin{questao} 
  Sejam $k$ um número natural, $u,v$ números reais e $P(x) =
  (x-u^k)(x-uv)(x-v^k) = x^3+ax^2+bx+c$. Mostre que se $k=2$ e
  $a,b,c$ são racionais, então o produto $uv$ é racional.\\ 
  O mesmo vale para $k=3$?
\end{questao}

\begin{questao} 
  Mostre que as raízes cúbicas de três primos distintos não podem
  ser termos (não necessariamente consecutivos) de uma mesma progressão
  aritmética.
\end{questao}

\begin{questao}
  \begin{itemize}
    \item Para quais inteiros positivos $n$ existe um conjunto $S_n$ de $n$ inteiros
    positivos distintos tais que a média geométrica de qualquer subconjunto de
    $S_n$ é um inteiro?

    \item Existe um conjunto infinito $S$ de inteiros positivos tal que a média
    geométrica de qualquer subconjunto finito de $S$ é um inteiro?
  \end{itemize}
\end{questao}

\begin{questao}
  Determine o número máximo de elementos de $B \subset \{1,2,\ldots,n\}$ tal que
  para quaisquer $ a,b \in B, a \not = b$, $(a+b)$ não é múltiplo de $(a-b)$.
\end{questao}

\begin{questao}
  Prove que, para qualquer inteiro positivo $n$,
  $$ (n!+1, (n+1)!+1) = 1 $$.
\end{questao}

\begin{questao}
  Sejam $a_i, b_i, 1 \leq i \leq n$ inteiros distintos, tais que $(a_i,b_i) = 1$
  e $(b_i,b_j) = 1$ para $i \not= j$, mostre que a soma

  $$ \frac{a_1}{b_1} +  \frac{a_2}{b_2} + \ldots + \frac{a_n}{b_n} $$

  não é um número inteiro.
\end{questao}

\begin{questao}
  Para quais $n \in \mathbb{N}$ a função

  $$ f(n)=\frac{12n^3-5n^2-251n+389}{6n^2-37n+45} $$

  assume valores inteiros?
\end{questao}

\begin{questao}
  {\it (Números de Euclides)} Seja
  $E_1=1; E_n = 1+\prod_{j=1}^{n-1}{E_j} (n>1)$.

  Prove que $(E_i,E_j) = 1$ se, e somente se, $i \not= j$.
\end{questao}

\begin{questao}
  Determine o maior inteiro $k$ tal que 

  $$ 1991^k | 1990^{1991^{1992}} + 1992^{1991^{1990}} $$
\end{questao}

\begin{questao}

  \begin{itemize}
    \item Mostre que, se $S_0,a$ são inteiros primos entre si, com $S_0 > a \geq 1$,
    a sequência definida pela recursão

    $$ S_n-a = S_{n-1}(S_{n-1}-a), n \geq 1 $$

    consiste de naturais primos dois a dois.

    \item Mostre que o mesmo ocorre se $S_0$ é ímpar e

    $$ S_n=S_{n-1}^2-2, n \geq 1 $$
  \end{itemize}
\end{questao}

\begin{questao}
  Encontre todas as ternas de números naturais distintos, primos dois a dois,
  tais que a soma de quaisquer dois deles é divisível pelo terceiro.
\end{questao}

\begin{questao}
  Encontre todas as ternas de números naturais tais que a divisão
  euclidiana do produto de dois deles pelo terceiro tem como resto 1.
\end{questao}

\begin{questao}
  Um retângulo de lados inteiros $m,n$ é dividido em quadrados de lado
  $1$. Um raio de luz entra no retângulo por um dos vértices, no
  sentido da bissetriz do ângulo reto, e é refletido sucessivamente nos
  lados do retângulo. Quantos quadrados são atravessados pelo raio de
  luz?
\end{questao}

\begin{questao}
  Um retângulo $m \times n$ é dividido em quadrados de lado
  unitário. Qual é o número de quadrados atravessados pela diagonal?
\end{questao}

\begin{questao}
  Sendo $x+y=5432$ e $[x,y]=223020$ ($x,y \in \mathbb{Z}^{*}_{+}$), determine $x$ e $y$.
\end{questao}

\begin{questao}
  Encontre todos os inteiros $a,b,c$ com $1 < a < b < c$ tais que
  $(a-1)(b-1)(c-1)$ é divisor de $abc-1$.
\end{questao}

\begin{questao}
  Se os números naturais $a,b,l,m$ satisfazem as condições

  $$ (l,m)=1; a^l=b^m $$

  então existe um número natural $n$ tal que $a=n^m, b=n^l$.
\end{questao}

\begin{questao}
  Prove que, para todo inteiro positivo $m$, existe um número infinito
  de pares de inteiros $(x,y)$ tais que 

  \begin{itemize}[itemsep=1ex, leftmargin=1cm]
    \item $x$ e $y$ são primos entre si;
    \item $y$ divide $x^2+m$;
    \item $x$ divide $y^2+m$.
  \end{itemize}
\end{questao}

\begin{questao}
  Sejam $n$ um número inteiro maior que $6$ e $a_1,a_2,\ldots,a_k$
  todos os números naturais menores que $n$ e primos com $n$. Se

  $$ a_2-a_1 = a_3-a_2 = \ldots = a_k-a_{k-1} > 0 $$

  demonstre que $n$ é primo ou $n$ é uma potência de $2$.
\end{questao}

\begin{questao}
  Prove que para todo inteiro positivo $n$ temos
  $n^2|(n+1)^n-1$.
\end{questao}

\begin{questao}
  {\it(Números de Fermat)} Seja $F_k=2^{2^k}+1$.
  \begin{itemize}
    \item Prove que $(F_i,F_j)=1$ se e somente se $i \not= j$.
    \item Prove que os números primos da forma $2^n+1$
    são números (primos) de Fermat.
  \end{itemize}

\end{questao}

\begin{questao}
  Mostre que se $m,n$ são números naturais e $m$ é ímpar,
  então $(2^m-1,2^n+1) = 1$.
\end{questao}

\begin{questao}
  Encontre todos os inteiros $a,b,c$ tais que $abc=a+b+c$.
\end{questao}

\begin{questao}
  Determine todos os pares $(a,b)$ de inteiros positivos
  tais que $ab^2+b+7$ divide $a^2b+a+b$.
\end{questao}

\begin{questao}
  Encontre todos os inteiros positivos que podem ser
  representados unicamente na forma

  $$ \frac{x^2+y}{xy+1} $$

  em que $x,y$ são inteiros positivos.
\end{questao}

\begin{questao}
  Encontre todos os pares de inteiros positivos $(m,n)$ tais
  que $m^2-n$ divida $m+n^2$ e  $n^2-m$ divida $n+m^2$.
\end{questao}

%%% Local Variables:
%%% mode: latex
%%% coding: utf-8-unix
%%% fill-column: 80
%%% TeX-master: "MASTER"
%%% End:
