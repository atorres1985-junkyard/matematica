\begin{questao}
  No triângulo $ABC$, $AB=AC$. Uma circunferência é
  tangente internamente ao circuncírculo do triângulo $ABC$ e também
  aos lados $AB,AC$ nos pontos $P,Q$ respectivamente. Prove que o
  ponto médio do segmento $PQ$ é o incentro do triângulo $ABC$.
\end{questao}

\begin{questao} O incírculo de um triângulo $ABC$ ($\angle B$ reto) é
  tangente aos lados $BC,CA,AB$ nos pontos $D,E,F$
  respectivamente. Se $I$ é o centro do incírculo e $CI \cap EF =
  L, DL \cap AB=N $, mostre que $AI=ND$
\end{questao}

\begin{questao} Seja $ABC$ um triângulo retângulo e $BC$ sua
  hipotenusa, $D$ o pé da altura por $A$, $I$ o incentro do
  triângulo $ABC$, $J$ o incentro do triângulo $ABD$ e $K$ o
  incentro do triângulo $ACD$. Prove que $I$ é ortocentro do
  triângulo $AJK$.
\end{questao}

\begin{questao} No triângulo $ABC$, $AB=AC$. Uma circunferência é
  tangente internamente ao circuncírculo do triângulo $ABC$ e também
  aos lados $AB,AC$ em $P,Q$ respectivamente. Prove que o ponto
  médio do segmento $PQ$ é o incentro do triângulo $ABC$.
\end{questao}

\begin{questao} Suponha que três circunferências $C_1,C_2,C_3$ com o
  mesmo raio $r$, intersectam-se num ponto comum $O$ e também se
  intersectam duas a duas nos pontos $P_1,P_2,P_3$. Prove que o
  circuncírculo de $P_1P_2P_3$ também tem raio $r$.
\end{questao}

\begin{questao} Segmentos $AC, BD$ intersectam-se em um ponto $P$ tal
  que $PA=PD, PB=PC$. Seja $O$ o circuncentro do triângulo
  $PAB$. Prove que as retas $CD$ e $OP$ são perpendiculares.
\end{questao}

%%% Local Variables:
%%% mode: latex
%%% coding: utf-8-unix
%%% fill-column: 80
%%% TeX-master: "MASTER"
%%% End:
