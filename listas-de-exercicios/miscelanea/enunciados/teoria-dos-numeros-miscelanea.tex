\begin{questao}
  Existe um número natural que não seja divisor de qualquer número natural cuja
  representação decimal consista somente de zeros e uns, com não mais de 1988
  uns?
\end{questao}

\begin{questao}
  Encontre todos os naturais $n$ que satisfazem

  $$ S_{10}(n) = S_{10}(2n) = \ldots = S_{10}(n^2) $$
  ($S_{10}(n)$ é a soma dos dígitos decimais de $n$.)
\end{questao}

\begin{questao}
  Se um inteiro positivo $k$ contém todos os dez dígitos
  $0,1,2,\ldots,9$ e esta propriedade persiste através de seus
  $n$ primeiros múltiplos $k,2k,3k,\ldots,nk$, dizemos que $k$ é
  $n$-persistente.

  Prove que para cada $n$ existe pelo menos um número
  $n$-persistente.
\end{questao}

\begin{questao}
  Mostre que existe um número natural da forma $1999\ldots
  91$ (com mais de dois noves) que é múltiplo de $1991$.
\end{questao}

\begin{questao}
  Prove que para todo número natural $m$ existe $n, n>m$
  tal que a representação decimal de $5^n$ pode ser obtida a partir
  da representação decimal de $5^m$ adicionando uma certa quantidade
  de dígitos à direita.
\end{questao}

\begin{questao}
  Encontre todos os naturais $n$ tais que todos os números
  naturais cuja representação decimal é formada por $n-1$ dígitos
  $1$ e um dígito $7$ são primos.

\end{questao}

\begin{questao}
  Sejam $a,b,n$ inteiros positivos tais que $b>1$ e
  $b^n-1|a$. Mostre que a representação do números $a$ na base
  $b$ contém pelo menos $n$ dígitos diferentes de zero.

\end{questao}

\begin{questao}
  Um número {\it esburacado} é um número cujos algarismos são não
  nulos e zero, alternadamente, sendo que o dígito das unidades não é
  nulo. Por exemplo, $10402$ é esburacado, mas $2001$ e $2010$
  não. Determine todos os números que não dividem nenhum número
  esburacado.
\end{questao}

\begin{questao}
  Prove que existe uma potência de $2$ tal que seus
  primeiros $1992$ algarismos sejam iguais a $1$.

\end{questao}

\begin{questao}
  Prove que existe um natural $n$ tal que a expansão
  decimal de $n^{1992}$ começa com $1992$ algarismos iguais a
  $1$.

\end{questao}

\begin{questao}
  Seja $a_n$ o último dígito diferente de zero na
  representação decimal do número $n!$. A sequência
  $a_1,a_2,\ldots$ torna-se periódica após um número finito de
  termos?
\end{questao}

\begin{questao}
  A sequência de números reais $a_1,a_2,a_3,\ldots$ satisfaz a
  relação

  $$ \frac{a_{n+2}+a_{n+1}}{a_{n+1}+a_{n-1}} =
  \frac{a_{n+1}}{a_{n-1}} $$

  Sendo $a_1=2,a_2=500,a_3=2000$, demonstre que todos os termos da
  sequência são inteiros e $a_{2000}$ é divisível por $2^{2000}$.
\end{questao}

\begin{questao}
  Sendo $m$ um inteiro positivo, definimos
  $\epsilon_2(m)$ como o maior inteiro $k$ tal que $2^k|m!$.

  Prove que existem infinitos inteiros $m$ tais que
  $m-\epsilon_2(m)=1989$.

\end{questao}

\begin{questao}
  Encontre todos os inteiros positivos $n$ tais que, para
  quaisquer divisores $a,b$ de $n$, com $mdc(a,b)=1$, o número
  $a+b-1$ também divide $n$.

\end{questao}

\begin{questao}
  Seja $\sigma(n)$ a soma de todos os divisores positivos
  de $n$, em que $n$ é um inteiro positivo (por exemplo,
  $\sigma(6)=12$ e $sigma(11)=12$). Dizemos que $n$ é {\it quase
    perfeito} se $\sigma(n)=2n-1$ (por exemplo, $4$ é quase
  perfeito pois $\sigma(4)=7$). Sejam $n \pmod k$ o resto da
  divisão de $n$ por $k$ e $s(n)=\sum_{k=1}^{n}{n \pmod k}$ (por
  exemplo, $s(6)=0+0+0+2+1+0=3$ e
  $s(11)=0+1+2+3+1+5+4+3+2+1+0=22$.

  Prove que $n$ é quase perfeito se, e somente se, $s(n)=s(n-1)$.

\end{questao}

\begin{questao}
  O produto de $2001$ inteiros positivos distintos possui exatamente
  $2000$ divisores primos distintos. Mostre que podemos escolher
  alguns desses $2001$ números distintos de modo que seu produto
  seja quadrado perfeito.
\end{questao}

\begin{questao}
  Considere a sequência $a_n$:
  \begin{align*}
    a_0 & = 9\\
    a_{k+1} & = 3a_k^4 + 4a_k^3 (k \geq 0)
  \end{align*}

  Prove que $a_{10}$ tem mais de mil noves em sua representação decimal.
\end{questao}

\begin{questao}
  Seja $p$ um número primo. Encontre todos os inteiros positivos
  $x,y,z$ tais que $x^p+y^p=p^z$.
\end{questao}

\begin{questao}
  Determine todos os naturais $k>1$ tais que, para naturais
  distintos $m,n$, os números $k^m+1$ e $k^n+1$ podem ser
  obtidos um a partir do outro revertendo-se a ordem dos dígitos em
  suas representações decimais.
\end{questao}

\begin{questao}
  Os inteiros positivos $a_1,a_2,\ldots$, todos menores que
  $1998$, formam uma sequência que satisfaz a seguinte condição: se
  $m,n$ são inteiros positivos, então $a_m+a_n$ é divisível por
  $a_{m+n}$. Prove que esta sequência é periódica a partir de certo
  ponto.
\end{questao}

\begin{questao}
  Prove que, para todo $n \in N$, 

  $$ \sqrt[n]{\sqrt{3}+\sqrt{2}} + \sqrt[n]{\sqrt{3}-\sqrt{2}} $$

  é irracional.
\end{questao}

\begin{questao}
  Prove que, para todo natural $n$, existe um racional $a_n$ tal
  que $x^2+x/2+1$ divide $x^{2n}+a_nx^n+1$.
\end{questao}

\begin{questao}
  Mostre que $n^{12}-n^8-n^4+1$ é divisível por $2^9$ para todo
  $n$ ímpar.
\end{questao}

\begin{questao}
  Encontre todos os valores inteiros de $x$ tais que 

  $$ x^7 \equiv 10 \pmod{209} $$
\end{questao}

\begin{questao}
  Prove que, para todo inteiro positivo $k$, existe um natural que
  não pode ser escrito como soma de menos de

  $$ 2^k + \left \lfloor \left( \frac{3}{2} \right)^k \right \rfloor -
  2 $$

  $k$-ésimas potências positivas.
\end{questao}

\begin{questao}
  Começando em $(1,1)$, uma pedra é movida seguindo as regras:
  \begin{enumerate}[label=\emph{\alph*})]
    \item De $(a,b)$, podemos movê-la para $(2a,b)$ ou $(a,2b)$;

    \item De $(a,b)$, podemos movê-la para $(a,b-a)$ se $b>a$
    ou $(a-b,b)$ se $a>b$.
  \end{enumerate}

  Para quais pares $(x,y)$ de inteiros positivos podemos mover a
  peça para $(x,y)$?
\end{questao}

\begin{questao}
  Prove que é impossível desenhar no plano cartesiano um polígono com
  número ímpar de lados, tal que
  \begin{enumerate}[label=\emph{\roman*})]
    \item todos os seus vértices têm coordenadas inteiras, e
    \item todos os seus lados têm medida ímpar.
    \end{enumerate}
\end{questao}

\begin{questao}
  Mew e Pikachu jogam o seguinte jogo: cada um, em sua vez, escreve um
  divisor positivo de $100!$ e que não tenha sido escrito
  anteriormente. Após cada jogada, calcula-se o máximo divisor comum
  dos números já escritos, e se este valor for igual a $1$, o jogo
  termina e o jogador que escreveu o último número perde. Se Mew
  começou o jogo, qual dos dois monstros fofinhos tem a estratégia
  vencedora?
\end{questao}

\begin{questao}
  Um conjunto finito de inteiros positivos é chamado de {\it conjunto
    DS} se cada um dos seus elementos divide a soma de todos os
  elementos do conjunto. Demonstre que todo conjunto de inteiros
  positivos é subconjunto de algum conjunto DS.
\end{questao}

\begin{questao}
  Mostre que não existem inteiros positivos $a,b$ tais que
  $(36a+b)(36b+a)$ seja potência de $2$.
\end{questao}

%%% Local Variables:
%%% mode: latex
%%% coding: utf-8-unix
%%% fill-column: 80
%%% TeX-master: "MASTER"
%%% End:
