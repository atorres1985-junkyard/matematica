\begin{questao}  
  São dados $2n+3$ pontos no plano, em posição
  geral. Prove que existe uma circunferência que contém três pontos
  e divide os restantes pela metade, i.e., metade situa-se no seu
  interior e a outra metade no seu exterior.
\end{questao}

\begin{questao}
  Sejam $C_1$ uma circunferência, $AB$ um de seus
  diâmetros, $t$ sua tangente em $B$ e $M$ um ponto de $C_1$
  distinto de $A$. Constrói-se uma circunferência $C_2$ tangente a
  $C_1$ em $M$ e à reta $t$.

  \begin{itemize}
    \item Determinar o ponto $P$ de tangência de $t$ e $C_2$ e encontrar o lugar
    geométrico dos centros das circunferências $C_2$ ao variarmos $M$.

    \item Demonstre que existe uma circunferência ortogonal a todas as
    circunferências $C_2$.
  \end{itemize}
\end{questao}

\begin{questao}
  Seja $ABC$ um triângulo e $S$ uma circunferência
  tangente aos lados $CA,CB$ em $D,E$ respectivamente, e tangente
  internamente ao circuncírculo do triângulo $ABC$. Mostre que o
  incentro do triângulo $ABC$ é ponto médio de $DE$.
\end{questao}

\begin{questao}
  Se $A,B,C,D$ são quatro pontos coplanares, então $AB
  \cdot CD + AD \cdot BC \geq AC \cdot BD $, com igualdade se, e
  somente se, $A,B,C,D$ são colineares ou concíclicos.
\end{questao}

\begin{questao}
  Duas circunferências$C_1,C_2$ com centros $O_1,O_2$
  são tangentes externamente. A reta $\ell$ toca ambas as
  circunferências em pontos distintos $A$ e $B$. Construa uma
  circunferência tangente às duas circunferências e à reta $\ell$.
\end{questao}

\begin{questao}
  Duas circunferências $C_1,C_2$ se interceptam nos pontos
  $A,B$. Um ponto $C$, diferente de $A$ e $B$ e exterior às
  circunferências $C_1$ e $C_2$, é tomado sobre a reta
  $AB$. Construa uma circunferência que passe por $C$ e é tangente
  a $C_1$ e $C_2$.
\end{questao}

\begin{questao}
  O incírculo de um triângulo $ABC$ toca os lados
  $BC,CA,AB$ nos pontos $A_1,B_1,C_1$ respectivamente. Sejam
  $A_0,B_0,C_0$ os pontos médios de $B_1C_1,C_1A_1,A_1B_1$
  respectivamente. Prove que o incentro de $ABC$, o circuncentro de
  $ABC$ e o circuncentro de $A_1B_1C_1$ são colineares.
\end{questao}

\begin{questao}
  $N$ e $S$ são dois pontos diametralmente opostos de
  uma circunferência $C$, $\ell$ é uma reta tangente a $C$ no
  ponto $S$. A partir de um ponto $O$ exterior a $C$ e não
  pertencente à reta tangente a $C$ em $N$, traçamos tangentes
  $OA$ e $OB$ ($A$ e $B$ são os pontos de tangência). Sejam
  $O', A', B'$ as projeções, a partir de $N$,
  dos pontos $O,A,B$ sobre a reta $\ell$ (figura a seguir). Prove
  que $O'$ é ponto médio de $A' B'$.
\end{questao}

\begin{questao}
  Sejam $X,Y,Z$ três pontos colineares (veja a figura a
  seguir), tais que $Y$ está entre $X$ e $Z$, e sejam
  $C,C_1,C_0$ semicircunferências, todas sobre o mesmo lado de
  $XZ$, e tendo $XZ,XY,YZ$ como diâmetros,
  respectivamente. $K_1,K_2, \ldots$ denotam circunferências
  tangentes a $C$ e $C_1$, com $K_1$ tangente a $K_0$, $K_2$
  a $K_1$, $K_3$ a $K_2$, e assim por diante. Se o raio de
  $K_n$ é $r_n$, prove que a distância do centro de $k_n$ até
  $XZ$ será $2nr_n$.
\end{questao}

\begin{questao}
  Uma circunferência $C_0$ de raio 1km é tangente a uma
  reta $\ell$ em $Z$. Uma circunferência $C_1$ de raio 1mm é
  tangente a $C_0$ e a $\ell$, e está à direita de $C_0$. Uma
  família de círculos $C_i$ é construído de forma que $C_i$ é
  tangente a $C_0$, a $\ell$ e a $C_{i-1}$, além de estar à
  direita de todas essas circunferências. Quantas circunferências
  constituirão essa família?
\end{questao}

\begin{questao}
  Prove que o Círculo dos Nove pontos do $\triangle ABC$ é
  tangente ao incírculo e aos três excírculos do $\triangle ABC$ (os
  pontos de tangência são conhecidos como {\it pontos de Feuerbach}.)
\end{questao}

\begin{questao}
  ({\it Corolário de Steiner}) Se duas circunferências
  $C_1,C_2$ são tais que podemos construir uma cadeia de $n$
  circunferências tangentes (figura a seguir), partindo de uma certa
  circunferência, então pode-se construir uma tal cadeia de
  $n$ circunferências a partir de qualquer circunferência que seja
  tangente a $C_1$ e $C_2$.
\end{questao}

\begin{questao}
  ({\it Problema de Apolônio}) Construa uma circunferência
  tangente a três circunferências dadas $C_1,C_2,C_3$.
\end{questao}

\begin{questao}
  Mostre que em um quadrilátero bicêntrico os centros dos
  círculos associados e o ponto de intersecção das diagonais são
  colineares.
\end{questao}

\begin{questao}
  Considere o triângulo $ABC$, seu círculo circunscrito
  $\Gamma$ e seu circuncentro $O$. Sejam $AA',
  BB',CC'$ diâmetros e $A'',B'',C''$
  os simétricos dos pontos $A,B,C$ em relação às mediatrizes dos
  lados $BC,CA,AB$ respectivamente. Prove que os circuncírculos dos
  triângulos $(O A' A''),(O B' B''),(O
  C' C'')$ possuem outro ponto em comum diferente de
  $O$.
\end{questao}

\begin{questao}
  Seja $P$ um ponto qualquer interior ao círculo
  $\Gamma$. Tracemos por $P$ duas cordas quaisquer
  $AA',BB'$ perpendiculares e pelo mesmo ponto tracemos
  $PC$ perpendicular a $AB, C \in AB$.
  \begin{itemize}
    \item Prove que a reta $PC$ passa pelo ponto médio $I$ de
    $A' B'$.

    \item Prove que o produto $PC \cdot PI$ permanece constante
    enquanto as cordas variam.
  \end{itemize}
\end{questao}

\begin{questao}
  Considere dois círculos $\Gamma$ e $\Gamma_1$
  tangentes internamente no ponto $O$. Por um ponto $A$ do círculo
  $\Gamma$ traçamos uma tangente que corta o círculo $\Gamma_1$
  nos pontos $B,C$. Prove que a reta $OA$ é bissetriz do ângulo
  $BOC$.
\end{questao}

\begin{questao}
  Seja $\Gamma$ uma circunferência qualquer e dois pontos
  $A,B$ não pertencentes a $\Gamma$. Pelo ponto $B$ tracemos uma
  secante qualquer $t$ que corta $\Gamma$ nos pontos $C,D$. As
  retas $AC,AD$ cortam $\Gamma$ nos pontos $E,F$
  respectivamente. Determine o lugar geométrico do circuncentro do
  triângulo $AEF$.
\end{questao}

\begin{questao}
  No triângulo $ABC$, sejam $I$ o incentro e $D,E,F$
  os seus pontos de tangência com os lados $BC,CA,AB$
  respectivamente. Seja $P$ o outro de intersecção da reta $AD$
  com a circunferência. Se $M$ for o ponto médio de $EF$,
  demonstre que os quatro pontos $P,M,I,D$ pertencerão a uma mesma
  circunferência ou estarão alinhados.
\end{questao}

\begin{questao}
  $P$ é um ponto interior ao triângulo $ABC$. Determine
  uma reta através de $P$ que corta $AB$ em $M$ e $AC$ em
  $N$ tal que $\frac{1}{PM}+\frac{1}{PN}$ seja máximo.
\end{questao}

\begin{questao}
  Os pontos $A,B,C,D$ estão, nesta ordem, sobre uma mesma
  reta. Um círculo $\Gamma$ passa pelos pontos $B,C$ e
  $AM,AN,DK,DL$ são tangentes a $\Gamma$.
  \begin{itemize}
    \item Prove que os pontos $P=MN \cap BC,Q=KL \cap BV$ não
    dependem de $\Gamma$.

    \item Se $AD=a,BD=b\, (a>b)$ e o segmento $BC$ varia ao longo
    de $AD$, encontre o comprimento mínimo do segmento $PQ$.
  \end{itemize}
\end{questao}

\begin{questao}
  Seja $M_a,M_b,M_c$ os pontos médios dos lados
  $BC,CA,AB$ do $\triangle ABC$ e $H_a,H_b,H_c$ os pés das
  alturas do triângulo $M_aM_bM_c$. Prove que o incentro de
  $H_aH_bH_c$ e os circuncentros de $ABC,M_aM_bM_c,H_aH_bH_c$ são
  colineares.
\end{questao}

\begin{questao}
  Sejam $ABC$ um triângulo acutângulo e $D$ um ponto
  dentro do triângulo tais que

  \begin{eqnarray*}
    AC \cdot BD & = & AD \cdot BC \\
    \angle ADB & = & \angle ACB + 90^\circ
  \end{eqnarray*}

  \begin{itemize}
    \item Calcule $$ \frac{AB \cdot CD}{AC \cdot BD} $$

    \item Prove que as tangentes às circunferências circunscritas aos
    triângulos $ACD$ e $BCD$, traçadas pelo ponto $C$, são
    perpendiculares.
  \end{itemize}
\end{questao}

\begin{questao}
  Seja $A$ um dos semiplanos abertos que uma reta $r$
  determina sobre um plano $\Omega$. Esse semiplano, com a seguinte
  alteração nas definições usuais, chama-se plano H (plano de
  Poincaré): reta de H é a intersecção com $A$ de:
  \begin{itemize}
    \item toda circunferência de $\Omega$ com centro em $r$;

    \item toda reta de $\Omega$ perpendicular a $r$.
  \end{itemize}

  Prove que, em H:
  \begin{itemize}
    \item Por um ponto fora de uma reta passam infinitas paralelas à
    reta;

    \item A soma dos ângulos internos de um triângulo varia entre
    $0$ e $\pi$.
  \end{itemize}
\end{questao}

\begin{questao}
  É dada uma circunferência $k$ cd centro $M$ e raio
  $r$. Sejam $AB$ um diâmetro de $k$ e $K$ um ponto fixado do
  segmento $AM$. Denotamos por $t$ a reta tangente a $k$ em
  $A$. Para cada corda $CD (\not = AB)$ passando pelo ponto $K$
  traçamos as retas $BC,BD$, as quais cortam $t$ nos pontos
  $P,Q$ respectivamente. Prove que o produto $AP \cdot AQ$
  permanece constante enquanto variamos $CD$.
\end{questao}

\begin{questao}
  Dizemos que uma circunferência $S$ corta uma
  circunferência $\Sigma$ {\it diametralmente} se, e somente se, a
  corda comum a ambas é um diâmetro de $S$.\\
  Sejam $S_A,S_B,S_C$ três circunferências com centros distintos
  $A,B,C$ respectivamente.
  \begin{itemize}
    \item Prove que $A,B,C$
    são colineares se, e somente se, não existe nenhuma circunferência
    $S$ que corta $S_A,S_B,S_C$ diametralmente ou tal
    circunferência não é única.

    \item Prove ainda que se existe mais de uma circunferência $S$
    que corta $S_A,S_B,S_C$ diametralmente, então tais
    circunferências $S$ passam através de dois pontos fixos.

    \item Localize tais pontos em relação às circunferências
    $S_A,S_B,S_C$.
  \end{itemize}
\end{questao}

\begin{questao}
  Seja $ABC$ um triângulo acutângulo. A circunferência com
  diâmetro $AB$ intercepta a altura $CC'$ e sua extensão nos
  pontos $M,N$, e a circunferência de diâmetro $AC$ intercepta a
  altura $BB'$ e sua extensão nos pontos $P,Q$. Prove que os
  pontos $M,N,P,Q$ estão situados sobre uma mesma circunferência.
\end{questao}

\begin{questao}
  Dada uma semicircunferência de diâmetro $AB=2R$, e uma
  circunferência $S$ variável, tangente em $M$ à
  semicircunferência e em $N$ ao segmento $AB$. A circunferência
  $S$ corta $MA$ em $C$ e $MB$ em $D$. Prove que

  \begin{itemize}    
    \item $CD$ é paralelo a $AB$, $MN$ é bissetriz do arco
    $AMB$ e $MN$ passa por um ponto fixo $I$;
    \item A circunferência $AMN$ é tangente em $A$ a $AI$ e
    calcule $IM \cdot IN$ em função de $R$;

    \item O eixo radical de duas quaisquer destas circunferências
    $S$ passa por $I$.
  \end{itemize}
\end{questao}

\begin{questao}
  Seja $P$ um ponto pertencente ao interior do ângulo
  $\angle AOB$. Determine $X$ sobre $OA$ e  $Y$ sobre $OB$
  tais que $P$ pertença ao segmento $XY$ e o produto $PX \cdot
  PY$ seja máximo.
\end{questao}

\begin{questao}
  Duas circunferências se interceptam nos pontos
  $P,Q$. Mostre como construir um segmento $AB$ através de $P$ e
  com extremidades sobre as duas circunferências, tal que $PA \cdot
  PB$ seja máximo.
\end{questao}

\begin{questao}
  Seja $P$ um ponto pertencente ao interior da região
  $R$ comum aos círculos $C_1,C_2$. Seja $UV$ uma corda de $R$
  através de $P$ (veja a figura). Determine como construir a corda
  que minimiza o produto $PU \cdot PV$.
\end{questao}

\begin{questao}
  Na figura a seguir, as retas $AW$ e $BC$ são tangentes
  a ambas as circunferências, assim como o arco $BAC$. Prove que
  $W$ é o incentro do triângulo ABC.
\end{questao}

\begin{questao}
  Sejam uma circunferência de centro $O$ e raio $R$, uma
  reta $\Delta$ exterior e perpendicular em $A$ a um raio
  $OA$. Seja $M$ um ponto que se move sobre $\Delta$. Os pontos
  de contato das tangentes traçadas a partir de $M$ são $P$ e
  $P'$. A corda $PP'$ corta $OM$ em $N$ e $OA$ em $B$.
  \begin{itemize}
    \item Demonstre que $OA \cdot OB = OM \cdot ON = R^2$. Deduza
    que  $B$ está fixo e determine o lugar geométrico de $N$.

    \item Determine o lugar geométrico do centro da circunferência
    inscrita no triângulo $MPP'$.

    \item Determine o lugar geométrico do ortocentro do triângulo
    $MPP'$.
  \end{itemize}
\end{questao}

\begin{questao}
  Uma bola movimenta-se com reflexão perfeita sobre uma mesa
  de bilhar cujo bordo é uma circunferência. Prove que uma condição
  necessária e suficiente para que a bola passe infinitas vezes por um
  mesmo ponto é que ela passe três vezes por esse ponto.
\end{questao}

\begin{questao}
  ({\it A Borboleta}) Através do ponto médio $M$ da corda
  $PQ$ de uma circunferência, traçamos cordas $AB,CD$; as
  cordas $AD,BC$ encontram $PQ$ nos pontos $X,Y$. Prove que
  $M$ é ponto médio de $XY$.
\end{questao}

\begin{questao}
  Prove que na figura anterior as retas $AC$ e $BD$
  interceptam $PQ$ em pontos equidistantes de $M$.
\end{questao}

\begin{questao}
  Sejam $PT$ e $PB$ retas tangentes a uma
  circunferência, $AB$ o diâmetro através de $B$, e $TH$ uma
  perpendicular ao segmento $AB$ em $T$. Prove que $AP$ bissecta
  $TH$.
\end{questao}

\begin{questao}
  Suponha que o incírculo (com centro $I$) do $\triangle
  ABC$ toque o lado $BC$ em $X$ e seja $A'$ o ponto médio deste
  lado. Prove que $AI$ bissecta $AX$.
\end{questao}

\begin{questao}
  Seja $ABC$ um triângulo de lados $AB=c,BC=a,CA=b$,
  onde $a<b<c$ e $a,b,c$ estão em progressão aritmética. Sabendo
  que o produto $a \cdot c = 294cm^2$, determine o menor valor que a
  expressão $H=R^2+r^2$ pode assumir, em que $R$ é o circunraio e
  $r$ é o inraio de $ABC$.
\end{questao}

\begin{questao}
  ({\it Fórmula de Euler}) Sejam $O,I$ o circuncentro e o
  incentro de um triângulo com circunraio $R$ e inraio $r$. Prove
  que $OI^2=R^2-2Rr$.
\end{questao}

\begin{questao}
  Uma reta através do incentro $I$ do triângulo $ABC$
  intercepta o circuncírculo nos pontos $F,G$, e o incírculo nos
  pontos $D,E$ com $D$ entre $I$ e $F$. Prove que $DF \cdot
  EG \geq r^2$, onde $r$ é o inraio. Quando ocorre a igualdade?
\end{questao}

%%% Local Variables:
%%% mode: latex
%%% coding: utf-8-unix
%%% fill-column: 80
%%% TeX-master: "MASTER"
%%% End:
