\begin{questao}
  No triângulo $ABC$, $\angle C = 100^\circ$. Seja $D$ um ponto
  pertencente a $AC$ tal que $CD=BC$. Se $BD=AC$, calcule
  $\angle A$.
\end{questao}

\begin{questao}
  Em um triângulo $ABC$ as medidas dos lados $BC,CA,AB$ formam,
  nesta ordem, uma progressão aritmética.

  Demonstre que 
  $$ \tan\frac{\angle A}{2} \cdot \tan\frac{\angle C}{2} =
  \frac{1}{2} $$
\end{questao}

\begin{questao}
  As alturas de um triângulo acutângulo $ABC$, onde $AB>AC$,
  intersectam os lados $BC,CA,AB$ nos pontos $D,E,F$
  respectivamente. Se $EF$ intersecta $BC$ no ponto $P$ e a reta
  que passa por $D$ e é paralela a $EF$ intersecta $AC$ e $AB$
  em $Q$ e $R$ respectivamente, seja $N$ um ponto sobre o lado
  $BC$ tal que $\angle NQP + \angle NRP < 180^\circ$.

  Prove que $BN>CN$.
\end{questao}

\begin{questao}
  Seja $\triangle ABC$ isósceles de base $BC$. Sabendo que
  $\angle A = 20^\circ, \angle DBC=60^\circ, \angle ECB = 50^\circ$,
  sendo $EC,BD$ cevianas, determine $\angle BDE$.
\end{questao}

\begin{questao}
  No triângulo $ABC$, seja $T$ o pé da bissetriz interna por
  $A$, e $M,N$ pontos médios de $AB,AC$ respectivamente.

  Prove que $$ \frac{1}{\tan\angle ATM} + \frac{1}{\tan\angle ATN} =
  \frac{2}{\tan\frac{\angle ATM}{2}} $$
\end{questao}

\begin{questao}
  Um quadrilátero convexo está inscrito numa circunferência de
  raio unitário. Demonstre que a diferença a diferença entre seu
  perímetro e a soma das diagonais é maior do que zero e menor do que
  dois.
\end{questao}

\begin{questao}
  Dado um quadrilátero $ABCD$, o círculo que tangencia $AD,DC,CB$
  encontra esses lados em $K,L,M$ respectivamente. Seja $N$ a
  intersecção de $KM$ e a reta por $L$ que é paralela a $AD$, e
  seja $P$ a intersecção de $LN$ e $KC$.

  Mostre que $PL=PN$.
\end{questao}

\begin{questao}
  As retas tangentes ao circuncírculo do triângulo acutângulo $ABC$
  através de $B$ e $A$ cortam a tangente através de $C$ em $T$
  e $U$ respectivamente. $AT$ corta $BC$ em $P$, e $Q$ é
  ponto de $AP$. $BU$ corta $CA$ em $R$, $S$ é ponto médio
  de $BR$.

  Prove que $\angle ABQ = \angle BAS$.
\end{questao}



\begin{questao}
  A circunferência inscrita no triângulo $ABC$ é tangente aos lados
  $BC,CA,AB$ nos pontos $D,E,F$ respectivamente. $AD$ corta a
  circunferência num segundo ponto $Q$.

  Demonstre que a reta $EQ$ passa pelo ponto médio de $AF$ se, e
  somente se, $AC=BC$.
\end{questao}

\begin{questao}
  Uma circunferência de centro $O$ está inscrita no quadrilátero
  $ABCD$. Mostre que se os perímetros dos triângulos $AOB, BOC,
  COD$ são iguais, então $ABCD$ é um losango.
\end{questao}

\begin{questao}
  O prolongamento da bissetriz $AL$ do triângulo acutângulo
  $ABC$ intercepta a circunferência circunscrita no ponto $N$. A
  partir do ponto $L$ traçam-se as perpendiculares $LK$ e $LM$ aos
  lados $AB$ e $BC$ respectivamente. Prove que a área do triângulo
  $ABC$ é igual à área do quadrilátero $AKNM$.

\end{questao}

\begin{questao}
  As tangentes a uma circunferência de centro $O$,
  traçadas por um ponto exterior $C$, tocam a circunferência nos
  pontos $A$ e $B$. Seja $S$ um ponto qualquer da
  circunferência. As retas $SA,SB,SC$ cortam o diâmetro perpendicular
  a $OS$ nos pontos $A',B',C'$
  respectivamente. 

  Prove que $C'$ é ponto médio de $A'B'$.

\end{questao}

\begin{questao}
  Dado o triângulo $ABC$, mostre como construir com régua
  e compasso um triângulo $A' B' C'$ de área mínima
  com $C' \in AC$, $A' \in AB$ e $B' \in BC$ tal
  que $\widehat{B' A' C'} = \widehat{BAC}$ e
  $\widehat{A' C' B'} = \widehat{ACB}$.

\end{questao}

\begin{questao}
  Considere um triângulo $ABC$ com $\angle B = 90^\circ$. Seja
  $D$ um ponto do prolongamento de $AC$ tal que $\angle CBD =
  30^\circ$.

  Sabe-se que $AB=CD=1$. Determine a medida de $AC$.
\end{questao}

\begin{questao}
  Em um triângulo $ABC$, o ângulo $\angle BCA$ é obtuso e
  $\angle BAC = 2 \angle ABC$. A reta por $B$ perpendicular a $BC$
  intercepta $AC$ em $D$. Seja $M$ o ponto médio de $AB$. Prove
  que $\angle AMC = \angle BMD$.

\end{questao}

\begin{questao}
  Seja $M$ um ponto no interior do triângulo $ABC$ tal que
  $\angle AMC = 90^\circ$, $\angle AMB = 150^\circ$ e $\angle BMC =
  120^\circ$. Os circuncentros dos triângulos $AMC$, $AMB$ e
  $BMC$ são $P$, $Q$ e $R$ respectivamente. Prove que a área do
  triângulo $PQR$ é maior que a área do triângulo $ABC$.

\end{questao}

\begin{questao}
  Seja $ABC$ um triângulo e $A',B',C'$
  pontos médios dos arcos $BC,CA,AB$ do circuncírculo de $ABC$,
  respectivamente. As retas $A' B'$ e $A'
  C'$ interceptam o lado $BC$ em $M$ e $N$,
  respectivamente. Defina os pares de pontos $P,Q$ e $R,S$
  analogamente. Prove que $MN=PQ=RS$ se, e somente se, $ABC$ é
  equilátero.

\end{questao}

\begin{questao}
  No triângulo $ABC$, com $\angle BAC = 60^\circ$,
  construímos uma paralela $IF$ ao lado $AC$, onde $F$ está sobre
  $AB$ e $I$ é o incentro do $\bigtriangleup ABC$. O ponto $P$
  sobre $BC$ é tal que $3BP=BC$. Mostre que $\angle BFP = \angle
  ABC/2$.

\end{questao}

\begin{questao}
  Sejam $M,N$ pontos sobre os lados $AB,BC$ do paralelogramo
  $ABCD$ tais que $AM=NC$. As retas $AM,CN$ se encontram no
  ponto $Q$. Mostre que $DQ$ bi-secta o ângulo $D$ do
  paralelogramo.
\end{questao}

\begin{questao}
  São dados um ângulo de vértice $O$ e um ponto $A$ interior ao
  ângulo. Os pontos $M,N$ pertencem aos lados do ângulo e são tais
  que $\angle OAM= \angle OAN$. Mostre que todas as retas $MN$
  nessas condições passam por um mesmo ponto ou são paralelas entre si.
\end{questao}

\begin{questao}
  Considere o triângulo $ABC$. Constrói-se um quadrado exterior a
  $ABC$ tal que $AB$ é um de seus lados. Seja $O$ o centro do
  quadrado e $M,N$ pontos médios de $BC,AC$ respectivamente. Dados
  $BC=a,AC=b$, encontre o valor máximo de $OM+ON$ quando varia o
  ângulo $\angle C$.
\end{questao}

\begin{questao}
  Sobre os lados de um triângulo $ABC$, são construídos,
  externamente, triângulos $ABR,BCP,CAQ$ com $\angle CBP = \angle CAQ
  = 45^\circ, \angle BCP = \angle ACQ = 30^\circ, \angle ABR = \angle
  BAR = 15^\circ$. Prove que $\angle QRP = 90^\circ$ e $QR=RP$.

\end{questao}

\begin{questao}
  Seja $O,O_a$ os centros do incírculo e do excírculo
  tangente ao lado $BC$ do triângulo $ABC$. A mediatriz do segmento
  $OO_a$ intercepta as retas $AB,AC$ nos pontos $L,N$,
  respectivamente. Sabendo que o circuncírculo do $\bigtriangleup ABC$
  tangencia a reta $LN$, prove que $\bigtriangleup ABC$ é isósceles.

\end{questao}

\begin{questao}
  Inscrevemos um triângulo $ABC$ em uma circunferência de
  raio $1$. Seja $I$ o incentro do $\bigtriangleup ABC$.
  Prove que se $IA \cdot IB \cdot IC = 1$, então o triângulo $ABC$ é
  equilátero.

\end{questao}

\begin{questao}
  No triângulo retângulo $ABC$, temos $\angle A=90^\circ$ e
  $\angle B=60^\circ$. $T$ é ponto médio de $AB$ e $P,Q$ são
  pontos dos lados $AC,BC$ tais que $PQT$ é equilátero.

  Dado que $AB=4$, calcule $PQ$.
\end{questao}

\begin{questao}
  Calcule a área de um hexágono equilátero inscrito em uma
  semicircunferência de raio $1$.

\end{questao}

\begin{questao}
  Seja $ABC$ um triângulo acutângulo com circuncentro
  $O$. Seja $PA$ uma altura do triângulo com $P$ no lado
  $BC$. Considere que $\angle BCA \geq \angle ABC+30^\circ$.
  Prove que $\angle CAB + \angle COP < 90^\circ$.

\end{questao}

\begin{questao}
  Em um triângulo $ABC$, a bissetriz do ângulo $A$ intersecta o
  lado $BC$ no ponto $A_1$ e o círculo circunscrito ao triângulo
  $ABC$ nos pontos $A,A_2$. Analogamente se definem os pontos
  $B_1, B_2, C_1,C_2$.

  Prove que $$ \frac{AA_1}{BA_2+CA_2} + \frac{BB_1}{CB_2+AB_2}
  +\frac{CC_1}{AC_2+BC_2} \geq \frac{3}{4} $$
\end{questao}

\begin{questao}
  Seja $\Gamma$ uma circunferência de centro $O$
  tangente aos lados $AB$ e $AC$ do triângulo $ABC$ nos pontos
  $E$ e $F$. A reta perpendicular ao lado $BC$ por $O$
  intercepta $F$ no ponto $D$. Prove que $A,D$ e $M$ (ponto
  médio de $BC$) são colineares.

\end{questao}

\begin{questao}
  Sejam $AH_a,BH_b,CH_c$ as alturas de um triângulo
  acutângulo $ABC$.
  A circunferência inscrita no triângulo $ABC$ é
  tangente aos lados $BC,CA,AB$ nos pontos $T_a,T_b,T_c$,
  respectivamente.
  Seja $\ell_a$ a reta simétrica da reta $H_bH_c$ relativamente à
  reta $T_bT_c$, $\ell_b$ a reta simétrica da reta $H_cH_a$ relativamente à
  reta $T_cT_a$, $\ell_c$ a reta simétrica da reta $H_aH_b$ relativamente à
  reta $T_aT_b$.
  Prove que $\ell_a,\ell_b,\ell_c$ determinam um triângulo cujos
  vértices pertencem à circunferência inscrita no triângulo $ABC$.

\end{questao}

\begin{questao}
  Num triângulo $ABC$, seja $AP$ a bissetriz interna de $\angle
  BAC$ com $P$ no lado $BC$, e seja $BQ$ a bissetriz interna de
  $\angle BAC$ com $Q$ no lado $CA$.
  Sabemos que $\angle BAC = 60^\circ$ e $AB+BP = AQ+QB$.

  Quais os possíveis valores dos ângulos do triângulo $ABC$?
\end{questao}

%%% Local Variables:
%%% mode: latex
%%% coding: utf-8-unix
%%% fill-column: 80
%%% TeX-master: "MASTER"
%%% End:
