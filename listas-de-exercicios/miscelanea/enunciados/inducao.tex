\begin{questao}
  A função $f(x,y)$ satisfaz

  \begin{enumerate}

  \item $f(0,y) = y+1$;

  \item $f(x+1,0) = f(x,1)$;

  \item $f(x+1,y+1) = f(x,f(x+1,y))$ para todo $x,y \in \mathbb{N}$.
  \end{enumerate}

  Determine $f(4,1981)$.
\end{questao}

\begin{questao}
  \begin{enumerate}

  \item O plano euclidiano é dividido em regiões por um número finito de
    retas. Mostre que é possível colorir cada uma dessas regiões de vermelho ou
    azul de modo que duas regiões adjacentes nunca tenham a mesma cor.

  \item O mesmo problema para o plano dividido em circunferências.
  \end{enumerate}
\end{questao}

\begin{questao}
  Um grupo de $n$ pessoas disputa um torneio, no qual todos jogam contra todos e
  cada partida tem um vencedor. Mostre que é possível colocar os $n$ jogadores
  numa fila, na qual, o primeiro derrotou o segundo, o segundo derrotou o
  terceiro, \ldots, o $(n-1)$-ésimo derrotou o $n$-ésimo.
\end{questao}

\begin{questao}
  Se cada pessoa, num grupo de $n$ pessoas, é amiga de pelo menos metade das
  pessoas do grupo, prove que é possível que as $n$ pessoas sentem-se em
  círculo, de tal forma que toda pessoa tenha como vizinhos apenas amigos.
\end{questao}

\begin{questao}
  Prove que

  $$ \frac{4^n}{n+1} < \binom{2n}{n} $$

  para todo número inteiro $n>1$.
\end{questao}

\begin{questao}
  Seja $S$ um conjunto de $n$ inteiros positivos ímpares $a_1 < a_2 < \ldots <
  a_n$ tais que não existam duas diferenças $|a_i-a_j|$ iguais. Prove que

  $$ \sum_{i=1}^{n}{a_i} \geq \frac{n(n^2+2)}{3} $$
\end{questao}

\begin{questao}
  Sejam $0<a_1<a_2<\ldots<a_n$ e $e_i= \pm 1$. Prove que $\sum_{i=1}^{n}{e_i
    a_i}$ assume pelo menos $\binom{n+1}{2}$ valores distintos quando $e_i$
  varia sobre as $2^n$ possíveis combinações de sinais.
\end{questao}

\begin{questao}
  Seja $n$ um inteiro positivo. Mostre que existe $n$ inteiro positivo tal que
  $(\sqrt{2}+1)^n = \sqrt{m}+\sqrt{m-1}$.
\end{questao}

\begin{questao}
  Sejam $n$ um inteiro positivo, e $a_i \geq 1$ para $i=1,2,\ldots,n$. Mostre
  que

  $$(1+a_1)(1+a_2)\ldots(1+a_n) \geq \frac{2^n}{n+1}(1+a_1+a_2+\ldots+a_n)$$
\end{questao}

\begin{questao}
  Sejam $n,k$ inteiros positivos com $k \leq n$ e seja $S$ um conjunto contendo
  $n$ números reais distintos. Seja $T$ o conjunto de todos os números reais da
  forma $x_1+x_2+\ldots+x_k$, onde $x_1,x_2,\ldots,x_k$ são elementos distintos
  de $S$. Prove que $T$ contém pelo menos $k(n-k)+1$ elementos distintos.
\end{questao}

\begin{questao}
  $N$ sinais luminosos $s_1,s_2,\ldots,s_n$ igualmente espaçados estão
  distribuídos ao longo de uma ferrovia. Como uma regra de segurança, um trem
  não pode ultrapassar o sinal $s_i$ se outro trem está no trajeto entre $s_i$ e
  $s_{i+1}$. (Assuma que os trens têm comprimento zero.)

  Uma série de $K$ trens de carga deve ser levada de $s_1$ até $s_N$. Cada trem
  trafega em uma velocidade distinta, porém constante, sempre que não está
  bloqueado pela regra de segurança. Mostre que o tempo decorrido entre a
  partida do primeiro trem de $s_1$ e a chegada do último trem a $s_N$ é
  independente da ordem na qual os trens estão arranjados.
\end{questao}

\begin{questao}
  Assuma que o conjunto dos números naturais é particionado em $n$ subconjuntos
  disjuntos $A_i,A_1 \cup A_2 \cup \ldots \cup A_n = \mathbb{N}$. Prove que um
  destes conjuntos, digamos $A_i$, tenha a seguinte propriedade: existe um
  inteiro $m$ tal que para todo $k$ podemos encontrar naturais
  $a_0,a_1,a_2,\ldots,a_{k-1} \in A_i$ com $0 \leq a_{j+1}-a_j \leq m$ para $0
  \leq j < k-1$.
\end{questao}

\begin{questao}
  Prove que se retirarmos uma casa de um tabuleiro $2^n \times 2^n$, podemos
  então cobri-lo com {\it L-trominós}.
\end{questao}

\begin{questao}
  Um conjunto de $2^n$ objetos é particionado em um número arbitrário de
  subconjuntos não vazios. Então podemos realizar as seguintes operações sobre
  tais conjuntos: escolhidos dois subconjuntos, comparamos suas cardinalidades e

  \begin{enumerate}

  \item Se são iguais, unimos tais conjuntos;

  \item Caso contrário, a transferência é feita do de maior cardinalidade para o
    de menor, de tal forma que a cardinalidade deste último dobre.
  \end{enumerate}

  Prove que é possível combinar todos os subconjuntos em um único conjunto.
\end{questao}

\begin{questao}
  É dado no plano um conjunto finito $E$ de pontos de coordenadas inteiras. É
  sempre possível colorir todos os pontos de $E$ com duas cores, vermelho ou
  branco, de modo que para toda reta $r$ paralela, quer ao primeiro, quer ao
  segundo eixo coordenado, a diferença entre o número de pontos vermelhos e o
  número de número de pontos brancos, pertencentes a $r$, seja 1, 0 ou -1?
  Justifique.
\end{questao}

\begin{questao}
  Encontre todas as funções $f$ que satisfazem
  \begin{enumerate}

  \item $f(x,x) = x$;

  \item $f(x,y) = f(y,x)$;

  \item $f(x,y) = f(x,x+y)$;
  \end{enumerate}

  assumindo que as variáveis e os valores de $f$ sejam inteiros positivos.
\end{questao}

\begin{questao}
  Mostre que todo inteiro positivo pode ser escrito como a soma de números de
  Fibonacci distintos.
\end{questao}

\begin{questao}
  Determine o valor máximo de $m^2+n^2$, onde $m$ e $n$ são inteiros
  satisfazendo $m,n \in \{1,2,\ldots,1981\}$ e $(n^2-mn-m^2)^2 = 1$.
\end{questao}

\begin{questao}
  Suponha que uma função $f$ definida nos inteiros positivos satisfaça

  \begin{eqnarray*}
    f(1) & = & 1 \\
    f(2) & = & 2 \\
    f(n+2) & = & f(n+2-f(n+1)) + f(n+1-f(n))
  \end{eqnarray*}

  \begin{enumerate}

  \item Mostre que
    \begin{enumerate}

    \item $0 \leq f(n+1) - f(n) \leq 1$;

    \item Se $f(n)$ é ímpar, então $f(n+1) = f(n) + 1$;
    \end{enumerate}

  \item Determine todos os valores de $n$ para os quais $f(n) = 2^{10}+1$.
  \end{enumerate}
\end{questao}

\begin{questao}
  Prove que a equação $x^4+y^4=z^4$ não tem solução no conjunto dos inteiros
  positivos.
\end{questao}

\begin{questao}
  Seja $f(x)=x^2+x+p, p \in \mathbb{N}$. Prove que se todos os números $f(0),
  f(1), \ldots,f(\lfloor \sqrt{p/3} \rfloor)$ são primos, então os números
  $f(0),f(1), \ldots,f(p-2)$ são primos.
\end{questao}

\begin{questao}
  Sejam $a$ e $b$ números inteiros positivos tais que $ab+1$ divide
  $a^2+b^2$. Prove que $\frac{a^2+b^2}{ab+1}$ é um quadrado perfeito.
\end{questao}

\begin{questao}
  Para quais pares de inteiros $a$ e $b$, $ab$ divide $a^2+b^2+1$?
\end{questao}

\begin{questao}
  Uma certa organização tem $n$ membros, e $n+1$ comitês distintos de três
  membros. Prove que há dois comitês com exatamente um membro em comum.
\end{questao}

\begin{questao}
  Encontre todas as funções $f: Z \rightarrow Z$ tais que $f(f(n))+f(n) = 2n+3,
  \forall n \in \mathbb{Z}$ e $f(0) = 1$.
\end{questao}

\begin{questao}
  Seja $M$ um conjunto não vazio de inteiros positivos tal que se $x$ pertence a
  $M$, então $4x$ e $\lfloor \sqrt{x} \rfloor$ pertencem a $M$. Prove que $M$ é
  o conjunto dos inteiros positivos.
\end{questao}

\begin{questao}
  Se $p$ é um polinômio de coeficientes inteiros, denotamos por $w(p)$ o número
  de coeficientes ímpares de $p$. Para todo natural $i$, seja
  $Q_i=(1+x)^i$. Mostre que, para toda família finita de inteiros
  $i_1,i_2,\ldots,i_n$ tais que

  $$ 0 \leq i_1 < i_2 < \ldots < i_n $$

  tem-se

  $$ w(Q_{i_1}+Q_{i_2}+\ldots +Q_{i_n}) \geq w(Q_{i_1}) $$
\end{questao}

\begin{questao}
  Determine o menor número natural $n$ tal que, para todo $p,p \geq n$, seja
  possível dividir um quadrado dado em $p$ quadrados (não necessariamente
  iguais).
\end{questao}

\begin{questao}
  Um inteiro $n$ é denominado ``legal'' se podemos escrever

  $$ n = a_1+a_2+\ldots+a_k$$

  onde $a_1,a_2,\ldots,a_k$ são inteiros positivos (não necessariamente
  distintos) que satisfazem

  $$ \frac{1}{a_1}+\frac{1}{a_2}+\ldots+\frac{1}{a_k}=1$$

  Dado que os inteiros entre $33$ e $73$ são legais, prove que todo inteiro
  $\geq 33$ é legal.
\end{questao}

\begin{questao}
  Considere a seguinte operação sobre um natural $a$ qualquer:

  \begin{center}
    $(*)a$ é o natural obtido somando-se os quadrados dos algarismos de $a$.
  \end{center}

  Partindo de um natural $a$, e construindo a sequência
  $a_1,a_2,\ldots,a_n,\ldots$, onde $a_{k+1}$ é obtido de $a_k$ pela operação
  $(*)$, mostre que das duas uma:

  \begin{enumerate}

  \item a partir de um certo $k$,$a_k=1$;

  \item a partir de um certo ponto a sequência fica repetindo os números

    $$ 145,42,20,4,16,37,58,89,145\ldots $$

  \end{enumerate}
\end{questao}

\begin{questao}
  Prove que todo inteiro maior que 128 é uma soma de quadrados distintos.
\end{questao}

\begin{questao}
  Seja $A \subset \mathbb{Z}$ com as seguintes propriedades:
  \begin{enumerate}

  \item $0 \in A$,

  \item Se $n \in A$, então $3n \in A$, $3n+4 \in A$ e $3n+11 \in A$.
  \end{enumerate}

  Prove que, dado $k \in \mathbb{Z}$ existem $a,b \in A$ com $a-b=k$.
\end{questao}

\begin{questao}
  Se $a,b,c \geq 1$, prove que $4(abc+1) \geq (1+a)(1+b)(1+c)$.
\end{questao}

\begin{questao}
  Se $A_1+\ldots+A_n = \pi, 0 < A_i \leq \pi, i=1,\ldots,n$, então

  $$ \sin {A_1} + \ldots + \sin {A_n} \geq n \sin \frac{\pi}{n} $$
\end{questao}

\begin{questao}
  Seja $S$ um quadrado reticular $n \times n, n \geq 3$. Mostre que é possível
  desenhar uma linha poligonal constituída por $2n-2$ segmentos e que cruza
  todos os $n^2$ pontos reticulares de $S$.
\end{questao}

\begin{questao}
  Seja $F_i$ o $i$-ésimo termo da sequência de Fibonacci. Prove que

  $$ F^2_{n+1} + F^2_{n} = F_{2n+1} $$
\end{questao}

\begin{questao}
  Considere a recorrência (Números de Knuth)

  \begin{eqnarray*}
    K_0 & = & 1;\\ K_{n+1} & = & 1 + min(2K_{\lfloor\frac{n}{2}\rfloor},
    3K_{\lfloor\frac{n}{3}\rfloor}), n \geq 0.
  \end{eqnarray*}

  Prove que $K_n \geq n$.
\end{questao}

\begin{questao}
  Quantas permutações dos inteiros $1,2,\ldots,n$ são tais que todo inteiro é
  seguido (mas não necessariamente imediatamente seguido) por um inteiro que
  difere deste inteiro por $1$? Por exemplo, com $n=4$, $1432$ é uma permutação
  aceitável, mas $2431$ não é.
\end{questao}

\begin{questao}
  Dados um triângulo retângulo e um conjunto finito de pontos em seu interior,
  prove que estes pontos podem ser ligados por uma linha poligonal tal que a
  soma dos quadrados dos segmentos que a constituem é menor ou igual ao quadrado
  da hipotenusa.
\end{questao}

\begin{questao}
  Seja $(a,b,c,d \in \mathbb{R})$ $f(x) = \frac{ax+b}{cx+d}$ tal que $f(0) \not
  = 0$ e $f(\ldots(f(0))) = 0$.  Prove que $f(\ldots(f(x))) = x$ para todo $x$
  onde a expressão estiver bem definida.
\end{questao}

\begin{questao}
  Sejam dados $2n+1$ pontos no plano. Construa um $2n+1$-ágono tal que estes
  pontos sejam os pontos médios de seus lados.
\end{questao}

\begin{questao}
  Sejam dados $n$ pontos no plano. Construa um $n$-ágono tal que seus lados
  sejam bases dos triângulos isósceles cujos vértices sejam os $n$ pontos dados
  e cujos ângulos em tais vértices sejam $\alpha_1,\alpha_2,\ldots,\alpha_n$.
\end{questao}

\begin{questao}
  Sejam dados uma circunferência e $n$ pontos do plano. Construa o $n$-ágono
  inscrito na circunferência e cujos lados passam por estes pontos.
\end{questao}

\begin{questao}
  Sejam $l_1,l_2$ duas retas paralelas. Divida o segmento ${AB}$ pertencente à
  reta $l_1$ em $n$ partes iguais usando somente uma régua.
\end{questao}

\begin{questao}
  Determinar o lugar geométrico dos pontos para os quais é constante a soma dos
  quadrados de suas distâncias a $n$ pontos fixos.
\end{questao}

\begin{questao}
  Prove que se $a_1,a_2,\ldots,a_k$ são números reais entre $0$ e $1$ cuja soma
  é maior ou igual a $2^n, n \in \mathbb{Z}_+$, então

  $$ \frac{a_1}{a_1}+\frac{a_2}{a_1+a_2}+\frac{a_3}{a_1+a_2+a_3}
  +\ldots+\frac{a_n}{a_1+a_2+\ldots+a_n} > \frac{n+1}{2} $$
\end{questao}

\begin{questao}
  Em um baralho, algumas cartas estão com a face virada para cima, enquanto as
  outras estão com a face virada para baixo. Pedro pega uma pilha de várias
  cartas consecutivas do baralho, tal que a carta de cima da pilha e a carta de
  baixo estão com a face para cima. Então ele vira toda a pilha e a coloca
  exatamente no mesmo lugar.

  Prove que em algum momento todas as cartas estarão com as faces para baixo,
  não importando como Pedro escolha suas pilhas.
\end{questao}

\begin{questao}
  Mostre que existe um subconjunto $A \subset \{1,2,\ldots,2^{1996}-1\}$
  satisfazendo

  \begin{enumerate}

  \item $1 \in A$ e $2^{1996}-1 \in A$;

  \item Todo elemento de $A$, excetuando-se o número $1$, é a soma de dois
    elementos (não necessariamente distintos) de $A$;

  \item $|A| \geq 2012$
  \end{enumerate}
\end{questao}

\begin{questao}
  Várias diagonais que não se interceptam dividem um polígono convexo em
  triângulos. O número de triângulos adjacentes a cada vértice do polígono é
  escrito ao lado do vértice. É possível reconstruir esta triangularização
  conhecendo-se apenas estes números?
\end{questao}

\begin{questao}
  Determine todas as funções $f: Z^+ \rightarrow Z^+$ satisfazendo
  simultaneamente

  \begin{enumerate}

  \item $f(1)>0$

  \item $f(m^2+n^2) = (f(m))^2 + (f(n))^2$ para todos os inteiros não-negativos
    $m,n$.
  \end{enumerate}
\end{questao}

\begin{questao}
  Seja $n$ um inteiro positivo. Seja $T$ o conjunto de pontos $(x,y)$ no plano
  onde $x,y$ são inteiros não negativos e $x+y<n$. Cada ponto de $T$ é pintado
  de vermelho ou azul. Se um ponto $(x;y)$ é vermelho, então todos os pontos
  $(x';y')$ com $x'<x,y'<y$ também são. Um conjunto X é um conjunto de $n$
  pontos azuis com abscissas todas diferentes, e um conjunto Y é um conjunto de
  $n$ pontos azuis com ordenadas todas diferente.  Prove que o número de
  conjuntos X é igual ao número de conjuntos Y.
\end{questao}

\begin{questao}
  Seja $n$ um inteiro positivo. Uma sequência de $n$ inteiros positivo (não
  necessariamente distintos) é dita {\it cheia} se satisfaz a seguinte condição:
  para cada inteiro positivo $k \geq 2$, se o número $k$ aparece na sequência
  então o número $k-1$ também aparece, e além disso, $k-1$ aparece pela primeira
  vez antes de $k$ aparecer pela última vez.

  Para cada $n$, quantas sequências cheias existem?
\end{questao}

\begin{questao}
  Um mágico tem cem cartões numerados de 1 a 100. Coloca-os em três caixas, uma
  vermelha, uma branca e uma azul, de modo que cada caixa contém pelo menos um
  cartão.

  Uma pessoa da plateia escolhe duas das três caixas, seleciona um cartão de
  cada caixa e anuncia a soma dos números dos dois cartões que escolheu. Ao
  saber esta soma, o mágico identifica a caixa da qual não se retirou nenhum
  cartão.

  De quantas maneiras podem ser colocados todos os cartões nas caixas de modo
  que este truque sempre funcione? (Duas maneiras consideram-se diferentes se
  pelo menos um cartão é colocado numa caixa diferente.)
\end{questao}

%%% Local Variables:
%%% mode: latex
%%% coding: utf-8-unix
%%% fill-column: 80
%%% TeX-master: "MASTER"
%%% End:
