\begin{questao}
  Cada vértice de um poliedro de faces triangulares é colorido com uma
  dentre quatro cores distintas. Demonstre que o número de faces deste
  poliedro com vértices de três cores diferentes é par.
\end{questao}

\begin{questao}
  $P_1,P_2,\ldots,P_{1982}$ são pontos distintos de um mesmo
  plano. Prove que um ponto $P$ do plano, que não pertença a um
  segmento $P_mP_n$, pertence ao interior de um número par de
  triângulos $P_iP_jP_k$.
\end{questao}

\begin{questao}
  Um triângulo é dividido em triângulos menores de modo que quaisquer
  dois dentre os triângulos ou não tenham pontos em comum, ou tenham
  um ponto em comum, ou tenham um lado (completo) em comum. Os
  vértices do triângulo maior são numerados: 1,2,3. Os vértices dos
  triângulos menores também são numerados: 1,2,3. A numeração é
  arbitrária, exceto pela regra que os vértices sobre o lado oposto a
  um vértice não recebem a numeração deste vértice oposto.

  Mostre que entre os triângulos menores existe um cujos vértices são
  numerados 1,2,3.
\end{questao}

\begin{questao}
  Dado um inteiro $n$, seja $d(n)$ o número de divisores de
  $n$. Seja $\overbar{d}(n)$ o número médio de divisores dos
  números entre $1$ e $n$, ou seja, 

  $$ \overbar{d}(n) = \frac{1}{n}\sum_{1 \leq i \leq n}{d(i)} $$

  Mostre que 

  $$ \sum_{2 \leq i \leq n}{\frac{1}{i}} \leq \overbar{d}(n) \leq
  \sum_{1 \leq i \leq n}{\frac{1}{i}} $$
\end{questao}

\begin{questao}
  Na Terra de Oz há $n$ castelos e várias estradas. Cada estrada
  liga dois castelos. Diz a lenda que se houver quatro castelos
  ligados em ciclo (ou seja, se houver quatro castelos A,B,C,D tais
  que A e B estão ligados, assim como B e C, C e D, D e A), um dragão
  surgirá do meio dos quatro castelos e destruirá a Terra de Oz.

  Mostre que, para que tal desgraça não ocorra, o número de estradas
  na Terra de Oz não pode exceder $\frac{n}{4}(1+\sqrt{4n-3})$.
\end{questao}

\begin{questao}
  Seja $p_n(k)$ o total de permutações do conjunto
  $\{1,2,\ldots,n\}$ com exatamente $k$ pontos fixos. Prove que

  $$  \sum_{1 \leq k \leq n}{k \cdot p_n(k)} 
  = \sum_{1 \leq k \leq n}{(k-1)^2 \cdot p_n(k)} = n! $$
\end{questao}

\begin{questao}
  Sejam $n,k$ dois inteiros positivos, e seja $S$ um conjunto de
  $n$ pontos no plano tais que

  \begin{enumerate}
    \item não haja três pontos de $S$ que sejam colineares;

    \item para qualquer ponto $P$ de $S$, há pelo menos $k$ pontos
    de $S$ que são equidistantes de $P$.
  \end{enumerate}

  Prove que $k < 1/2 + \sqrt{2n}$
\end{questao}

%%% Local Variables:
%%% mode: latex
%%% coding: utf-8-unix
%%% fill-column: 80
%%% TeX-master: "MASTER"
%%% End:
