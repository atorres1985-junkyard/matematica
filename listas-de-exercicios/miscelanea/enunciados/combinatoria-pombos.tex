\begin{questao}
  Seja $S$ um quadrado de lado $2$. Mostre que, dados nove pontos interiores a
  $S$, há três que são vértices de um triângulo de área menor ou igual a $1/2$.
\end{questao}

\begin{questao}
  Mostre que, dado um ponto interior de um quadrado unitário, existem dois
  vértices do quadrado tais que a área do triângulo formado pelos dois vértices
  e o ponto é menor ou igual a $1/8$ (considere que a área de três pontos
  colineares é nula).
\end{questao}

\begin{questao}
  Seja $x$ um número real. Prove que dentre os números 

  $$ \, 2x, \ldots, (n-1)x$$

  existe um que difere de um inteiro por no máximo $1/n$.
\end{questao}

\begin{questao}
  \begin{itemize}
    \item Mostre que de quaisquer $52$ inteiros é sempre possível escolher um par
    cuja soma ou diferença é divisível por $100$.

    \item Seja $k \geq 1$ um número natural. Determine o menor inteiro $n$ com a
    seguinte propriedade: para qualquer escolha de $n$ inteiros haverá um par
    cuja soma ou diferença é divisível por $2k+1$.
  \end{itemize}
\end{questao}

\begin{questao}
  Durante seu treinamento um jogador de xadrez joga pelo menos uma vez por dia e
  não mais que doze vezes por semana. Prove que há um período de dias
  consecutivos no qual ele joga exatamente vinte partidas.
\end{questao}

\begin{questao}
  Escolhem-se $55$ elementos do conjunto
  $\{1,2,\ldots,100\}$. Prove que dois desses elementos terão
  diferença $9$, outros dois $10$, outro par $12$, mas não
  haverá obrigatoriamente um par cuja diferença será $11$.
\end{questao}

\begin{questao}
  Na terra de Oz há vários castelos e de cda um partem três
  estradas. Um cavaleiro errante deixa seu castelo ancestral e viaja
  pelo país. Para manter a viagem interessante, quando chega a um
  determinado castelo, ele vira à direita, se tinha virado à esquerda
  no castelo pelo qual passou anteriormente, e vice-versa. Prove que o
  cavaleiro sempre retornará ao seu próprio castelo.
\end{questao}

\begin{questao}
  Tomamos dois círculos concêntricos, um de raio $2$
  dividido em duzentos setores iguais, pintados alternadamente de
  preto e branco, e outro de raio $1$ também dividido em duzentos
  setores iguais, mas pintados arbitrariamente de preto e
  branco. Prove que podemos girar o círculo menor, de forma que as
  cores deste coincidam com as do maior em pelo menos cem setores.
\end{questao}

\begin{questao}
  Durante as férias escolares, vinte colegas de classe
  viajam para locais distintos. Cada um deles envia um cartão postal
  para dez dentre seus colegas. Mostre que pelo menos dois colegas de
  classe trocaram cartões.
\end{questao}

\begin{questao}
  Dentro de uma sala de área $5$, colocam-se nove tapetes
  de área $1$ e forma arbitrária. Prove que há dois tapetes que se
  sobrepõem de pelo menos $1/9$.
\end{questao}

\begin{questao}
  Cada um de dezessete cientistas corresponde com todos os
  outros. Eles se correspondem sobre três tópicos e cada dois deles
  trata de exatamente um tópico. Prove que há pelo menos três
  cientistas que mandam cartas uns para os outros sobre um mesmo
  tópico.
\end{questao}

\begin{questao}
  No espaço s]ao dados $p_n=\lfloor en! \rfloor + 1$
  pontos. Cada par de pontos é ligado por uma linha e cada linha é
  pintada com uma de $n$ cores. Prove que há ao menos um triângulo
  monocromático.\\
  (Dica: $e=1/0!+1/1!+1/2!+1/3!+\ldots$.)
\end{questao}

\begin{questao}
  Prove que todo inteiro $k>1$ possui um múltiplo menor
  que $k^4$ e que, na notação decimal, pode ser escrito com no
  máximo quatro algarismos distintos.
\end{questao}

\begin{questao}
  Determine o maior inteiro $n$ tal que existe uma pintura
  com duas cores das casas de um tabuleiro $n \times n$ com a
  seguinte propriedade: as quatro casas(distintas) nos cantos de um
  sub-retângulo arbitrário, formado por casas do tabuleiro, não podem
  ter todos a mesma cor.
\end{questao}

\begin{questao}
  Seja $a_1,a_2,\ldots,a_{k^2+1}$ uma sequência de
  números. Mostre que ela contém uma subsequência monotônica de
  $k+1$ termos.
\end{questao}

\begin{questao}
  Seja $\{P_1,P_2,\ldots,P_{1987}\}$ um conjunto de
  $1987$ pontos no interior de um círculo de raio $1$, sendo
  $P_1$ o centro do círculo. Para cada $k=1,2,\ldots,1987$, seja
  $x_k$ a distância de $P_k$ ao ponto mais próximo de $P_k$ e
  distinto de $P_k$. Prove que

  $$x_1^2+x_2^2+\ldots+x_{1987}^2 leq 9$$
\end{questao}

\begin{questao}
  Seja $M$ um conjunto de $3n$ pontos coplanares tal que
  a distância máxima entre dois quaisquer deles é a unidade. Prove
  que:
  \begin{itemize}
    \item entre quaisquer quatro pontos de $M$ existem dois cuja
    distância é menor ou igual a $1/\sqrt{2}$.

    \item existe um círculo de raio no máximo $\sqrt{3}/2$ que contém
    $M$.

    \item há um par de pontos cuja distância é menor ou igual a $4/(3\sqrt{n}-\sqrt{3})$
  \end{itemize}
\end{questao}

\begin{questao}
  \begin{itemize}
    \item Mostre que entre cinco pontos no plano, três a três não
    colineares, existem quatro que são os vértices de um quadrilátero
    convexo.

    \item Mostre que dado um natural $m \geq 4$, existe um menor
    natural $f(m)$ tal que quaisquer $f(m)$ pontos do plano, três
    a três não colineares, contêm $m$ pontos que são vértices de um
    polígono convexo.
  \end{itemize}
\end{questao}

\begin{questao}
  Considere $n$ pontos no plano de modo que não haja três
  pontos colineares. Pintamos cada segmento que liga dois desses
  pontos de vermelho ou azul. Mostre que se existem $k$ pontos tais
  que todos os segmentos que ligam quaisquer dois desses pontos são de
  mesma cor, então $2^{k/2} \geq n \geq 2^k$.
\end{questao}

\begin{questao}
  Um torneio de tênis com $n$ participantes tem a
  propriedade $S_k$ se, para todo conjunto $X$ de $k$
  participantes do torneio, existe um participante não pertencente a
  $X$ que venceu todos os participantes de $X$.

  Mostre que para cada $k$ existe um torneio com a propriedade $S_k$.
\end{questao}

\begin{questao}
  Dezenove flechas são arremessadas são arremessadas sobre
  um alvo com formato de um hexágono de lado unitário. Mostre que duas
  bdessas flechas estarão a uma distância de no máximo $\sqrt(3)/3$
  uma da outra.
\end{questao}

\begin{questao}
  Sejam $x_1,x_2,\ldots,x_n$ números reais tais que
  $x_1^2+x_2^2+\ldots+x_n^2=1$. Mostre que para qualquer valor
  inteiro $k>1$, existem inteiros $e_i$, não todos nulos, com a
  propriedade que $|e_i| < k$, e tais que

  $$|e_1x_1+e_2x_2+\ldots+e_nx_n| \leq \frac{(k-1)\sqrt{n}}{k^n-1}$$
\end{questao}

\begin{questao}
  Mostre que se há $n$ pessoas em uma festa, então duas
  delas conhecem o mesmo número de pessoas entre as presentes.
\end{questao}

\begin{questao}
  Dados três inteiros distintos, sempre é possível escolher
  dois dentre eles, digamos $a,b$, tais que $a^3b-ab^3$ é múltiplo
  de $10$.
\end{questao}

\begin{questao}
  Dado um inteiro $n$, mostre que existe um múltiplo de
  $n$ que se escreve com os algarismos $0,1$ somente. (Por
  exemplo, se $n=3$, temos $111,1011,1100101101,\ldots,$.)
\end{questao}

\begin{questao}
  \begin{itemize}
    \item Dado um conjunto de $101$ inteiros positivos, nenhum dos
    quais excede $200$, mostre que pelo menos um membro deste
    conjunto deve dividir outro membro do conjunto.

    \item Construa um conjuntode $100$ inteiros positivos, menores ou
    iguais a $200$, tal que nenhum membro deste conjunto divida
    outro membro do conjunto.

    \item Prove que se escolhermos $100$ elementos do conjunto
    $\{1,2,\ldots,100\}$ e um deles for menor que $16$, então pelo
    menos um membro deste subconjunto deve dividir outro elemento do subconjunto.
  \end{itemize}
\end{questao}

\begin{questao}
  Mostre que entre sete inteiros positivos distintos menores que $127$ é
  possível escolher um par (digamos $(x,y)$) tal que

  $$1 < \frac{y}{x} \leq 2$$
\end{questao}

\begin{questao}
  Dado um conjunto de dez inteiros positivos estritamente menores que cem, prove
  que há dois subconjuntos disjuntos não-vazios cujas somas dos elementos são
  iguais.
\end{questao}

\begin{questao}
  Prove que qualquer conjunto formado or sete inteiros positivos menores ou
  iguais a $24$ possui dois subconjuntos de mesma soma.
\end{questao}

\begin{questao}
  Dado um conjunto de $n+1$ inteiros positivos, nenhum dos
  quais excede $2n$, mostre que há um par de números primos entre si
  neste conjunto.
\end{questao}

\begin{questao}
  Prove que, dados $m$ inteiros $a_1,a_2,\ldots,a_m$ existem inteiros $k,l$
  ($1 \leq k < l \leq m$) tais que $a_{k+1}+a_{k+2}+\leq+a_l$ é divisível por
  $m$.
\end{questao}

\begin{questao}
  \begin{itemize}
    \item Suponha que cada casa de um tabuleiro $4 \times 7$ é pintada
    de preto ou branco. Prove que, em qualquer coloração desse tipo,
    haverá um retângulo, formado por casas do tabuleiro, cujas casas
    que contêm seus vértices são todas da mesma cor.

    \item Exiba uma pintura de um tabuleiro $4 \times 6$ na qual os
    quatro cantos de cada um dos retângulos (veja o item acima) não
    sejam todos da mesma cor.
  \end{itemize}
\end{questao}

\begin{questao}
  Quinze cadeiras estão colocadas ao redor de uma mesa
  circular e sobre esta estão colocados, em frente a cada uma das
  cadeiras, o nome de quinze convidados. Ao chegarem, os convidados
  não percebem isto e nenhum senta-se em frente ao seu nome. Prove que
  a mesa pode ser girada de forma que pelo menos dois convidados
  fiquem corretamente sentados.
\end{questao}

\begin{questao}
  Um paciente deve tomar $48$ pílulas em $30$ dias,
  tomando pelo menos uma pílula por dia. Demonstre que existe uma
  sequência de dias durante os quais ele toma exatamente $11$
  pílulas.
\end{questao}

\begin{questao}
  Uma sociedade internacional tem membros de seis países
  diferentes. A lista de membros contém $1978$ nomes, numerados de
  $1$ a $1978$. Prove que há pelo menos um membro cujo número é a
  soma dos números de dois membros de seu próprio país ou o dobro do
  número de um membro do seu próprio país.
\end{questao}

\begin{questao}
  Duas pirâmides com vértices $B,C$ e base comum $A_1 A_2
  \ldots A_7$ são dadas. As arestas $BA_i,CA_i (i=1,2,\ldots,7)$ são
  coloridos de vermelho ou azul. Prove que existe nesta configuração
  um triângulo monocromático.
\end{questao}

\begin{questao}
  Cada um dos $36$ segmentos ligando $9$ pontos sobre
  uma circunferência são coloridos de vermelho ou azul. Suponha que
  cada triângulo determinado por $3$ dos $9$ pontos contém pelo
  menos um lado vermelho. Prove que há quatro pontos tais que os seis
  segmentos ligando-os são vermelhos.
\end{questao}

\begin{questao}
  Sejam $n$ pontos do plano tais que não há dois pares de
  pontos equidistantes. Une-se, por um segmento, cada ponto ao mais
  próximo. Prove que nenhum ponto está ligado a mais de cinco pontos.
\end{questao}

\begin{questao}
  Seja $A$ um conjunto de seis pontos coplanares em posição
  geral. Mostre que existem três pontos que formam um triângulo com um
  ângulo interno menor que $30^\circ$.
\end{questao}

\begin{questao}
  Dados $n$ pontos no plano, demonstre que existem três
  deles que determinam um ângulo menor ou igual a $\pi/n$.
\end{questao}

\begin{questao}
  Seja $S=ABCD$ um quadrado de lado unitário e $n$ um
  inteiro positivo arbitrário. No interior de $S$ desenhamnos uma
  curva $P$, de medida maior que $2n$, formada somente por segmentos
  de reta. Prove que existe uma reta $L$, paralela a um lado de $S$,
  que corta $P$ pelo menos $n+1$ vezes (a curva poligonal $P$ pode
  ser constituída de quantos pedaços se queira e pode cortar a si
  própria).
\end{questao}

%%% Local Variables:
%%% mode: latex
%%% coding: utf-8-unix
%%% fill-column: 80
%%% TeX-master: "MASTER"
%%% End:
