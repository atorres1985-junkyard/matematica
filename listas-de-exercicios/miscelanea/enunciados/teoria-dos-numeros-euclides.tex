\begin{questao}
  Ache o menor inteiro positivo $a$ para o qual a equação diofantina
  $1001x + 770y = 1000000+a$ é solúvel, e mostre que, neste caso, a
  equação acima tem cem soluções em inteiros positivos.
\end{questao}

\begin{questao}
  Resolva as seguintes congruências lineares:
  \begin{enumerate}

    \item $5x \equiv 2 \pmod{26}$


    \item $36x \equiv 8 \pmod{102}$
  \end{enumerate}
\end{questao}

\begin{questao}
  Resolva em inteiros positivos a equação
  \begin{align*}
    7x+3y+19z & = 2530 \\
    8x+6y+33z & = 3753
  \end{align*}
\end{questao}

\begin{questao}
  Resolva os seguintes sistemas de congruências lineares:
  \begin{enumerate}

    \item 
    \begin{align*}
      x & \equiv 5 \pmod{12} \\
      x & \equiv 7 \pmod{19}
    \end{align*}

    \item 
    \begin{align*}
      x & \equiv 5 \pmod{6} \\
      x & \equiv 4 \pmod{11} \\
      x & \equiv 3 \pmod{17}
    \end{align*}
  \end{enumerate}
\end{questao}

\begin{questao}
  Quando o sr. Smith descontou um cheque de $x$ dólares e $y$
  cents, recebeu por engano $y$ dólares e $x$ cents, e percebeu
  que tinha ganho dois cents a mais que o dobro da quantia
  descontada. Qual era o valor do cheque?
\end{questao}

\begin{questao}
  Um rapaz recebeu $\$ 100,00$ de sua mãe para comprar alguns itens
  A, de preço $\$ 13,00$; alguns B, de preço $\$ 7,00$; e alguns
  C, de preço $\$ 18,00$ em um supermercado, mas esqueceu a
  quantidade exata de cada item, lembrando apenas que não haveria
  troco. Encontre a probabilidade de o rapaz acertar o pedido da mãe.
\end{questao}

\begin{questao}
  Seja $S$ um conjunto não vazio de inteiros tal que 
  \begin{enumerate}

    \item a diferença $x-y$ está em $S$ quando $x,y$ estão em
    $S$;

    \item todos os múltiplos de $x$ estão em $S$ quando $x$
    está em $S$.
  \end{enumerate}

  \begin{enumerate}

    \item Prove que existe um inteiro positivo $d$ em $S$ tal que
    $S$ consiste de todos os múltiplos de $d$.

    \item Mostre que a parte acima aplica-se ao conjunto $S(a,b) =
    \{ma+nb: m,n \in N\}$, e mostre que o $d$ resultante é
    $mdc(a,b)$.
    \in N\}
  \end{enumerate}
\end{questao}

\begin{questao}
  Dados naturais $a \geq 1, m,n$, então 

  $$ mdc(a^m-1,a^n-1) = a^{mdc(m,n)}-1$$
\end{questao}

\begin{questao}
  A medida de um ângulo é $180^\circ/n$, onde $n$ é um inteiro
  positivo divisível por $3$. 
\end{questao}

\begin{questao}
  Resolva a equação

  $$ (n+1)^2x-n^2y=1 $$
\end{questao}

\begin{questao}
  Resolva a equação $nx+(n+1)y+(n+2)z = n^2$ e mostre que, se $n
  \geq 5$, então sempre haverá uma solução em inteiros positivos.
\end{questao}

\begin{questao}
  Sendo $a,b>0, MDC(a,b)=1$, prove que, para todo $n > ab-a-b$,
  a equação $ax+by=n$ tem solução em inteiros não negativos e que o
  mesmo não ocorre para $ax+by=ab-a-b$.
\end{questao}

\begin{questao}
  Existe um inteiro $L$ tal que, se $m,n$ são inteiros maiores que
  $L$, então um retângulo $m \times n$ pode ser construído com
  retângulos $4 \times 6$ e $5 \times 7$?
\end{questao}

\begin{questao}
  Escrevem-se, sucessivamente, inteiros positivos
  $a_1,a_2,a_3,\ldots$ respeitando a seguinte condição: $a_{n+1}$
  não é da forma $k_1a_1+k_2a_2+\ldots+k_na_n$, com
  $k_1,k_2,\ldots,k_n$ inteiros não negativos.

  É possível que este processo seja infinito?
\end{questao}

\begin{questao}
  Considere a função $f:Z \rightarrow Z$ tal que, para todo $x \in
  Z$,

  \begin{align*}
    f(92+x) & = f(92-x) \\
    f(19 \cdot 92+x) & = f(19 \cdot 92-x) \\
    f(1992+x) & = f(1992-x)
  \end{align*}

  É possível que todos os divisores de $92$ sejam valores de $f$?
\end{questao}

\begin{questao}
  Sejam $a,b,c$ inteiros positivos, primos dois a dois. Mostre que
  $2abc-ab-bc-ca$ é o maior inteiro que não pode ser representado na
  forma $xbc+yca+zab$, onde $x,y,z$ são inteiros não negativos.
\end{questao}

\begin{questao}
  Determine para quais inteiros positivos $k$ o conjunto $\{1990,
  1990+1, \ldots, 1990+k\}$ pode ser particionado em dois
  subconjuntos disjuntos $A,B$ tais que a soma dos elementos de
  $A$ seja igual à soma dos elementos de $B$.
\end{questao}

\begin{questao}
  Para $n,k \in N$ definimos $F(n,k) = \sum_{1 \leq r \leq n
  }{r^{2k-1}}$. Prove que $F(n,1)$ divide $F(n,k)$.
\end{questao}

\begin{questao}
  Sendo $r$ um inteiro positivo, definimos a sequência $a_n$ como:

  \begin{align*}
    a_1 & = 1 \\
    a_{n+1} & = \frac{n \cdot a_n+2(n+1)^{2r}}{n+2}
  \end{align*}

  Prove que $a_n$ é inteiro, e determine os valores de $n$ para os
  quais $a_n$ é par.
\end{questao}

\begin{questao}
  Prove que existe uma sucessão $a_0,a_1,a_2,\ldots,a_k,\ldots$ onde
  $a_i \in \{0,1,2,3,4,5,6,7,8,9\}$, com $a_0=6$, tal que para
  todo inteiro positivo $n$, sendo
  $x_n=a_0+10a_1+100a_2+\ldots+10^{n-1}a_{n-1}$, $x_n^2-x_n$ é
  divisível por $10^n$.
\end{questao}

\begin{questao}
  Existe um múltiplo de $5^{100}$ que não contenha zeros em sua
  representação decimal?
\end{questao}

\begin{questao}
  Um matemático excêntrico possui uma escada com $n$ degraus. Sempre
  que o matemático dá um passo para cima na escada, ele avança $a$
  degraus, e sempre que dá um passo para baixo, ele avança $b$
  degraus. Ache o valor mínimo de $n$, em função de $a$ e $b$,
  para o qual o matemático pode subir do chão até o alto da escada e
  voltar a descer.
\end{questao}

\begin{questao}
  Sejam $n,k$ inteiros positivos primos entre si, com $1 \leq k <
  n$, e seja $M = \{1,2,\ldots,n-1\}$. Cada elemento de $M$ é
  pintado de azul ou branco.

  Suponha que as duas condições a seguir sejam satisfeitas:
  \begin{itemize}

    \item para todo $i \in M$, $i$ e $n-i$ são da mesma cor.


    \item para todo $i \in M$, $i \not = k$, $i$ e $k-i$ são da
    mesma cor.
  \end{itemize}

  Demonstre que todos os elementos de $M$ são da mesma cor.
\end{questao}

%%% Local Variables:
%%% mode: latex
%%% coding: utf-8-unix
%%% fill-column: 80
%%% TeX-master: "MASTER"
%%% End:
