\begin{questao}
  Seja $a$ um número real, e $f: R \rightarrow R$ uma função tal que
  $f(0)=\frac{1}{2}$ e $f(x+y) = f(x) \cdot f(a-y) + f(y) \cdot f(a-x)$ para
  todos $x,y \in R$. Prove que $f$ é uma função constante.
\end{questao}

\begin{questao}
  Uma função $f: Z \rightarrow Z$ é tal que $f(x)=x-10$ se $x>100$ e
  $f(x)=f(f(x+11))$ se $x \geq 100$. Determine, justificando, o conjunto de
  valores da função $f$.
\end{questao}

\begin{questao}
  Determine todas as funções $f: R \rightarrow R$ tais que
  $$ f(x^2) - f(y^2) +2x+1 = f(x+y) \cdot f(x-y) $$ para todos os reais $x,y$.
\end{questao}

\begin{questao}
  Determine todas as funções $f: N^{*} \rightarrow N^{*}$ tais que

  \begin{itemize}[itemsep=1ex, leftmargin=1cm]

  \item Se $x<y$ então $f(x) < f(y)$;

  \item Para todos $x,y \in N^{*}$, temos $f(yf(x)) = x^2 f(xy)$.

  \end{itemize}
\end{questao}

\begin{questao}
  Determine todas as funções $f: R \rightarrow R$ satisfazendo

  $$(x-y)f(x+y)-(x+y)f(x-y)=4xy(x^2-y^2)$$

  para todos os reais $x,y$.
\end{questao}

\begin{questao}
  Encontre todas as funções $f: Q^+ \rightarrow Q^+$ tais que
  $$f \left(x+\frac{y}{x} \right) = f(x) + \frac{f(y)}{f(x)}+2y$$ para todo $x,y
  \in \mathbb{Q^+}$.
\end{questao}

\begin{questao}
  Determine todas as funções $f: R-\{-1,0,1\} \rightarrow R$ tais que $(f(x)^2)
  \cdot f \left( \frac{1-x}{1+x} \right) = 64x$ para todo $x \in R-\{-1,0,1\}$.
\end{questao}

\begin{questao}
  Encontre a função $f$ que satisfaz
  $$ f(x) + f \left( \frac{1}{1-x} \right) = x $$ para todo $x \not = 0,1$.
\end{questao}

\begin{questao}
  Existe uma função $f:N \rightarrow N$ tal que $f(f(n)) = n + 1987$ para todo
  natural $n$?
\end{questao}

\begin{questao}
  Seja $f: N^{*} \rightarrow N^{*}$ tal que para todo $m,n \in N^{*}$, temos
  $f(f(m)+f(n)) = m+n$. Determine os possíveis valores de $f(1988)$.
\end{questao}

\begin{questao}
  A função $f: R^* \rightarrow R$ satisfaz

  \begin{enumerate}

  \item $f(x)-f(y) = f(x)f(1/y)-f(y)f(1/x)$ para quaisquer $x,y \in
    \mathbb{R^*}$;

  \item $f(x)=1/2$ para pelo menos um $x \in \mathbb{R^*}$.
  \end{enumerate}

  Determine $f(-1)$.
\end{questao}

\begin{questao}
  Seja $f: R^2 \rightarrow R$ uma função real. Determine $f(19,94)$ se, para
  quaisquer três reais $x,y,z$ vale a igualdade $x+f(y,z) = f(x,y)+f(x,z)$.
\end{questao}

\begin{questao}
  Determine todas as funções $f: R \rightarrow R^*$ satisfazendo

  $$ f(y)+f(x+y)+f(xy) = \frac{(f(x))^2+f(y)}{f(x-y)} + 1 $$

  para quaisquer reais $x,y$.
\end{questao}

\begin{questao}
  Seja $f: R^+ \rightarrow R^+$ tal que $f(xf(y)) = x^py^q$ para quaisquer $x,y
  \in \mathbb{R+}$, onde $p,q$ são inteiros positivos fixos. Mostre que $q=p^2$.
\end{questao}

\begin{questao}
  A função $f: R \rightarrow R$ satisfaz $4f(f(x)) = 2f(x)+x$ para todo real
  $x$. Prove que $f(x)=0$ se, e somente se, $x=0$.
\end{questao}

\begin{questao}
  Seja $f$ uma função satisfazendo as seguintes condições:

  \begin{enumerate}

  \item Se $x>y$ e $f(y)-y \geq v \geq f(x)-x$, então $f(x)=v+z$ para algum
    número $z$ entre $x$ e $y$;

  \item A equação $f(x)=0$ tem alguma raiz e entre as raízes desta equação
    existe uma que é maior que as demais;

  \item $f(0)=1$;

  \item $f(1987) \leq 1988$;

  \item $f(x) \cdot f(y) = f(x \cdot f(y) + y \cdot f(x) -xy)$.
  \end{enumerate}


  Ache $f(1978)$.
\end{questao}

\begin{questao}
  Seja $a > 1$ e $f:R_{+} \rightarrow R_{+}$ uma função estritamente crescente
  satisfazendo as seguintes condições:

  \begin{enumerate}

  \item $\lim_{x \rightarrow +\infty}{(f(x+1)-f(x))} = 1$; e

  \item $f(ax) \cdot f \left( \frac{x}{a} \right) = (f(x))^2$ para todo $x \geq
    0$.
  \end{enumerate}

  Mostre que $\lim_{x \rightarrow +\infty}{f(x)/x} = 1$ e encontre todas as
  funções satisfazendo as propriedades acima.
\end{questao}

\begin{questao}
  Sejam $a,b$ números naturais, com $1 \leq a \leq b$, e $M = \left \lfloor
  \frac{a+b}{2} \right \rfloor$. Define-se a função $f: Z \rightarrow Z$ por

  $$ f(x) = \left\{
  \begin{split}
    n+a &, n < M\\ n-b &, n \geq M
  \end{split}
  \right. $$

  Encontre o menor natural $k$ tal que $f^k(0)=0$.
\end{questao}

\begin{questao}
  Prove que existe precisamente uma função definida para todos os reais não
  nulos satisfazendo


  \begin{enumerate}

  \item $f(x) = xf(1/x)$ para todos os reais não nulos;

  \item $f(x)+f(y) = 1 + f(x+y)$ para todos os reais não nulos $x,y, x \not =
    y$.
  \end{enumerate}

\end{questao}

\begin{questao}
  Determine todas as funções $f: R-\{2/3\} \rightarrow {R}$ satisfazendo

  $$ 498x - f(x) = \frac{1}{2} \cdot f \left(\frac{2x}{3x-2} \right) $$

  para todo $x \not= 2/3$.
\end{questao}

\begin{questao}
  Para todas as funções $f: R \rightarrow R$ satisfazendo


  \begin{enumerate}

  \item $f(xy) = xf(y)+yf(x)$

  \item $f(x+y)=f(x^{1993})+f(y^{1993})$ $x,y, x \not = y$.
  \end{enumerate}

  Determine $f(\sqrt{5753})$.
\end{questao}

\begin{questao}
  Determine todas as funções $f: Q \rightarrow R$ satisfazendo

  $$f(x+y)=f(x)+f(y)+2xy$$

  para quaisquer $x,y \in Q$.
\end{questao}

\begin{questao}
  Determine todas as funções $f: R^+ \rightarrow R$ satisfazendo


  \begin{enumerate}

  \item $f(xy) = f(x)f(3/y) + f(y)f(3/x)$ para quaisquer $x,y \in R^+$;

  \item $f(1)=1/2$.
  \end{enumerate}

\end{questao}

\begin{questao}
  Determine todas as funções $f: R \rightarrow R$ tais que

  $$ f(x-f(y)) = f(f(y)) + xf(y) +f(x) - 1 $$

  para quaisquer $x,y \in R$.
\end{questao}

\begin{questao}
  Sejam $f,g$ funções dos inteiros nos inteiros tais que

  \begin{enumerate}

  \item $f(m+f(f(n))) = -f(f(m+1))-n$ para $m,n$ inteiros;

  \item $g$ é um polinômio de coeficientes inteiros tal que $g(n) = g(f(n))$
    para todo inteiro $n$.
  \end{enumerate}


  Determine $f(1991)$ e $g$ (em sua forma mais geral).
\end{questao}

\begin{questao}
  Existe uma função $f: N \in N$ tal que

  \begin{align}
    f(1) &= 2 \\ f(f(n)) &= f(n) + n \\ f(n) &< f(n+1)
  \end{align}

  para todos $n \in N$?
\end{questao}

\begin{questao}
  A função $f:N \rightarrow N$ é definida como se segue:

  \begin{align}
    f(1)=1 &, f(3)=3 \\ f(2n) &= f(n) \\ f(4n+1) &= 2f(2n+1) - f(n) \\ f(4n+3)
    &= 3f(2n+1) - 2f(n)
  \end{align}

  Encontre todos os valores de $n$ com $f(n)=n$ e $1 \leq n \leq 1988$.
\end{questao}

\begin{questao}
  A função $f:N^{*} \rightarrow N^{*}$ estritamente crescente satisfaz $f(f(n))
  = 3n$. Determine $f(1994)$.
\end{questao}

\begin{questao}
  Prove que não existe uma função $f: Z \rightarrow Z$ para a qual $f(f(x)) =
  x+1$ para todo $x \in Z$
\end{questao}

\begin{questao}
  Seja $Q^+_*$ o conjunto dos racionais positivos. Construa uma função $f: Q^+_*
  \rightarrow Q^+_*$ tal que $f(xf(y)) = f(x)/y$ para todo $x,y \in {Q^+_*}$.
\end{questao}

\begin{questao}
  Determine todas as funções $f:R^2 \rightarrow R$ tais que $f(a;a)=a$ para todo
  $a \in R$ e $a+b<c+d \Rightarrow f(a;b) < f(c;d)$ para quaisquer $a,b,c,d \in
  R$.
\end{questao}

\begin{questao}
  Seja $f: R \rightarrow R$ uma função tal que $f(1)=1$, $f(a+b)=f(a)+f(b)$ para
  todo $a,b$ e $f(x)f(1/x)=1$ para todo $x \not= 0$. Prove que $f(x)=x$ para
  todo $x \in R$.
\end{questao}

\begin{questao}
  A função $f(x)$ é definida para todo $x>0$ e satisfaz as condições

  \begin{itemize}[itemsep=1ex, leftmargin=1cm]

  \item $f(x)$ é estritamente crescente em $(0,+\infty)$

  \item $f(x) > -1/x$ para $x>0$

  \item $f(x) \cdot f(f(x)+1/x) = 1$ para todo $x>0$
  \end{itemize}


  \begin{enumerate}

  \item Determine $f(1)$.

  \item Dê um exemplo de função $f(x)$ satisfazendo as condições acima.
  \end{enumerate}

\end{questao}

\begin{questao}
  Encontre todas as sequências de inteiros positivos satisfazendo $f(f(f(n))) +
  f(f(n)) + f(n) = 3n$.
\end{questao}

\begin{questao}
  Encontre todas as funções $f: R \rightarrow R$ tais que $f(x^2+f(y)) =
  y+(f(x))^2$.
\end{questao}

\begin{questao}
  Encontre todas as funções $f$, definidas no conjunto dos reais não negativos e
  assumindo valores reais não negativos, tais que

  \begin{enumerate}

  \item $f(xf(y))f(y)=f(x+y)$ para todo $x,y \geq 0$;

  \item $f(2)=0$;

  \item $f(x)\not=0$ para $0 \leq x < 2$.
  \end{enumerate}

\end{questao}

\begin{questao}
  Seja $f: R \rightarrow R$ tal que, para uma determinada constante $a$, tem-se
  $f(x+a) = \frac{1}{2} + \sqrt{f(x)-(f(x))^2}$, para todo $x \in R$.

  \begin{enumerate}

  \item Prove que $f$ é periódica.

  \item Dê um exemplo de função $f$ como descrita acima, para $a=1$.
  \end{enumerate}

\end{questao}

\begin{questao}
  
  $G$ é um conjunto de funções não constantes de variável real $x$ da forma
  $f(x) = ax+b; a,b \in R$. G tem as propriedades a seguir:

  \begin{enumerate}

  \item Se $f,g \in G$, então $g \circ f \in G$;

  \item Se $f \in G$, então $f^{-1} \in G$;

  \item Para toda $f \in G$, existe um número real $x_f$ tal que $f(x_f) = x_f$.
  \end{enumerate}

  Prove que existe um real $K$ tal que $f(x) = k$ para toda $f \in G$.
\end{questao}

\begin{questao}
  Determine todas as funções $f: R \rightarrow R$ tais que

  \begin{enumerate}

  \item $f$ é estritamente crescente;

  \item $f(x)+g(x)=2x$ para todo real $x$ -- onde $g(x)$ é a função inversa de
    $f(x)$ (isto é, $f(g(x)) = g(f(x)) = x$).
  \end{enumerate}
\end{questao}

\begin{questao}
  Seja $f: N \rightarrow N$ uma função que verifica as condições a seguir:
  satisfazendo

  \begin{enumerate}

  \item Se $n=2^j-1$ para $j=0,1,2,\ldots$, então $f(n)=0$.

  \item Se $n \not= 2^j-1$ para $j=0,1,2,\ldots$, então $f(n+1) = f(n)-1$.
  \end{enumerate}

  \begin{enumerate}

  \item Demonstre que, para todo inteiro não-negativo $n$, existe um inteiro
    não-negativo $k$ tal que $f(n)+n=2^k-1$.

  \item Calcule $f(2^{1990})$.
  \end{enumerate}

\end{questao}

\begin{questao}
  Defina $f: Q \rightarrow R$ por:

  \begin{enumerate}

  \item $f(0)=0$, e para $\alpha,\beta \in Q$ arbitrários,

  \item $f(\alpha \beta) = f(\alpha) \cdot f(\beta)$;

  \item $f(\alpha + \beta) \leq f(\alpha) + f(\beta)$;

  \item $f(m) \leq 1989$ para $m \in Z$.
  \end{enumerate}

  Prove que $f(\alpha + \beta) = max\{f(\alpha), f(\beta)\}$ se $f(\alpha) \not
  = f(\beta)$.
\end{questao}

\begin{questao}
  Sejam $f_1,f_2,\ldots,f_n,\ldots$ funções dos naturais nos naturais. Demonstre
  que existe uma função $F$ dos naturais nos naturais com a seguinte
  propriedade: para todo $n$, existe $m_0$ tal que $$m > m_0 \Rightarrow F(m) >
  2^mf_n(m)$$
\end{questao}

\begin{questao}
  Seja $f:N^{*} \rightarrow N^{*}$ tal que

  \begin{enumerate}

  \item $f(f(n)) = 4n+9$ para todo $n \in N^{*}$;

  \item $f(2^k) = 2^{k+1}+3$ para todo inteiro não negativo $k$.
  \end{enumerate}

  Determine $f(1789)$.
\end{questao}

\begin{questao}
  Encontre todas as funções $f: {Q^+_*} \rightarrow {Q^+_*}$ tais que
  $f(x+1)=f(x)+1$ e $f(x^3)=(f(x))^3$.
\end{questao}

\begin{questao}
  Seja $f(n)$ uma função definida no conjunto dos inteiros positivos, com
  valores no mesmo conjunto. Prove que se $f(n+1) > f(f(n))$ para todo inteiro
  positivo $n$, então $f(n)=n$ para todo inteiro positivo $n$.
\end{questao}

\begin{questao}
  Considere todas as funções $f:N^{*} \rightarrow N^{*}$ que satisfazem
  $f(t^2f(s)) = s(f(t))^2$ para todos os inteiros positivos $s,t$. Determine o
  menor valor possível para $f(1998)$.
\end{questao}

\begin{questao}
  Seja $S$ o conjunto dos reais maiores que $-1$. Encontre todas as funções $f:
  S \rightarrow S$ satisfazendo as condições


  \begin{enumerate}

  \item $f(x+f(y)+xf(y)) = y+f(x)+yf(x), \cap x,y \in S$

  \item $f(x)/x$ é estritamente crescente para $-1<x<0$ e $x>0$.
  \end{enumerate}
\end{questao}

\begin{questao}
  Encontre todas as funções $f$ definidas no conjunto dos reais positivos e
  assumindo valores neste conjunto e que satisfaz as condições:

  \begin{enumerate}

  \item $f(xf(y)) = yf(x)$ para todo $x,y \in \mathbb{R^*_+}$;

  \item $f(x) \rightarrow 0$ quando $x \rightarrow +\infty$.
  \end{enumerate}

\end{questao}

\begin{questao}
  A função $f: Z^{*}_{+} \rightarrow Z^{*}_{+}$ satisfaz as condições:

  \begin{itemize}[itemsep=1ex, leftmargin=1cm]

  \item $f(m+n)-f(m) \cdot f(n) = p$, sendo $p=0$ ou $p=1)$;

  \item $f(2)=0$;

  \item $f(3)>0$;

  \item $f(9999)=3333$;
  \end{itemize}

  Ache $f(1982)$.
\end{questao}

\begin{questao}
  Mostre que não existem funções $f: R \rightarrow R$ tais que $f(f(x)) =
  x^2-1996$.
\end{questao}

\begin{questao}
  Seja $\mathbb{N_0}$ o conjunto dos inteiros não negativos. Encontre todas as
  funções $f: N_0 \rightarrow N_0$ tais que

  $$f(m+f(n)) = f(f(m)) + f(n), \forall m,n \in \mathbb{N_0}$$
\end{questao}

%%% Local Variables:
%%% mode: latex
%%% coding: utf-8-unix
%%% fill-column: 80
%%% TeX-master: "MASTER"
%%% End:
