\begin{questao}
  Dentro de uma caixa há $1995$ bolas pretas e $2000$ bolas brancas, e fora dela
  há $5000$ bolas brancas. Retiramos da caixa $2$ bolas. Se elas forem da mesma
  cor então retornamos uma bola branca. Se elas forem de cores diferentes
  retornamos uma bola preta. Repete-se o processo até que reste uma única bola
  na caixa. Qual pode ser a sua cor?
\end{questao}

\begin{questao}
  Doze anões vivem em uma floresta e cada um deles tem uma casa que é pintada de
  vermelho ou azul. No i-ésimo mês de cada ano, o i-ésimo anão visita todos os
  seus amigos e se encontra a maioria deles vivendo em casas de cor diferente da
  sua própria, ele decide juntar-se a eles e muda a cor de sua casa. Mostre que,
  mais cedo ou mais tarde, nenhum anão precisará mudar a cor de sua casa. (As
  amizades são mútuas e não mudam.)
\end{questao}

\begin{questao}
  Seja $Q_0$ o quadrado de vértices $P_0(1,0),P_1(1,1),P_2(0,1),P_3(0,0)$. Seja
  $A_0$ o interior desse quadrado. Seja $P_{n+4}$ o ponto médio do segmento
  $P_nP_{n+1}$. $Q_n$ é o quadrado de vértices
  $P_{n+1},P_{n+2},P_{n+3},P_{n+4}$. Seja $A_n$ o interior de $Q_n$. Encontre a
  intersecção de todos os $A_n$.
\end{questao}

\begin{questao}
  É atribuído um inteiro a cada um dos vértices de um pentágono regular, de tal
  forma que a soma dos cinco números seja positiva. Se três vértices
  consecutivos recebem os números $x,y,z$ respectivamente, e $y<0$, então a
  seguinte operação é permitida: os números $x,y,z$ são trocados por
  $x+y,-y,y+z$ respectivamente. Tal operação é repetida enquanto houver um
  número negativo entre os cinco atribuídos. Determine se este processo
  necessariamente se encerra após um número finito de aplicações de tal
  operação.
\end{questao}

\begin{questao}
  Imagine um tabuleiro de xadrez infinito dividido em dois por uma de suas retas
  horizontais. Devemos agora colocar $m$ peças na parte inferior do tabuleiro,
  e, movendo-as segundo as regras do {\it Resta-Um}, trazer pelo menos uma delas
  até a $n$-ésima linha acima da divisória. Determine, para cada valor de $n$, o
  menor valor de $m$ para o qual isto é possível.

\end{questao}

\begin{questao}
  É dada uma equação do segundo grau $x^2+ax+b=0$ com raízes inteiras
  $a_1,b_1$. Consideramos a equação do segundo grau $x^2+a_1x+b_1=0$. Se a
  equação tem raízes inteiras $a_2,b_2$, consideramos a equação
  $x^2+a_2x+b_2=0$. Se a equação tem raízes inteiras $a_2,b_2$, consideramos a
  equação $x^2+a_3x+b_3=0$. E assim por diante. Se encontrarmos uma equação com
  $\Delta < 0$ ou com raízes que não sejam números inteiros, encerramos o
  processo.

  Exemplos:
  \begin{itemize}
  \item $x^2-3x+2=0 \rightarrow x^2+2x+1=0 \rightarrow x^2-x-1=0$, e não podemos
    prosseguir porque as raízes de $x^2-x-1=0$ são
    $(1+\sqrt{5})/2,(1-\sqrt{5})/2$, números não inteiros.

  \item $x^2-3x+2=0 \rightarrow x^2+x+2=0$, e não podemos continuar pois $\Delta
    = -7 < 0$.

  \item $x^2=0 \rightarrow x^2=0$ e neste caso podemos continuar o processo
    indefinidamente (isto é, em nenhuma equação obtida ocorre $\Delta<0$ ou
    raízes não inteiras).
  \end{itemize}

  \begin{itemize}
  \item Determine uma outra equação que, como $x^2=0$, nos permita continuar o
    processo indefinidamente.

  \item Determine todas as equações do segundo grau completas a partir das quais
    possamos continuar o processo indefinidamente.
  \end{itemize}  
\end{questao}

\begin{questao}
  Num tabuleiro infinito no qual estão dispostas $n^2$ peças, formando um
  quadrado $n \times n$, jogamos o {\it Resta-Um}. Para quais valores de $n$ há
  uma estratégia vencedora, ou seja, é possível termina o jogo com apenas uma
  peça no tabuleiro?

\end{questao}

\begin{questao}
  Um tabuleiro $n \times n$ é preenchido com peças brancas e pretas, de acordo
  com as seguintes regras: Inicialmente (tabuleiro vazio) uma peça preta é
  colocada em uma casa vazia e todas as peças, se houver alguma, situadas em
  casas vizinhas (i.e., com uma aresta em comum), são trocadas por peças de cor
  oposta. Este processo se prolonga até o tabuleiro estar completamente
  preenchido. Prove que, ao final do processo, restará ao menos uma peça preta
  sobre o tabuleiro.

\end{questao}

\begin{questao}
  Seja $Q$ um poliedro convexo com $1994$ vértices. Mostre que podemos associar
  a cada aresta de $Q$ um dos números $+1,-1$ de modo que, para todo vértice $v$
  de $Q$, o produto dos números associados às arestas que têm $v$ como
  extremidade seja $-1$.

\end{questao}

\begin{questao}
  Sejam quatro inteiros $a,b,c,d$ não todos iguais. Começamos com a quádrupla
  $(a,b,c,d)$ e substituímos repetidamente por $(a-b,b-c,c-d,d-a)$. Mostre que
  pelo menos um componente da quádrupla pode se tornar arbitrariamente grande.
\end{questao}

\begin{questao}
  Pablo escolhe um inteiro positivo $n$ e escreve no quadro-negro os $2n+1$
  números

  $$ \frac{n}{1},\frac{n}{2},\frac{n}{3},\ldots,\frac{n}{2n+1}$$

  Laura escolhe dois números escritos por Pablo, $a,b$, apaga-os e escreve o
  número $2ab-a-b+1$. Depois de repetir este procedimento $2n$ vezes, sobra
  apenas um número no quadro negro. Determinar os possíveis valores deste
  número.

\end{questao}

\begin{questao}
  Considere o conjunto dos cem números

  $$1,\frac{1}{2},\frac{1}{3},\ldots,\frac{1}{100}$$

  Eliminam-se dois elementos quaisquer $a,b$ deste conjunto e inclui-se o número
  $ab+a+b$, restando assim um conjunto com um elemento a menos. Após $99$ destas
  operações, resta apenas um número. Quais valores pode assumir esse número?
\end{questao}

\begin{questao}
  Arnaldo e Bernaldo brincam da seguinte maneira: cada um deles escreve em uma
  folha de papel um inteiro positivo e dá a folha a Cernaldo, que será o juiz
  desse jogo. Cernaldo escreve em um quadro-negro dois inteiros, um dos quais é
  a soma dos inteiros escritos por ambos os jogadores. Então Cernaldo pergunta a
  Arnaldo: ``Você pode dizer qual foi o inteiro escolhido pelo seu
  adversário?''. Caso ele responda ``Não'', Cernaldo coloca a mesma questão para
  Bernaldo. Caso ele responda ``Não'', Cernaldo coloca a mesma questão para
  Arnaldo. E assim por diante. Assumindo que ambos os estudantes são sinceros e
  inteligentes, demonstre que, após um número finito de questões, um dos
  estudantes responderá ``Sim''.
\end{questao}

\begin{questao}
  $P_1,P_2,\ldots,P_{1982}$ são pontos distintos sobre um mesmo plano. Prove que
  para todo ponto $P$ do plano, que não pertença a um segmento $P_iP_j$,
  pertence a um número par de triângulos $P_iP_jP_k$.
\end{questao}

\begin{questao}
  Mostre que, se cobrirmos um quadrado $6 \times 6$ com dominós, deverá haver
  uma linha horizontal ou vertical que não intercepta nenhum dominó.
\end{questao}

\begin{questao}
  Um cubo $20 \times 20 \times 20$ é composto de $2000$ tijolos, de tamanho $2
  \times 2 \times 1$. Prove que é possível furar o cubo com uma agulha de modo
  que a agulha atravesse o cubo sem atravessar nenhum tijolo.
\end{questao}

\begin{questao}
  Seja $n$ um inteiro ímpar maior que $1$ e seja $A$ uma matriz simétrica $n
  \times n$ tal que cada linha e cada coluna de $A$ consiste de alguma
  permutação de $1,2,\ldots,n$. Mostre que cada um dos números $1,2,\ldots,n$
  aparece na diagonal principal.
\end{questao}

\begin{questao}
  Um chapéu contém $1989$ cartões, nos quais estão escritos os números
  $1,2,\ldots,1989$. Dois cartões são retirados do chapéu, a diferença entre os
  seus números é escrita em um novo cartão que é colocado no chapéu, enquanto os
  dois cartões originais são destruídos. Este procedimento é repetido até o
  chapéu conter somente um certão. Mostre que o número que aparece neste cartão
  é ímpar.
\end{questao}

\begin{questao}
  Sejam $P_1,P_2,\ldots,P_{13}$ pontos coplanares e suponha que eles são ligados
  por segmentos $P_1P_2,P_2P_3,\ldots,P_{16}P_{17}$. É possível traçar uma reta
  que passe através do interior de cada um desses segmentos?
\end{questao}

\begin{questao}
  Imagine uma prisão constituída de $64$ celas arranjadas como as casas de um
  tabuleiro $8 \times 8$. Há portas entre quaisquer celas vizinhas. Um
  prisioneiro de uma cela do canto recebe a promessa de que se ele puder
  alcançar a cela do canto oposto, atravessando exatamente uma vez cada uma das
  outras celas, ele estará livre. Poderá o prisioneiro obter sua liberdade?
\end{questao}

\begin{questao}
  Existe uma permutação dos números $1,1,2,2,3,3,\ldots,1986,1986$ tal que, para
  todo $k$, há exatamente $k$ números entre os dois $k$\' s?
\end{questao}

\begin{questao}
  São dadas cem caixas de fósforos numeradas de um a cem. Podemos perguntar se o
  total de fósforos de um conjunto de quinze caixas é par ou ímpar. Qual é o
  número mínimo de perguntas necessárias para decidir a paridade do número de
  fósforos na caixa um?
\end{questao}

%%% Local Variables:
%%% mode: latex
%%% coding: utf-8-unix
%%% fill-column: 80
%%% TeX-master: "MASTER"
%%% End:
