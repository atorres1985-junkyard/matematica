\begin{questao}
  Mostre que $1000^{1000} > 1001^{999}$.
\end{questao}

\begin{questao}
  Demonstre que, para $a>1$, $$ \frac{1}{\sqrt{a}} < \sqrt{a+1} - \sqrt{a-1} $$
\end{questao}

\begin{questao}
  Mostre que, para $a>b$ e $r>s$, $$ \frac{a^s+b^s}{a^r+b^r} >
  \frac{a^s-b^s}{a^r-b^r} $$
\end{questao}

\begin{questao}
  Mostre que $$ \frac{a+b}{1+a+b} < \frac{a}{1+a} + \frac{b}{1+b} $$
\end{questao}

\begin{questao}
  Mostre que $$ n(a^{2n+1}+1) \geq a+a^2+a^3+\ldots+a^{2n} $$
\end{questao}

\begin{questao}
  Mostre que $$ n! \leq \left( \frac{n+1}{2} \right)^n $$
\end{questao}

\begin{questao}
  Seja $ S(n) =
  1+\frac{1}{\sqrt{2}}+\frac{1}{\sqrt{3}}+\ldots+\frac{1}{\sqrt{n}} $. Mostre
  que

  \begin{enumerate}

  \item $ \sqrt{n} < S(n) < n $;

  \item $ S(n) < \sqrt{n+1} + \sqrt{n} - \sqrt{2} $
  \end{enumerate}

\end{questao}

\begin{questao}
  Mostre que $$ \frac{1}{n+1} + \frac{1}{n+2} + \frac{1}{n+3} + \ldots +
  \frac{1}{2n} < \frac{3}{4} $$
\end{questao}

\begin{questao}
  Mostre que $$ \frac{1}{1^2} + \frac{1}{2^2} + \frac{1}{3^2} + \ldots +
  \frac{1}{n^2} < \frac{7}{4} - \frac{1}{n} $$
\end{questao}

\begin{questao}
  Mostre que $$ (a-b)^2(a+b-c) + (b-c)^2(b+c-a) + (c-a)^2(c+a-b) \geq 0 $$
\end{questao}

\begin{questao}
  Prove que, para $a,b$ reais positivos,

  $$ \frac{a+b}{2} -\sqrt{ab} \geq
  \frac{(a-b)^2(a+3b)(b+3a)}{8(a+b)(a^2+6ab+b^2)} $$
\end{questao}

\begin{questao}
  Prove que $$ \frac{A+B}{2} \geq \sqrt{AB} $$ para $A,B$ positivos.
\end{questao}

\begin{questao}

  \begin{enumerate}

  \item Mostre que $x^2+y^2+z^2 \geq xy+xz+yz$ para $x,y,z$ reais.

  \item Mostre que $x^4+y^4+z^4 \geq xyz(x+y+z)$ para $x,y,z$ reais.

  \item Mostre que
    $$a+b+c \leq \frac{ab}{c}+ \frac{ac}{b}+ \frac{bc}{a}$$
  \end{enumerate}

\end{questao}

\begin{questao}
  Mostre que
  $$ 4x(x+y)(x+z)(x+y+z) + y^2z^2 \geq 0 $$ para $x,y,z$ reais.
\end{questao}

\begin{questao}
  Mostre que os reais positivos $a,b,c$ são lados de um triângulo se, e somente
  se, satisfazem

  $$ (a^2+b^2+c^2)^2 > 2(a^4+b^4+c^4) $$
\end{questao}

\begin{questao}
  Demonstre que, para $A>B \geq 0$ e $n \geq 2$,

  $$ nB^{n-1}(A-B) < A^n-B^n < nA^{n-1}(A-B) $$
\end{questao}

\begin{questao}
  Prove a {\it Desigualdade de Bernoulli}: para $x \geq -1$ e $n \geq 2$,

  $$ (1+x)^n \geq 1+nx $$
\end{questao}

\begin{questao}
  Sejam $a_1,a_2,\ldots,a_k$ números não negativos e $A$ o maior deles. Mostre
  que, para $n$ natural maior que $1$,

  $$ \frac{a_1^n+a_2^n+ \ldots +a_k^n}{k} \leq (\frac{a_1+a_2+ \ldots
    +a_k}{k})^n + (n-1)n^{n/{1-n}}A^n $$
\end{questao}

\begin{questao}
  Sejam $x_1,x_2,\ldots,x_n$ números reais tais que

  $$ x_{k+1} \geq \frac{1}{k}(x_1+x_2+\ldots +x_k) $$

  para todo $k \geq 1$. Mostre que

  $$ x_1+2x_2+\ldots +nx_n \geq \frac{n+1}{2}(x_1+x_2+\ldots +x_n) $$
\end{questao}

\begin{questao}
  Para todo natural $n > 2$, prove que

  \begin{enumerate}

  \item $\prod_{k=0}^{n} \binom{n}{k} \leq \left( \frac{2^n-2}{n-1}
    \right)^{n-1}$

  \item $ n! < \frac{n+1}{2} $

  \item $1 \times 3 \times 5 \times \ldots \times (2n-1) < n^n $
  \end{enumerate}

\end{questao}

\begin{questao}
  Sejam $x$ um real positivo e $m$ um número natural. Prove que

  $$ \frac{x(x+1)(x+2) \ldots (x+m-1)}{m!} \geq
  x^{1+\frac{1}{2}+\frac{1}{3}+\ldots+\frac{1}{m}} $$
\end{questao}

\begin{questao}
  Prove que a seguinte desigualdade é válida para quaisquer reais positivos
  $a_1,a_2,a_3,a_4$:

  $$ a_1 \cdot a_2^2 \cdot a_3^3 \cdot a_4^4 \leq \left(
  \frac{a_1+2a_2+3a_3+4a_4}{10}\right) $$
\end{questao}

\begin{questao}
  Sendo $a,b,c$ reais positivos tais que
  $\frac{1}{a}+\frac{1}{b}+\frac{1}{c}=1$, mostre que

  $$(a-1)(b-1)(c-1) \geq 8$$
\end{questao}

\begin{questao}
  Sejam $a_1,a_2,\ldots,a_n,b_1,b_2,\ldots,b_n$ números reais positivos. Prove
  que

  $$ \left( \frac{a_1}{b_1} + \frac{a_2}{b_2} + \ldots + \frac{a_n}{b_n} \right)
  \cdot (a_1b_1 + a_2b_2 + \ldots + a_nb_n) \geq (a_1+a_2+\ldots+a_n)^2 $$

  com igualdade se, e somente se, $b_1 = b_2 = \ldots = b_n$.
\end{questao}

\begin{questao}
  Sejam $a_1,a_2,\ldots,a_n,b_1,b_2,\ldots,b_n$ reais positivos tais que
  $\sum_{k=1}^{n}a_k = \sum_{k=1}^{n}b_k$. Mostre que

  $$ \sum_{k=1}^{n}\frac{(a_k)^2}{a_k+b_k} = \frac{1}{2} \sum_{k=1}^{n}a_k $$
\end{questao}

\begin{questao}
  Se $a,b,c$ são números reais positivos, prove que

  \begin{enumerate}

  \item $a^3+b^3+c^3 \geq a^2b+b^2c+c^2a$

  \item $\frac{a+b+c}{abc} \leq \frac{1}{a^2}+\frac{1}{b^2}+\frac{1}{c^2}$
  \end{enumerate}

\end{questao}

\begin{questao}
  Seja $\{a_k\}(k=1,2,\ldots,n,\ldots)$ uma sequência de inteiros positivos
  distintos dois a dois. Mostre que, para todo inteiro positivo $n$,

  $$ \sum_{k=1}^{n}\frac{a_k}{k^2} \geq \sum_{k=1}^{n}\frac{1}{k} $$
\end{questao}

\begin{questao}
  Demonstre que se $a,b,c$ são as medidas dos lados de um triângulo e $p$ é seu
  semiperímetro, então

  $$ \frac{a^n}{b+c} + \frac{b^n}{a+c} + \frac{c^n}{a+b} \geq \left( \frac{2}{3}
  \right) \cdot p^{n-1} $$

  para $n \geq 1$.
\end{questao}

\begin{questao}
  Se $x,y,z$ são números positivos e $x^2+y^2+z^2=8$, demonstre que

  $$ x^3+y^3+z^3 \geq 16 \sqrt{\frac{2}{3}} $$
\end{questao}

\begin{questao}
  Sejam $x$ e $y$ dois números reais estritamente positivos cuja soma é
  $8$. Prove que

  $$ \left( x+\frac{1}{y} \right)^2 + \left( y+\frac{1}{x} \right)^2 \geq
  \frac{289}{8} $$
\end{questao}

\begin{questao}
  Prove que, se $a,b,c$ são positivos e $a+b+c=\pi$, então
  $\sin(a)\sin(b)\sin(c) \geq 1/8$.
\end{questao}

\begin{questao}
  Se $a,b,c$ são positivos, prove que

  $$ abc \leq (a+b-c)(a+c-b)(b+c-a) $$
\end{questao}

\begin{questao}
  Se $a,b,c$ denotam as medidas dos lados de um triângulo, mostre que

  $$ 3(ab+bc+ca) \leq (a+b+c)^2 \leq 4(ab+bc+ca) $$
\end{questao}

\begin{questao}
  Sejam $x_1,x_2,y_1,y_2$ reais tais que $x_1^2+x_2^2 \leq 1$. Mostre que

  $$ (x_1y_1+x_2y_2-1)^2 \geq (x_1^2+x_2^2-1)(y_1^2+y_2^2-1) $$
\end{questao}

\begin{questao}
  Se $x$ e $y$ são reais não negativos e $x+y+\sqrt{2x^2+2xy+3y^2} = 4$, prove
  que $x^2y < 4$.
\end{questao}

\begin{questao}
  Sejam $a$, $b$, $c$ e $d$ números reais positivos. Prove que

  $$ \frac{1}{a}+\frac{1}{b}+\frac{4}{c}+\frac{16}{d} \geq \frac{64}{a+b+c+d} $$
\end{questao}

\begin{questao}
  Sejam $a_1,a_2,\ldots,a_n$ reais positivos tais que $a_1+a_2+\ldots+a_n <
  1$. Prove que

  $$ \frac{a_1a_2\ldots a_n\left( 1-(a_1+a_2+\ldots +a_n) \right)}
  {(a_1+a_2+\ldots +a_n)(1-a_1)(1-a_2)\ldots(1-a_n)} \leq \frac{1}{n^{n+1}} $$
\end{questao}

\begin{questao}
  Seja $P$ um ponto interior a um dado triângulo $ABC$. Sejam $D$, $E$ e $F$ os
  pés das perpendiculares traçadas desde $P$ até $BC$, $CA$ e $AB$
  respectivamente. Determine $P$ tal que

  $$ \frac{BC}{PD} + \frac{CA}{PE} + \frac{AB}{PF} $$

  é mínimo.
\end{questao}

\begin{questao}
  Seja $P$ um ponto no interior do triângulo $ABC$, e sejam $r,s,t$ as
  distâncias de $P$ aos lados $a,b,c$ do triângulo, respectivamente. Seja $R$ o
  raio da circunferência circunscrita ao triângulo. Prove que

  $$ \sqrt{r} + \sqrt{s} + \sqrt{t} \leq
  \frac{1}{\sqrt{2R}}(a^2+b^2+c^2)^{\frac{1}{2}} $$

  Em que condições se dá a igualdade?
\end{questao}

\begin{questao}
  Para $a$, $b$ e $c$ positivos, prove que

  $$ \frac{a^n}{b+c} + \frac{b^n}{a+c} + \frac{c^n}{a+b} \leq
  \frac{a^{n-1}+b^{n-1}+c^{n-1}}{2} $$
\end{questao}

\begin{questao}
  Suponha que $a_1,a_2,\ldots,a_n$ são reais positivos e $b_1,b_2,\ldots,b_n$ é
  uma permutação de $a_1,a_2,\ldots,a_n$. Mostre que

  $$ \frac{a_1}{b_1} + \frac{a_2}{b_2} + \ldots + \frac{a_n}{b_n} \geq 1 $$
\end{questao}

\begin{questao}
  Sejam $x_1, x_2,\ldots, x_n$ números positivos cuja soma é $1$. Prove que

  $$ \frac{x_1}{1+x_2+x_3+\ldots+x_n} + \frac{x_2}{1+x_1+x_3+\ldots+x_n} +
  \ldots + \frac{x_n}{1+x_1+x_2+\ldots+x_{n-1}} \geq \frac{n}{2n-1} $$
\end{questao}

\begin{questao}
  Sejam $a,b,c,d$ números reais não negativos tais que $ab+bc+cd+da = 1$. Prove
  que

  $$ \frac{a^3}{b+c+d} + \frac{b^3}{a+c+d} + \frac{c^3}{a+b+d} +
  \frac{d^3}{a+b+c} \geq \frac{1}{3} $$
\end{questao}

\begin{questao}
  Prove que, se $x_i>0$ para todo $i$, então

  $$ x_1^{x_1} + x_2^{x_2} + \ldots + x_n^{x_n} \geq (x_1\ldots
  x_n)^{\frac{x_1+\ldots+x_n}{n}} $$
\end{questao}

\begin{questao}
  Suponha que $x_1+x_2+\ldots+x_n=1$, onde $x_1, x_2, \ldots, x_n$ são reais
  positivos. Prove que

  $$ \frac{x_1}{\sqrt{1-x_1}} + \frac{x_2}{\sqrt{1-x_2}} + \ldots +
  \frac{x_n}{\sqrt{1-x_n}} \geq
  \frac{\sqrt{x_1}+\sqrt{x_2}+\ldots+\sqrt{x_n}}{\sqrt{n-1}} $$
\end{questao}

\begin{questao}
  Sejam $a$,$b$ e $c$ as medidas dos lados de um triângulo. Prove que
  $\sqrt{a+b-c} + \sqrt{b+c-a} + \sqrt{c+a-b} \leq \sqrt{a} + \sqrt{b} +
  \sqrt{c}$. Determine quando ocorre a igualdade.
\end{questao}

\begin{questao}
  Inscrevemos um triângulo $ABC$ em uma circunferência de raio $1$. Seja $I$ o
  incentro do $ABC$. Mostre que se $IA \cdot IB \cdot IC = 1$, então o triângulo
  $ABC$ é equilátero.
\end{questao}

\begin{questao}
  Prove que

  $$ \frac{a}{b+2c+3d} + \frac{b}{c+2d+3a} + \frac{c}{d+2a+3b} +
  \frac{d}{a+2b+3c} \geq \frac{2}{3} $$

  para quaisquer reais positivos $a,b,c,d$.
\end{questao}

\begin{questao}
  Suponha que $a,b,c > 0$. Prove que

  $$ \frac{a+\sqrt{ab}+\sqrt[3]{abc}}{3} \leq \sqrt[3]{a \frac{a+b}{2}
    \frac{a+b+c}{3}} $$
\end{questao}

\begin{questao}
  Sejam $a_1,a_2,\ldots,a_n$ reais positivos e $s = a_1+a_2+\ldots+a_n$. Prove
  que

  $$ (1+a_1)(1+a_2)\ldots(1+a_n) \leq
  1+s+\frac{s^2}{2!}+\frac{s^3}{3!}+\ldots+\frac{s^n}{n!} $$
\end{questao}

\begin{questao}
  As cem raízes do polinômio

  $$P(x) = x^{100}-600x^{999}+a_{98}x^{98}+\ldots+a_1x+a_0$$

  são reais e $P(7) > 1$. Prove que $P(x)$ tem pelo menos uma raiz maior do que
  $7$.
\end{questao}

\begin{questao}
  Seja $S$ um conjunto de $n$ inteiros positivos ímpares $a_1<a_2<a_3 < \ldots <
  a_n $ tais que não existam duas diferenças $|a_i-a_j|$ iguais. Prove que

  $$\sum_{i=1}^{n}a_i \geq \frac{n(n^2+2)}{3}$$
\end{questao}

\begin{questao}
  Mostre que

  $$ \frac{a}{b} + \frac{b}{c} + \frac{c}{a} \geq \frac{a+b}{b+c} +
  \frac{b+c}{a+b} + 1 $$

  para quaisquer reais positivos $a,b,c$.
\end{questao}

\begin{questao}
  Sejam $a,b,c$ números reais positivos tais que $abc=1$. Prove que

  $$ \frac{1}{a^3(b+c)} + \frac{1}{b^3(c+a)} + \frac{1}{c^3(a+b)} \geq
  \frac{3}{2} $$
\end{questao}

\begin{questao}
  Colocamos o número $1/(i+j-1)$ sobre a $i$-ésima coluna e a $j$-ésima linha de
  um tabuleiro $n \times n$. Então $n$ torres são colocadas sobre o tabuleiro,
  de forma que quaisquer duas não se ataquem. Prove que a soma dos números sob
  as torres é maior ou igual a $1$.
\end{questao}

\begin{questao}
  Prove que, dentre os triângulos cujo incírculo tem raio $1$, o triangulo
  equilátero tem perímetro mínimo.
\end{questao}

\begin{questao}
  Sejam $d_a$, $d_b$ e $d_c$ as distâncias desde um ponto interior de um
  triângulo até seus lados $a$, $b$ e $c$, respctivamente. Prove que

  $$ 2S^2 \geq 27d_ad_bd_cR $$

  onde $R$ é o circunraio e $S$ é a área do triângulo.
\end{questao}

\begin{questao}
  Para um triângulo de lados $a,b,c$ e área $A$, prove que

  $$ a^2+b^2+c^2 \geq 4\sqrt{3}A $$
\end{questao}

\begin{questao}
  Em um triângulo $ABC$, as bissetrizes $AD$, $BE$ e $CF$ encontram-se no ponto
  $I$. Mostre que

  $$ \frac{1}{4} < \frac{IA}{AD} \cdot \frac{IB}{BE} \cdot \frac{IC}{CF} \leq
  \frac{8}{27} $$
\end{questao}

\begin{questao}
  Sejam $m$ e $n$ inteiros positivos tais que $n \leq m$. Prove que

  $$ 2^nn! \leq \frac{(m+n)!}{(m-n)!} \leq (m^2+m)^n $$
\end{questao}

\begin{questao}
  Seja $ABC$ um triângulo e $P$ um ponto em seu interior. Mostre que pelo menos
  um dos ângulos $\widehat{PAB}$, $\widehat{PBC}$, $\widehat{PCA}$ é menor ou
  igual a $30^\circ$.
\end{questao}

\begin{questao}
  Sejam $x_1,x_2,\ldots,x_n$ números reais satisfazendo $|x_1+x_2+\ldots+x_n| =
  1$ e $\displaystyle |x_i| \leq \frac{n+1}{2}$ para $i=1,2,\ldots,n$. Prove que
  existe uma permutação $y_1,y_2,\ldots,y_n$ de $x_1,x_2,\ldots,n_n$
  satisfazendo

  $$|y_1+2y_2+\ldots+ny_n| \leq \frac{n+1}{2}$$
\end{questao}

\begin{questao}
  Determine inteiros $a$, $b$, $c$, $d$, $e$ e $f$ tais que, para todo racional
  não negativo $R$,

  $$ \left| \frac{aR^2+bR+c}{dR^2+eR+f} - \sqrt[3]{2} \right| < \left| {R -
    \sqrt[3]{2}} \right| $$
\end{questao}

\begin{questao}
  Suponha que $n \geq 2$ e que $a_1,a_2,\ldots,a_n$ são números reais maiores
  que $1$. Mostre que

  $$ \frac{1}{1+a_1} + \frac{1}{1+a_2} + \ldots + \frac{1}{1+a_n} \geq
  \frac{n}{1+\sqrt[n]{a_1a_2\ldots a_n}} $$
\end{questao}

\begin{questao}
  Determine o menor número real $C$ tal que

  $$ \sum_{1 \leq i \leq j \leq n}{x_ix_j(x_i^2+x_j^2)} \leq
  C(x_1+x_2+\leq+x_n)^4 $$

  para todo inteiro $n \geq 2$ e números reais não-negativos $x_i$. Determine
  quando ocorre a igualdade.
\end{questao}

\begin{questao}
  Prove que, para números reais $0 \geq a,b,c \geq 1$,

  $$ \frac{a}{b+c+1} + \frac{b}{c+a+1} + \frac{c}{a+b+1} + (1-a)(1-b)(1-c) \leq
  1 $$
\end{questao}

\begin{questao}
  Sejam $a_1, a_2, \ldots, a_n (n>3)$ números reais tais que

  \begin{eqnarray*}
    a_1+a_2+ \ldots + a_n & \geq & n \\ a_1^2+a_2^2+ \ldots + a_n^2 & \geq & n^2
  \end{eqnarray*}

  Mostre que $max(a_1,a_2,\ldots,a_n) \geq 2$.
\end{questao}

\begin{questao}
  Seja $P$ um polinômio com coeficientes positivos. Prove que se $P(1) \geq 1$
  então $P(x) \cdot P(1/x) \geq 1$ para todo real $x$ positivo.
\end{questao}

\begin{questao}
  Se $a,b,c,d,e$ são números reais no intervalo $[p;q], p > 0$, mostre que

  $$(a+b+c+d+e)\left( \frac{1}{a} + \frac{1}{b} + \frac{1}{c} + \frac{1}{d} +
  \frac{1}{e} \right) \leq 25 + 6 \left( \sqrt{\frac{p}{q}} -\sqrt{\frac{q}{p}}
  \right)^2$$

  e determine quando ocorre a igualdade.
\end{questao}

\begin{questao}
  Mostre a {\it Desigualdade de Schur}: se $x,y,z$ são reais não negativos e
  $r>0$, então

  $$ x^r(x-y)(x-z) + y^r(y-x)(y-z) + z^r(z-x)(z-y) \geq 0 $$
\end{questao}

\begin{questao}
  Sejam $a,b,c$ números reais. Prove que

  $$ \frac{(b+c-a)^2}{(b+c)^2+a^2} + \frac{(c+a-b)^2}{(c+a)^2+b^2} +
  \frac{(a+b-c)^2}{(a+b)^2+c^2} \leq \frac{3}{5} $$
\end{questao}

\begin{questao}
  Quem é maior?

  \begin{enumerate}

  \item $31^{11}$ ou $17^{14}$

  \item $\sqrt{8}{8!}$ ou $\sqrt{9}{9!}$

  \item $\frac{10^{1997}+1}{10^{1998}+1}$ ou $\frac{10^{1998}+1}{10^{1999}+1}$

  \item $(1,000001)^{1000000}$ ou $2$
  \end{enumerate}

\end{questao}

\begin{questao}


  \begin{enumerate}

  \item Mostre que sendo $n$ um inteiro positivo
    $$ 2\sqrt{n+1}-2\sqrt{n} < \frac{1}{\sqrt{n}} < 2\sqrt{n}-2\sqrt{n-1}$$

  \item Determine
    $$ \left \lfloor
    1+\frac{1}{\sqrt{2}}+\frac{1}{\sqrt{3}}+\ldots+\frac{1}{\sqrt{1000000}}
    \right \rfloor $$
  \end{enumerate}

\end{questao}

\begin{questao}
  Mostre que

  \begin{enumerate}

  \item
    $\displaystyle \frac{1}{n+1}+ \frac{1}{n+2}+\ldots+\frac{1}{2n} \leq
    \frac{3}{4}$

  \item
    $\displaystyle \frac{1}{1^2}+ \frac{1}{2^2}+\ldots+\frac{1}{n^2} \leq
    \frac{7}{4} - \frac{1}{n}$
  \end{enumerate}

\end{questao}

\begin{questao}

  \begin{enumerate}

  \item Para $0<a<b$, mostre que
    $$(n+1)(b-a)a^n < b^{n+1}-a^{n+1} < (n+1)(b-a)b^n$$

  \item Mostre que
    $$ \left(1+\frac{1}{n}\right)^n < \left(1+\frac{1}{n+1}\right)^{n+1}$$

  \item Prove que $a^{n+1}+nb^{n+1} > (n+1)ab^n$ se $a \not = b$.
  \end{enumerate}

\end{questao}

\begin{questao}
  Seja $\displaystyle a = \frac{m^{m+1}+n^{n+1}}{m^m+n^n}$, onde $m$ e $n$ são
  inteiros positivos. Prove que

  $$a^m+a^n \geq m^m+n^n$$
\end{questao}

\begin{questao}
  Se $0 \leq a,b,c,d \leq 1$, prove que

  $$(1-a)(1-b)(1-c)(1-d)+a+b+c+d \geq 1$$
\end{questao}

\begin{questao}
  Se $0 \leq a,b,c \leq 1$, mostre que

  $$\frac{a}{b+c+1} + \frac{b}{a+c+1} + \frac{c}{a+b+1} +(1-a)(1-b)(1-c) \leq
  1$$
\end{questao}

\begin{questao}
  Seja $n$ um inteiro positivo, e $a_i \geq 1$ para $i=1,2,\ldots,n$. Mostre que

  $$ (1+a_1)(1+a_2)\ldots(1+a_n) \geq \frac{2^n}{n+1}(1+a_1+\ldots+a_n)$$
\end{questao}

\begin{questao}
  Sendo $n$ um inteiro positivo, mostre que

  $$ \frac{1}{2} \cdot \frac{3}{4} \cdot \frac{5}{6} \cdot \ldots \cdot
  \frac{2n-1}{2n} \leq {1}{\sqrt{3n+1}} $$
\end{questao}

\begin{questao}
  Sejam $x_1,x_2,\ldots,x_{n+1}$ reais positivos tais que

  $$ \frac{1}{1+x_1}+\frac{1}{1+x_2}+\ldots+\frac{1}{1+x_{n+1}} = 1 $$

  Prove que $x_1x_2\ldots x_{n+1} \geq n^{n+1}$.
\end{questao}

\begin{questao}
  Para $a$, $b$ e $c$ reais positivos, mostre que

  $$ a+b+c \leq \frac{a^2+b^2}{2c} +\frac{a^2+c^2}{2b} +\frac{b^2+c^2}{2a} \leq
  \frac{a^3}{bc} + \frac{b^3}{a} + \frac{c^3}{ab} $$
\end{questao}

\begin{questao}
  Sejam $a,b,c$ números reais positivos tais que $a+b+c=1$. Mostre que

  $$ \frac{a^7+b^7}{a^5+b^5} + \frac{b^7+c^7}{b^5+c^5} + \frac{c^7+a^7}{c^5+a^5}
  \geq \frac{1}{3} $$
\end{questao}

\begin{questao}
  Prove que

  $$ \frac{a}{\sqrt{a^2+8bc}} + \frac{b}{\sqrt{b^2+8ac}} +
  \frac{c}{\sqrt{c^2+8ab}} \geq 1 $$

  para quaisquer números reais positivos $a$, $b$ e $c$.
\end{questao}

\begin{questao}
  Se $a,b,c$ são números reais positivos satisfazendo $a+b+c=abc$, prove que
  $\frac{1}{\sqrt{1+a^2}} + \frac{1}{\sqrt{1+b^2}} + \frac{1}{\sqrt{1+c^2}} \leq
  \frac{3}{2}$, e determine quando ocorre a igualdade.
\end{questao}

\begin{questao}
  Sejam $a,b,c$ as medidas dos lados de um triângulo. Prove que

  $$ a^2b(a-b) + b^2c(b-c) + c^2a(c-a) \geq 0$$

  Determine quando ocorrerá a igualdade.
\end{questao}

\begin{questao}
  Sejam $a$, $b$ e $c$ reais positivos tais que $abc=1$. Prove que

  $$ \left( a-1+\frac{1}{b} \right) \left( b-1+\frac{1}{b} \right) \left(
  c-1+\frac{1}{a} \right) \leq 1$$
\end{questao}

%%% Local Variables:
%%% mode: latex
%%% coding: utf-8-unix
%%% fill-column: 80
%%% TeX-master: "MASTER"
%%% End:
