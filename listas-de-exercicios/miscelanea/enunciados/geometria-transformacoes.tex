\begin{questao}
  São dados um triângulo acutângulo $ABC$ e um ponto $P_0$ pertencente ao lado
  $\overbar{AB}$. Construímos os pontos $P_1,P_2,\ldots,P_n,\ldots$ da seguinte
  forma: $Q_0$ é a projeção ortogonal de $P_0$ sobre o lado $\overbar{BC}$,
  $R_0$ é projeção ortogonal de $Q_0$ sobre o lado $\overbar{CA}$, $P_1$ é a
  projeção ortogonal de $R_0$ sobre o lado $\overbar{AB}$, e assim
  sucessivamente, para todo natural $n$, $Q_n$ é a projeção ortogonal de $P_n$
  sobre o lado $\overbar{BC}$, $R_n$ é projeção ortogonal de $Q_n$ sobre o lado
  $\overbar{CA}$, $P_{n+1}$ é a projeção ortogonal de $R_n$ sobre o lado
  $\overbar{AB}$.

  Prove que existe um ponto $P$, pertencente ao lado $\overbar{AB}$, do qual os
  pontos $P_0,P_1,P_2,\ldots,P_n,\ldots$ se aproximam, com o crescer de $n$, no
  seguinte sentido: para todo natural $n$, a distância de $P$ a $P_{n+1}$ é
  menor que a distância de $P$ a $P_n$. Indique uma construção geométrica para
  obter o ponto $P$.
\end{questao}

\begin{questao}
  As medidas dos lados do triângulo $ABC$ são
  $AB=14,BC=13,CA=15$. A bissetriz do ângulo $A$ encontra
  $\overbar{BC}$ em $D$. Com centro em $D$, é construída uma
  circunferência que tangencia $\overbar{AC}$ em $L$. A extensão
  de $AD$ corta tal circunferência em $J$.

  Determine $AJ/AL$.
\end{questao}

\begin{questao}
  As circunferências $\Gamma_1,\Gamma_2,\Gamma_3,\Gamma_4$
  têm o mesmo raio e intersectam-se em um mesmo ponto em comum.\\
  $\Gamma_1$ e $\Gamma_2$ intersectam-se em $P$ e $O$.\\
  $\Gamma_2$ e $\Gamma_3$ intersectam-se em $Q$ e $O$.\\
  $\Gamma_3$ e $\Gamma_4$ intersectam-se em $R$ e $O$.\\
  $\Gamma_4$ e $\Gamma_1$ intersectam-se em $S$ e $O$.\\
  $P,Q,R,S$ estão nesta ordem em sentido anti-horário.\\
  Seja $p$ a reta tangente comum a $\Gamma_1$ e $\Gamma_2$.\\
  Seja $q$ a reta tangente comum a $\Gamma_2$ e $\Gamma_3$.\\
  Seja $r$ a reta tangente comum a $\Gamma_3$ e $\Gamma_4$.\\
  Seja $s$ a reta tangente comum a $\Gamma_4$ e $\Gamma_1$.\\
  $p$ e $q$ intersectam-se em $A$.\\
  $q$ e $r$ intersectam-se em $B$.\\
  $r$ e $s$ intersectam-se em $C$.\\
  $s$ e $p$ intersectam-se em $D$.\\

  Mostre que $A,B,C,D$ são concíclicos.
\end{questao}

\begin{questao}
  Três círculos congruentes têm um ponto $O$ em comum e
  são interiores a um triângulo. Cada círculo tangencia exatamente
  dois lados deste triângulo. prove que o circuncentro e o incentro e
  o ponto $O$ são colineares.
\end{questao}

\begin{questao}
  Seja $A$ um ponto de intersecção de duas circunferêcias
  coplanares $C_1,C_2$, não congruentes, de centros $O_1,O_2$,
  respectivamente. Uma das tangentes comuns toca $C_1$ em $P_1$ e
  $C_2$ em $P_2$, enquanto a outra toca $C_1$ em $Q_1$ e
  $C_2$ em $Q_2$. Seja $M_1$ o ponto médio de
  $\overbar{P_1Q_1}$ e $M_2$ o ponto médio de
  $\overbar{P_2Q_2}$. Prove que $\angle O_1AO_2 = \angle M_1AM_2$.
\end{questao}

\begin{questao}
  Um triângulo escaleno $A_1A_2A_3$ têm lados
  $a_1,a_2,a_3$ (onde $a_i$ é oposto a $A_i$). Para $i=1,2,3$,
  $M_i$ é o ponto médio do lado $a_i$ e $T_i$ é onde o incírculo
  toca o lado $a_i$. Denotamos por $S_i$ a reflexão de $T_i$ em
  relação à bissetriz interna do ângulo $A_i$. Prove que as retas
  $M_1S_1,M_2S_2,M_3S_3$ são concorrentes.
\end{questao}

\begin{questao}
  Prove que o único polígono regular que pode ter todos os
  seus vértices sobre os pontos de um reticulado é o quadrado.
\end{questao}

\begin{questao}

  \begin{enumerate}

    \item Sejam $ABCD$ um quadrado, $M$ o ponto médio de
    $\overbar{AB}$, $N$ o ponto médio de $BC$ e $I$ a
    intersecção de $DN$ e $CM$. Calcule a área do triângulo
    $NIC$, tomando $AB=1$.

    \item Seja um paralelogramo de $ABCD$, sejam $M,N,P,Q$ os pontos
    médios de $AB,BC,CD,DA$ respectivamente. Unindo cada vértice do
    paralelogramo aos pontos médios dos lados não adjacentes, obtemos
    o octógono estrelado $ANDMCQBPA$. Mostre que a área deste
    octógono é $3/5$ da área do paralelogramo.

    \item Considere o triângulo isósceles $ABC$ e seja $G$ seu
    centro de gravidade. Mostre que a área do quadrilátero côncavo
    $AGBCA$ é $2/3$ da área do triângulo $ABC$.

    \item A propriedade demonstrada no segundo item vale para qualquer
    quadrilátero ou só para o paralelogramo?
  \end{enumerate}
\end{questao}

\begin{questao}
  Sobre os lados $AB,BC,CA$ de um triângulo $ABC$
  tomam-se, respectivamente, os pontos
  $C',A',B'$, de tal forma que

  $$ \frac{AC'}{C'B} = \frac{BA'}{A'C} = \frac{CB'}{B'A} = r $$

  Os segmentos $AA'$ e $BB'$ intersectam-se em $B_1$.\\
  Os segmentos $BB'$ e $CC'$ intersectam-se em $A_1$.\\
  Os segmentos $CC'$ e $AA'$ intersectam-se em $B_1$.\\

  A partir de $r$ e da área $S$ do triângulo $ABC$, determine a
  área do triângulo $A_1B_1C_1$.
\end{questao}

\begin{questao}
  Seja $ABC$ um triângulo e sejam $K,L,M$ pontos
  pertencentes aos lados $BC,CA,AB$ respectivamente. Prove que a
  área de pelo menos um dos triângulos $AML, BKM, CLK$ é menor ou
  igual a um quarto da área do triângulo $ABC$.
\end{questao}

\begin{questao}
  Considere uma esfera $S$ e seja $\pi$ o plano que
  contém seu equador. Seja $P$ o pólo sul da esfera. Considere, no
  plano $\pi$, uma circunferência $C$. Seja $C'$ a projeção, com
  centro em $P$, de $C$ sobre a esfera. Mostre que $C'$ é um
  círculo.

  {\it Obs}.: Dado um ponto $Q \in \pi$, a reta que passa por $P$
  e $Q$ intersecta a esfera $S$ num ponto $Q' \not = P$. Dizemos
  que $Q'$ é a projeção com centro em $P$ de $Q$ sobre a esfera
  $S$.
\end{questao}

\begin{questao}
  Seja $ABC$um triângulo equilátero cujo lado mede $2$ e
  $\Gamma$ a circunferência inscrita em $ABC$.
  \begin{enumerate}

    \item Demonstre que, para todo $P \in \Gamma$,

    $$PA^2 + PB^2 +PC^2 = 5$$


    \item Demonstre que para todo ponto $P \in \Gamma$ é possível
    construir um triângulo cujos lados têm as mesmas medidas dos
    segmentos segmentos $AP,BP,CP$ ne que a área de tal triângulo é
    $\sqrt{3}/4$.
  \end{enumerate}
\end{questao}

%%% Local Variables:
%%% mode: latex
%%% coding: utf-8-unix
%%% fill-column: 80
%%% TeX-master: "MASTER"
%%% End:
