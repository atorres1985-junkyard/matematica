\begin{questao}
  Seja $A$ uma matriz $m \times n$ que satisfaz as
  seguintes condições:

  \begin{itemize}
    \item $m \leq n$;

    \item cada elemento de $A$ é $0$ ou $1$;

    \item se $f$ é uma função injetora de $\{1,2,\ldots,m\}$ em
    $\{1,2,\ldots,n\}$, então o elemento $A(i,f(i))$ é $0$ para
    algum $i (1 \leq i \leq n)$.
  \end{itemize}

  Mostre que existem conjuntos
  $S \subseteq \{1,2,\ldots,m\}, T \subseteq \{1,2,\ldots,n\}$ que satisfazem

  \begin{itemize}
    \item o elemento $A(i,j)$ é $0$ para qualquer $i \in S, j \in
    T$;

    \item $\#S + \#T > n$.
  \end{itemize}
\end{questao}

\begin{questao}
  Doze anões vivem em uma floresta e cada um deles tem uma
  casa que é pintada de vermelho ou azul. No $i$-ésimo mês de cada
  ano, o $i$-ésimo anão visita todos os seus amigos e se encontra a
  maioria deles vivendo em casas de cor diferente de sua própria, ele
  decide juntar-se a eles e muda a cor de sua casa. Mostre que, mais
  cedo ou mais tarde, nenhum anão precisará mudar a cor de sua
  casa. (As amizades são mútuas e não mudam.)
\end{questao}

\begin{questao}
  Considere em $\mathbb{R}^2$ o retângulo de vértices
  $(0,0), (m,0), (0,n), (m,n)$, onde $m,n$ são inteiros positivos
  ímpares. O retângulo é particionado em triângulos e a partição
  satisfaz as seguintes propriedades:

  \begin{itemize}
    \item Todo triângulo tem um lado sobre uma reta da forma $x=k$ ou
    $y=l$, $j,k \in \mathbb{Z}$. Um lado deste tipo será chamado
    de {\it lado bom};

    \item A altura relativa a qualquer lado bom é sempre igual a $1$.

    \item Todo ``lado mau'' (i.e., não bom) é comum a dois triângulos.
  \end{itemize}

  Mostre que existem pelo menos dois triângulos da partição com dois
  lados bons.
\end{questao}

\begin{questao}
  As cidades $C_1,\ldots,C_N$ são servidas por $n$
  companhias aéreas $A_1,\ldots,A_n$. Há um voo entre quaisquer duas
  cidades, e todas as companhias que oferecem o serviço o fazem em
  ambas as direções. Se $N \geq 2^n+1$, prove que pelo menos uma das
  companhias pode oferecer uma viagem com um número ímpar de escalas.
\end{questao}

\begin{questao}
  Prove as afirmações a seguir:

  \begin{itemize}
    \item Dados $1993$ pontos e $992015$ arestas ligando pares
    destes pontos, existem três pontos $A,B,C$ tais que $A$ não
    está ligado a $B$, nem $A$ a $C$, nem $B$ a $C$.

    \item Dados $1993$ pontos e $992015$ arestas ligando pares
    destes pontos (nunca há duas arestas ligando o mesmo par de
    pontos), existem três pontos $A,B,C$ tais que $A$ está ligado
    a $B$, $A$ está ligado a $C$ e $B$ está ligado a $C$.
  \end{itemize}
\end{questao}

\begin{questao}
  Quarenta e nove alunos resolvem um conjunto de três
  problemas. A pontuação para cada problema é um inteiro entre zero e
  sete. Prove que existem dois estudantes A e B tais que, para cada
  problema, o estudante A obteve pelo menos tantos pontos quanto B.
\end{questao}

\begin{questao}
  Há $n$ pessoas em uma festa. Prove que existem duas
  pessoas tais que, das $n-2$ pessoas restantes, há pelo menos
  $\left \lfloor \frac{n}{2} \right \rfloor - 1$ delas que conhece
  ambas ou não conhece nenhuma das duas.
\end{questao}

\begin{questao}
  Todas as ruas da cidade de Zurbagan são de mão
  dupla. Quando é necessário ter as ruas reparadas é introduzida,
  temporariamente, mão única em algumas dessas ruas, as restantes
  permanecendo de mão dupla. Após parte das ruas serem reparadas é
  introduzida mão dupla nestas ruas e nas outras é introduzida,
  temporariamente, mão única. Durante ambos os períodos de reparos é
  possível ir de um lugar a qualquer outro em Zurbagan. Prove que pode
  ser introduzida mão única em todas as ruas de Zurbagan de forma que
  ainda seja possível ir de um lugar a qualquer outro.
\end{questao}

\begin{questao}
  Há $n$ cidades em um certo país, quaisquer duas delas
  ligadas por uma estrada de mão única. Prove que para $n \not = 2$
  ou $n \not = 4$ então a direção do movimento através das estradas
  pode ser escolhida de forma que se possa ir de uma cidade a qualquer
  outra sem passar por mais de uma cidade. Prove também que no caso
  $n = 2$ ou $n = 4$ tal organização de tráfico é impossível.
\end{questao}

\begin{questao}
  Prove que se um retângulo $R$ pode ser particionado em
  retângulos que possuem um dos lados de medida igual a um número
  inteiro, $R$ possuirá um lado de medida inteira.
\end{questao}

\begin{questao}
  Seja $S$ um conjunto consistindo de $m$ pares
  $(a,b)$ de inteiros positivos que satisfazem $1 \leq a < b \leq
  n$. Mostre que há pelo menos

  $$ 4m \left( \frac{m-\frac{n^2}{4}}{3n} \right) $$

  ternas $(a,b,c)$ tais que $(a,b),(b,c),(a,c)$ pertencem a $S$.
\end{questao}

\begin{questao}
  São dados vinte e um pontos sobre uma
  circunferência. Prove que pelo menos cem dentre os arcos
  determinados por estes pontos subentendem ângulos centrais menores
  que $120^\circ$.
\end{questao}

\begin{questao}
  Em Pasárgada existem $N$ cidades e $2N-1$ estradas,
  sempre de mão única, ligando estas cidades; cada estrada liga apenas
  duas cidades. Pasárgada é totalmente interligada por estas cidades,
  isto é, a partir de qualquer cidade é possível chegar a qualquer
  outra por uma sequência de estradas. Prove que existe alguma estrada
  que pode ser interditada de forma que Pasárgada continue totalmente
  interligada pelas estradas restantes.
\end{questao}

\begin{questao}
  Durante uma conferência, cada um dos cinco matemáticos
  cochilaram exatamente duas vezes. Para cada par destes matemáticos,
  houve um momento em que ambos estavam cochilando
  simultaneamente. Prove que, em algum momento, três estavam
  cochilando ao mesmo tempo.
\end{questao}

\begin{questao}
  Dez localidades são servidas por duas linhas aéreas de tal
  forma que entre quaisquer duas localidades há um voo sem escalas e
  em ambos os sentidos. Prove que pelo menos uma das linhas pode
  fornecer dois passeios circulares disjuntos com um número ímpar de
  escalas.
\end{questao}

\begin{questao}  
  $1985$ pessoas participam de uma conferência. Em cada grupo de
  três pessoas pelo menos duas falam a mesma língua. Se cada pesssoa
  fala no máximo cinco línguas, mostre que existe uma língua falada
  por pelo menos $200$ pessoas nessa conferência.
\end{questao}

\begin{questao}
  Em um país você pode ir de avião de uma cidade até qualquer
  outra. Quando não há um voo direto entre duas cidades há um com
  escalas. Durante o voo de uma cidade até outra podemos passar no
  máximo uma vez em cada uma das cidades desse país. Nós chamamos de
  tamanho do trajeto entre duas cidades o número de escalas
  suficientes para se ir de uma cidade até outra (através de qualquer
  caminho).\\
  Prove que se existem dois trajetos de tamanho máximo então eles
  possuem uma escala em comum.
\end{questao}

\begin{questao}
  Em um torneio de tênis em turno completo prove que exatamente um das
  seguintes situações ocorre:

  \begin{itemize}
    \item Os jogadores podem ser particionados em dois grupos de modo
    que cada um dos jogadores de um destes grupos venceu todos os seus
    jogos contra os jogadores do outro grupo.

    \item Todos os participantes podem ser ordenados de $1$ até
    $n$ de forma que o $i$-ésimo jogador venceu o $i+1$-ésimo
    e o $n$-ésimo venceu o primeiro.
  \end{itemize}
\end{questao}

\begin{questao}
  Considere nove pontos no espaço de forma que nunca há quatro desses
  postos que ficam no mesmo plano. Cada par de pontos é ligado por uma
  aresta (isto é, um segmento de reta) e cada aresta é colorida de
  azul, vermelho ou não é colorida. Encontre o menor valor de $n$
  tal que sempre que se pintam exatamente $n$ arestas o conjunto das
  arestas coloridas necessariamente contém um triângulo cujas arestas
  têm todas a mesma cor.
\end{questao}

\begin{questao}
  É dado no plano um conjunto finito $E$ de pontos de coordenadas
  inteiras. É sempre possível colorir todos os pontos de $E$ com
  duas cores, vermelho ou branco, de modo que para toda reta $r$
  paralela, quer ao primeiro quer ao segundo eixo coordenado, a
  diferença entre o número de pontos vermelhos e o número de pontos
  brancos, pertencentes a $r$, seja $-1$, $0$ ou $1$?
  Justifique sua resposta.  
\end{questao}

\begin{questao}
  Quarenta e nove alunos resolvem um conjunto de três problemas. A
  pontuação para cada problema é um inteiro entre zero e sete. Prove
  que existem dois estudantes $A$ e $B$ tais que, para cada
  problema, $A$ obteve pelo menos tantos pontos quanto $B$.
\end{questao}

\begin{questao}
  Considere um cavalo de xadrez sobre um tabuleiro $4 \times N$. É
  possível, em $4N$ movimentos consecutivos, ele visitar cada uma
  das casas do tabuleiro e retornar à casa inicial?
\end{questao}

\begin{questao}
  Nove matemáticos encontram-se em uma conferência internacional e
  descobrem que, em qualquer grupo de três deles, pelo menos dois
  falam a mesma língua em comum. Se cada um dos matemáticos fala, no
  máximo, três línguas, prove que pelo menos três matemáticos falam
  uma língua em comum.
\end{questao}

\begin{questao}
  Seja $G$ um grafo conexo com $k$ arestas. Demonstre que é
  possível enumerar as arestas de $G$ de $1$ a $k$ de tal modo
  que para cada vértice $G$ que pertence a duas ou mais arestas, os
  números atribuídos a estas arestas não tenham divisor comum maior
  que $1$.
\end{questao}

\begin{questao}
  Uma tela de computador exibe uma figura constituída por $n$ pontos
  de uma superfície esférica, quatro a quatro não coplanares, e por
  todos os segmentos que eles determinam, cada um dos quais está
  colorido de azul ou vermelho. A cada um dos pontos está associada
  uma tecla, cujo toque provoca a mudança de coloração de todos os
  segmentos que incidem no ponto. Sabe-se que para cada três pontos
  existe uma sequência de toques que torna vermelhos os três lados do
  triângulo por eles determinado.

  \begin{itemize}
    \item Mostre que existe uma sequência de toques que torna
    vermelhos todos os segmentos da tela.

    \item Calcule, em função de $n$, o número mínimo de toques
    necessários para tornar vermelhos todos os segmentos, no caso mais
    desfavorável.
  \end{itemize}
\end{questao}

%%% Local Variables:
%%% mode: latex
%%% coding: utf-8-unix
%%% fill-column: 80
%%% TeX-master: "MASTER"
%%% End:
