\begin{questao}
  Seja $ABCD$ um quadrilátero circunscritível e
  $K,L,M,N$ os pontos de tangência dos lados $AB,BC,CD,DA$ na
  circunferência, respectivamente. As retas $AC$ e $BD$ se
  interceptam em $S$ e as retas $AB$ e $CD$ se interceptam em
  $P$. Mostre que se $S,K,M$ são colineares, então $P,N,L$ são
  colineares.
\end{questao}

\begin{questao}
  Sejam $A,B,C,D,E,F$ seis pontos distintos, nesta ordem,
  de uma circunferência. As retas tangentes à circunferência que
  passam por $A$ e $D$ e as retas $BF$ e $CE$ são
  concorrentes. Mostre que $AD,BC,EF$ ou são paralelas ou são
  colineares.
\end{questao}

\begin{questao}
  Seja $P$ um ponto no interior do triângulo
  $ABC$. Sejam $P_1$ e $P_2$ os pés das perpendiculares a $AC$
  e $BC$ por $P$, respectivamente. Sejam $Q_1$ e $Q_2$ os pés
  das perpendiculares a $AP$ e $BP$ por $C$,
  respectivamente. Mostre que $Q_1P_2, Q_2P_1,AB$ são colineares.
\end{questao}

\begin{questao}
  Seja $ABCD$ um quadrilátero inscrito na circunferência
  $\Gamma$. As retas $AB$ e $CD$ encontram-se em $P$ e as
  retas $AD$ e $BC$ encontram-se em $Q$. Sejam $E$ e $F$ os
  pontos de contato das retas tangentes a $Q$ que passam por
  $Q$. Mostre que $P,E,F$ são colineares.
\end{questao}

\begin{questao}
  Seja $ABC$ um triângulo com $\angle B <
  45^\circ$. Seja $D$ um ponto de $BC$ tal que o incentro do
  triângulo $ABC$ coincide com o circuncentro $O$ do triângulo
  $ABC$. Seja $\Gamma$ o circuncírculo do triângulo $AOC$. Seja
  $P$ a intersecção das retas tangentes a $\Gamma$ por $A$ e
  $C$. As retas $AD$ e $CO$ se cruzam em $Q$. Seja $X$ o
  ponto de intersecção de $PQ$ e a reta tangente a $\Gamma$ que
  passa por $O$. Mostre que $XO = XD$
\end{questao}

\begin{questao}
  Seja $H$ o ortocentro do triângulo acutângulo
  $ABC$. As tangentes por $A$ à circunferência de diâmetro $BC$
  a tocam em $P$ e $Q$. Prove que $P,H,Q$ são colineares.
\end{questao}

\begin{questao}
  Sejam $O$ e $N$ o circuncentro e o centro da
  circunferência dos nove pontos do triângulo $ABC$,
  respectivamente. Seja $N'$ um ponto tal que $\angle
  N' BA = NBC $ e  $\angle N' AB = NAC $. A mediatriz
  de $OA$ corta $BC$ em $A'$. Os pontos $B'$ e
  $C'$ são definidos analogamente. Prove que
  $A',B',C'$ pertencem a uma reta perpendicular a
  $ON'$.
\end{questao}

\begin{questao}
  As tangentes a uma circunferência de centro $O$,
  traçadas por um ponto exterior $C$, tocam a circunferência nos
  pontos $A$ e $B$. Seja $S$ um ponto qualquer da
  circunferência. As retas $SA,SB,SC$ cortam o diâmetro
  perpendicular a $OS$ nos pontos $A',B',C'$
  respectivamente. Prove que $C'$ é ponto médio de
  $A' B'$.
\end{questao}

\begin{questao}
  Seja $\Gamma$ uma circunferência de centro $O$
  tangente aos lados $AB$ e $AC$ do triângulo $ABC$ nos pontos
  $E$ e $F$. A reta perpendicular ao lado $BC$ por $O$
  intercepta $F$ no ponto $D$. Prove que $A,D$ e $M$ (ponto
  médio de $BC$) são colineares.
\end{questao}

%%% Local Variables:
%%% mode: latex
%%% coding: utf-8-unix
%%% fill-column: 80
%%% TeX-master: "MASTER"
%%% End:
