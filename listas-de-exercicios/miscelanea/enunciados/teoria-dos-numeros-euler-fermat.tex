\begin{questao}
  Encontre um par de inteiros positivos $a,b$ tais que
  \begin{itemize}
    \item $(ab(a+b))$ não é divisível por $7$;

    \item $(a+b)^7-a^7-b^7$ é divisível por $7^7$.
  \end{itemize}
\end{questao}

\begin{questao}
  Prove que o conjunto $\{2^k-3, k=2,3,\ldots\}$ contém um subconjunto infinito
  cujos elementos são primos dois a dois.
\end{questao}

\begin{questao}
  Seja $n>3$ um inteiro ímpar. Prove que existe um primo $p$ tal que $p$ divide
  $2^{\phi(n)}-1$ mas não divide $n$.
\end{questao}

\begin{questao}
  Prove que existe um número da forma $1999 \ldots 91$ (com mais de dois noves)
  que é múltiplo de $1991$.
\end{questao}

\begin{questao}
  Mostre que para todo inteiro positivo $n$ existe uma potência de $2$ em uma
  {\it string} de $n$ zeros sucessivos.
\end{questao}

\begin{questao}
  Um inteiro positivo é denominado {\it número duplo} se sua representação
  decimal consiste de um bloco de dígitos não iniciado por zero, seguido
  imediatamente de um bloco idêntico. Então, por exemplo, $360360$ é um número
  duplo, mas $36036$ não é.

  Mostre que existem infinitos números duplos que são quadrados perfeitos.
\end{questao}

\begin{questao}
  Prove que existe $n \in N$ tal que os últimos $1992$ algarismos
  de $n^{1991}$ são iguais a $1$.
\end{questao}

\begin{questao}
  Um número é dito {\it esburacado} se seus dígitos na base
  decimal são alternadamente nulo e não-nulo, e seu dígito das
  unidades é não-nulo. Por exemplo, $4050201$ é um número esburacado
  mas $4050$ não é. Encontre todos os números inteiros positivos que
  não dividem nenhum número esburacado.
\end{questao}

\begin{questao}
  Prove que existe uma potência de $2$ cujos últimos
  $1000$ dígitos são todos iguais a $1$ ou $2$.
\end{questao}

\begin{questao}
  Sendo $k \geq 2$ e $n_1,n_2,\ldots,n_k \geq 1$ números naturais tais que
  $n_2|2^{n_1}-1, n_3|2^{n_2}-1, \ldots, n_k|2^{n_{k-1}}-1, n_1|2^{n_k}-1$.

  Mostre que $n_1 = n_2 = \ldots = n_k = 1$.
\end{questao}

\begin{questao}
  Mostre que se $n$ é um inteiro maior do que $1$ então
  $n$ não divide $2^n-1$.
\end{questao}

\begin{questao}
  Determine todos os inteiros $n \geq 1$ tais que
  $(2^n+1)/n^2$ seja inteiro.
\end{questao}

\begin{questao}
  Existe um inteiro positivo com exatamente $2000$ fatores
  primos distintos tal que $n|(2^n+1)$?
\end{questao}

\begin{questao}
  Determine todos os pares $n,p$ de inteiros estritamente
  positivos tais que
  \begin{itemize}
    \item $p$ é primo,

    \item $n \leq 2p$,

    \item $(p-1)^n+1$ é divisível por $n^{p-1}$.
  \end{itemize}
\end{questao}

\begin{questao}
  Se $p>5$ é um primo, então o número $(p-1)!+1$ não é
  uma $k$-ésima potência de $p$, $k$ natural.
\end{questao}

\begin{questao}
  Se $p$ é um primo ímpar, mostre que

  $$ 1^{p-1} + 2^{p-1} + 3^{p-1} + \ldots + (p-1)^{p-1} \equiv
  p+(p-1)! \pmod{p^2} $$
\end{questao}

\begin{questao}
  \begin{itemize}
    \item Encontre todos os primos $p$ tais que $p$ divide
    $5^{p^2}+1$.

    \item Mostre que existem infinitos pares de primos $p,q$ tais
    que $p$ divide $q^{p^2}+1$.
  \end{itemize}
\end{questao}

\begin{questao}
  Mostre que se $n$ é ímpar então $2^{n!}-1$ é divisível por $n$.
\end{questao}

\begin{questao}
  Prove que, dado um primo $p$, existem infinitos números
  da forma $2^n-n$, $n$ natural, divisíveis por $p$.
\end{questao}

\begin{questao}
  Seja $a>1$ um número inteiro. Encontre todos os números
  naturais que dividem algum dos números da forma $a_n = a^n +
  a^{n-1} +  a^{n-2} + \ldots + a^{2} + a + 1$.
\end{questao}

\begin{questao}
  Determine todas as ternas de números $(a;m;n)$ tais que
  $a^m+1$ divida $(a+1)^n$.
\end{questao}

\begin{questao}
  Seja $A(m,n) = m^{3^{4n+6}} -m^{3^{4n}} - m^5 +
  m^3$. Encontre todos os valores de $n$ para os quais $A(m,n)$ é
  divisível por $1992$ para todo $m$ inteiro.
\end{questao}

\begin{questao}
  Dizemos que um número natural $n$ possui a propriedade
  $P$ se, e somente se,

  $$ n|a^n-1 \Rightarrow n^2|a^n-1$$

  \begin{itemize}
    \item Mostre que todo número primo possui a propriedade $P$.

    \item Mostre que existem infinitos números compostos com a
    propriedade $P$.
  \end{itemize}
\end{questao}

\begin{questao}
  Prove que, dados $a,b \in N^{*}$, existe $c \in N^{*}$ tal que
  infinitos números da forma $an+b, n \in N$ têm todos os fatores
  primos menores ou iguais a $c$.
\end{questao}

\begin{questao}
  Mostre que, para $n$ inteiro positivo, a sequência

  $$ 2,2^2,2^{2^2},\ldots \pmod{n} $$

  é constante a partir de certo ponto.
\end{questao}

\begin{questao}
  Os três últimos dígitos de $1978^m$ são iguais aos três
  últimos dígitos de $1978^n$ ($1 \leq m < n, m,n \in
  {N})$. Determine $m,n$ tais que $m+n$ seja mínimo.
\end{questao}

\begin{questao}
  Prove que todos os divisores de $2^p-1$, $p$ primo,
  são maiores que $p$.
\end{questao}

\begin{questao}
  Sejam $a,b,n$ inteiros positivos tais que $b>1$ e
  $(b^n-1)|a$. Mostre que a representação do número $a$ na base
  $b$ contém pelo menos $n$ dígitos diferentes de zero.
\end{questao}

\begin{questao}
  Sejam $a,n$ números inteiros positivos. Prove que se
  existe $s$ tal que $(a-1)^s$ é divisível por $n$, então
  $1+a+a^2+\ldots+a^{n-1}$ é divisível por $n$.
\end{questao}

\begin{questao}
  {\it(Teorema de Wolstenholme)} Seja $p \geq 5$ um número
  primo. Sendo $H_{p-1} = \frac{A_p}{B_p}$, com $A_p$ e $B_P$
  primos entre si, mostre que $A_p$ é divisível por $p^2$.

  (Nota: $H_n=1+1/2+1/3+\ldots+1/n$ denota o $n$-ésimo {\it número
    harmônico}.)
\end{questao}

\begin{questao}
  Seja $b,m,n$ inteiros positivos tais que $b>1$ e $m
  \not = n$. Prove que se $b^m-1$ e $b^n-1$ têm os mesmos
  divisores primos, então $b+1$ é potência de $2$.
\end{questao}

\begin{questao}
  Seja $p$ um número primo. Demonstre que existe um número
  primo $q$ tal que, para todo inteiro $n$, o número $n^p-p$ não
  é divisível por $q$.
\end{questao}

%%% Local Variables:
%%% mode: latex
%%% coding: utf-8-unix
%%% fill-column: 80
%%% TeX-master: "MASTER"
%%% End:
