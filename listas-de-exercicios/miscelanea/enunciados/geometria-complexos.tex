\begin{questao}
  Sejam $u_1,\ldots,u_m$ vetores no plano, cujas normas
  são maiores que $1$, e cuja soma é o vetor nulo. Mostre que existe
  uma permutação $v_1,\ldots,v_m$ destes vetores de tal forma que
  cada uma das somas parciais $v_1, v_1+v_2,\ldots,v_1+\ldots+v_m$ é
  menor ou igual a $\sqrt{5}$.
\end{questao}

\begin{questao}
  Dado um triângulo $ABC$, seja $G$ o seu baricentro e
  $M$ o ponto médio de $BC$. Sejam $X$ sobre $AB$ e $Y$
  sobre $AC$ tal que os pontos $X,G,Y$ são colineares e $XY$ e
  $BC$ são paralelas. Suponha que $XC$ e $GB$ intersectam-se em
  $Q$, e $YB$ e $GC$ em $P$. Moste que os triângulos $MPQ$
  e $ABC$ são semelhantes.
\end{questao}

\begin{questao}
  Seja $ABC$ um triângulo escaleno, com incentro $I$,
  ortocentro $H$ e baricentro $G$. Demonstre que $\angle GIH >
  90^\circ$.
\end{questao}

\begin{questao}
  Considere um poliedro convexo $P_1$ com nove vérices
  $A_1,A_2,\ldots,A_9$. Seja $P_i$ o poliedro obtido de $P_1$
  através da translação que leva o vértice $A_1$ até $A_i$
  $(i=2,3,\ldots,9)$. Prove que pelo menos dois dos poliedros
  $P_1,P_2,\ldots,P_9$ têm um ponto em comum.
\end{questao}

\begin{questao}
  Prove que para todo natural $m$, existe um conjunto
  finito de pontos no plano, digamos $S$, com a seguinte
  propriedade: para todo ponto $A \in S$ há exatamente $m$ pontos
  em $S$ que estão a uma unidade de distância de $A$.
\end{questao}

\begin{questao}
  Dado quatro planos paralelos distintos, prove que existe
  um tetraedro regular com um vértice sobre cada plano.
\end{questao}

\begin{questao}
  Suponha que cada lado do quadrilátero $ABCD$ seja
  estendido, sendo determinados quatro pontos $A', B',
  C', D'$ tais que $AB=BA',BC=CB',
  CD=DC', DA = AD'$. Agora suponha que são dados
  $A', B', C', D'$ ao invés de
  $A,B,C,D$. É possível construir o quadrilátero $ABCD$?
\end{questao}

\begin{questao}
  Prove que para todo inteiro positivo $n$, $(2+i)^n \not
  = (2-i)^n$. Conclua que os ângulos agudos do triângulo de lados
  $3,4,5$ não são racionais quando expressos em graus.
\end{questao}

\begin{questao}
  Determine se existem ou não $1975$ pontos sobre a
  circunferência unitária tais que a distância entre quaisquer dois
  deles é um número racional.
\end{questao}

\begin{questao}
  Seja $ABCD$ um quadrilátero com $AD=BC$ e $\angle A +
  \angle B = 120^\circ$. Três triângulos equiláteros $ACP,DCQ,DBR$
  são traçados sobre $AC,DC,DB$ de tal forma que o segmento $AB$
  fique o mais afastado possível dos novos vértices. Prove que os três
  novos vértices $P,Q,R$ são colineares.
\end{questao}

\begin{questao}
  São dados um triângulo $A_1A_2A_3$ e um ponto $P_0$
  no plano. Definimos $A_s=A_{s-3}$ para $s \geq 4$ e construímos
  uma sequência de pontos $P_0,P_1,P_2,\ldots$ tais que $P_{k+1}$
  é a imagem de $P_k$ sob a rotação com centro $A_{k+1}$ e ângulo
  de $120^\circ$ no sentido horário. Prove que se $P_{1986} = P_0$
  então $A_1A_2A_3$ é equilátero.
\end{questao}

\begin{questao}
  Prove que existe um polígono convexo de $1990$ lados que
  possui ambas as propriedades abaixo:
  \begin{itemize}
    \item todos os ângulos internos são iguais;
    \item Os comprimentos dos lados são os números
    $1^2,2^2,\ldots,1989^2,1990^2$ em alguma ordem.
  \end{itemize}
\end{questao}

\begin{questao}
  Seja $ABCDEF$ um hexágono inscrito em uma
  circunferência de raio $r$. Mostre que, se $AB=CD=EF=r$ então os
  pontos médios de $BC,DE,FA$ são vértices de um triângulo
  equilátero.
\end{questao}

\begin{questao}
  Seja $ABCDEF$ um quadrilátero convexo tal que

  $$\angle B + \angle D + \angle F = 360^\circ ;
  \frac{AB}{BC} \cdot \frac{CD}{DE} \cdot \frac{EF}{FA} = 1$$

  Prove que

  $$ \frac{BC}{CA} \cdot \frac{AE}{EF} \cdot \frac{FD}{DB} = 1$$
\end{questao}

\begin{questao}
  Seja $D$ um ponto no interior do triângulo acutângulo
  $ABC$, com $ DA \cdot DB \cdot AB + DB \cdot DC \cdot BC + DC
  \cdot DA \cdot CA = AB \cdot BC \cdot CA$. Determine as possíveis
  posições que $D$ pode ocupar.
\end{questao}

%%% Local Variables:
%%% mode: latex
%%% coding: utf-8-unix
%%% fill-column: 80
%%% TeX-master: "MASTER"
%%% End:
