\begin{questao}
  Calcule

  $$ \cfrac{1}{2+\cfrac{1}{3+\cfrac{1}{4+\cfrac{1}{\ddots+\cfrac{1}{1999}}}}} +
  \cfrac{1}{1+\cfrac{1}{1+\cfrac{1}{3+\cfrac{1}{4+\cfrac{1}{\ddots+\cfrac{1}{1999}}}}}} $$
\end{questao}

\begin{questao}
  Encontre todas as soluções reais da equação com $n$ frações

  $$ 1+\cfrac{1}{1+\cfrac{1}{1+\cfrac{1}{\ldots+\cfrac{1}{x}}}} = 1 $$
\end{questao}

\begin{questao}
  Se $x \not = 1, y \not = 1, x \not = y$ e $\frac{yz-x^2}{1-x} =
  \frac{xz-y^2}{1-y}$, então ambas as frações são iguais a $x+y+z$.
\end{questao}

\begin{questao}
  Sendo $a,b,c$ racionais distintos, prove que o número a seguir é sempre o
  quadrado de um número racional:

  $$ {\frac{1}{(a-b)^2} + \frac{1}{(b-c)^2} + \frac{1}{(c-a)^2}} $$
\end{questao}

\begin{questao}
  Seja $\alpha$ uma raiz da equação $x^3-3x+1 = 0$.

  \begin{enumerate}

  \item Racionalize $1/\alpha$.

  \item Prove que $\alpha^2-2$ é outra raiz da mesma equação.

  \end{enumerate}

\end{questao}

\begin{questao}
  Seja $\alpha = \sqrt[3]{12} + \sqrt[3]{18} + 1$.

  \begin{enumerate}

  \item Encontre um polinômio de coeficientes racionais do qual $\alpha$ seja
    raiz.

  \item Racionalize $1/\alpha$.
  \end{enumerate}

\end{questao}

\begin{questao}
  Fatore as expressões a seguir:

  \begin{enumerate}

  \item $x^5+x^4+1$

  \item $x^{10}+x^5+1$

  \item $(b-c)^3+(c-a)^3+(a-b)^3$

  \item $2(x^4+y^4+z^4+w^4) - (x^2+y^2+z^2+w^2) + 8xyzw$

  \item $n^4-20n^2+4$

  \item $x^4+4y^4$

  \item $x^4+y^4$

  \item $a^n-b^n$

  \item $a^n+b^n$ se $n$ é ímpar
  \end{enumerate}

\end{questao}

\begin{questao}

  \begin{enumerate}

  \item Fatore a expressão $a^3+b^3+c^3-3abc$.

  \item Usando a fatoração acima obtida, determine uma solução da equação cúbica
    $x^3+px+q = 0$.

  \end{enumerate}

\end{questao}

\begin{questao}
  Prove que existem infinitos números naturais com a seguinte propriedade: o
  número $n^4+a$ é composto para todo natural $n$.
\end{questao}

\begin{questao}
  Calcule
  $$ \frac{(10^4+324)(22^4+324)(34^4+324)(46^4+324)(58^4+324)}
  {(4^4+324)(16^4+324)(28^4+324)(40^4+324)(52^4+324)} $$
\end{questao}

\begin{questao}
  Supondo que o inteiro $n$ é soma de dois números triangulares,

  $$ n = \frac{a^2+a}{2} + \frac{b^2+b}{2} $$

  expresse $4n+1$ como soma de dois quadrados, $4n+1 = x^2+y^2$. Reciprocamente,
  se $4n+1$ é soma de dois quadrados, prove que $4n+1$ é soma de dois números
  triangulares.
\end{questao}

\begin{questao}
  Sejam $a,b$ números reais e seja $f(x) = {ax+b}^{-1}$. Para que valores de
  $a,b$ existem três números reais distintos $x_1,x_2,x_3$ tais que $f(x_1) =
  x_2, f(x_2) = x_3, f(x_3) = x_1$?
\end{questao}

\begin{questao}
  Temos que $a^3-3a^2+5a = 1, b^3-3b^2+5b=5$, onde $a,b \in \mathbb{R}$. Calcule
  $a+b$.
\end{questao}

\begin{questao}

  \begin{enumerate}

  \item Um carro viaja de A até B a uma velocidade de 40 km/h e então de B até A
    a uma velocidade de 60 km/h. A velocidade média deste carro foi maior ou
    menor que 50 km/h?

  \item Dados um copo de café e um copo de leite, cada um deles com a mesma
    quantidade de líquido, pegamos uma colher de leite e colocamos no café, e
    então pegamos uma colher desta mistura formada e colocamos no leite. Há mais
    café no copo de leite ou leite no copo de café?

  \item Suponha que a Terra é uma esfera e uma corda está amarrada ao redor da
    linha do Equador. Agora suponha que essa corda é aumentada em um metro,
    formando uma circunferência maior concêntrica com a linha do Equador, qual
    será a distância entre a superfície da Terra e a corda? E se eu fizesse o
    mesmo para uma bola de futebol, qual seria a distância?
  \end{enumerate}

\end{questao}

\begin{questao}
  Uma balança de Roberval é imprecisa, pois seus pratos possuem medidas
  diferentes e seus pratos possuem pesos diferentes. Três objetos de pesos
  diferentes $A,B,C$ são pesados separadamente. Quando colocados sobre o prato
  esquerdo são equilibrados por pesos $A_1,B_1,C_1$ respectivamente. Quando $A$
  e $B$ são colocados sobre o prato direito são equilibrados por pesos $A_2$ e
  $B_2$ respectivamente. Determine o peso $C$ em função de
  $A_1,B_1,C_1,A_2,B_2$.
\end{questao}

\begin{questao}
  Seja $P(x,y) = 5x^2-6xy+2y^2$.

  \begin{enumerate}

  \item Determine quantos elementos de $\{1,2,\ldots,100\}$ são valores de $P$.

  \item Prove que o produto de valores de $P$ é um valor de $P$.
  \end{enumerate}

\end{questao}

\begin{questao}
  Resolva as equações nos reais:

  \begin{enumerate}

  \item $x^3-6x-40 = 0$

  \item $y^4-12y^2-16y-4 = 0$
  \end{enumerate}

\end{questao}

\begin{questao}

  \begin{enumerate}

  \item Se $n$ é um inteiro positivo e $2n+1$ é quadrado perfeito, mostre que
    $n+1$ é a soma de dois quadrados perfeitos.

  \item Se $n$ é um inteiro positivo e $3n+1$ é quadrado perfeito, mostre que
    $n+1$ é a soma de três quadrados perfeitos.

  \item Se $p$ é primo e $pn+1$ é quadrado perfeito, mostre que $n+1$ é a soma
    de $p$ quadrados perfeitos.

  \item Se $a,b$ são inteiros consecutivos, mostre que $a^2+b^2+(ab)^2$ é
    quadrado perfeito.

  \item Se $2a$ é a média harmônica de $b$ e $c$, mostre que a soma dos
    quadrados de $a,b,c$ é o quadrado de um racional.

  \item Se $N$ está entre dois quadrados perfeitos consecutivos e difere destes
    por $x,y$ respectivamente, prove que $N-xy$ é quadrado perfeito.
  \end{enumerate}

\end{questao}

\begin{questao}
  Seja $\alpha$ uma raiz da equação $x^3-3x+1=0$. Prove que $\alpha^2-2$ também
  é uma raiz da mesma equação.
\end{questao}

\begin{questao}
  Resolva as equações, nos reais:

  \begin{enumerate}

  \item $ 20 \left(\frac{x-2}{x+1}\right)^2 - 5 \left(\frac{x+2}{x+1}\right)^2 +
    48 \cdot \frac{x^2-4}{x^2-1} = 0$

  \item $ (1+x)^8+(1+x^2)^4 = 2x^4$

  \item $ x^2+\frac{x^2}{(x+1)^2} = 3$

  \item $ \sqrt[3]{x}+\sqrt[3]{1-x} = \frac{3}{2} $

  \item $(x-2)(x+1)(x+4)(x+7) = 19$

  \item $ {\sqrt[4]{97-x} + \sqrt[4]{x} = 5}$
  \end{enumerate}

\end{questao}

\begin{questao}
  Encontre todas as soluções da equação ${(x+y+z)^3=x^3+y^3+z^3}$, com $x,y,z$
  reais.
\end{questao}

\begin{questao}
  Considere o sistema
  \begin{eqnarray*}
    x^2-y^2 & = & 0\\ (x-a)^2+y^2 & = & 1
  \end{eqnarray*}
  Determine $a$ real para que o número de soluções do sistema seja

  \begin{enumerate}

  \item dois;

  \item três.
  \end{enumerate}

\end{questao}

\begin{questao}
  Resolva as equações, nos reais:

  \begin{enumerate}

  \item
    \begin{eqnarray*}
      \frac{x^2}{y} + \frac{y^2}{x} & = & 12 \\ \frac{1}{x} + \frac{1}{y} & = &
      \frac{1}{3}
    \end{eqnarray*}

  \item
    \begin{eqnarray*}
      (x^2+1)(y^2+1) & = & 10 \\ (x+y)(xy-1) & = & 3
    \end{eqnarray*}

  \item
    \begin{eqnarray*}
      (x^2+y^2)xy & = & 78 \\ x^4+y^4 & = & 97
    \end{eqnarray*}

  \item
    \begin{eqnarray*}
      x^2-y^2 & = & 5 \\ x^2-xy+y^2 & = & 7
    \end{eqnarray*}

  \item
    \begin{eqnarray*}
      x^3+y^3 & = & 1 \\ x^2y+2xy^2+y^3 & = & 2
    \end{eqnarray*}

  \item
    \begin{eqnarray*}
      (x^2+y^2)\frac{x}{y} & = & 6 \\ (x^2-y^2)\frac{y}{x} & = & 1
    \end{eqnarray*}

  \item
    \begin{eqnarray*}
      2(x+y) & = & 5xy \\ 8(x^3+y^3) & = & 65
    \end{eqnarray*}

  \item
    \begin{eqnarray*}
      x+y & = & 1 \\ x^4+y^4 & = & 7
    \end{eqnarray*}

  \item
    \begin{eqnarray*}
      x^2+y^2+6x+2y & = & 0 \\ x+y+8 & = & 0
    \end{eqnarray*}

  \item
    \begin{eqnarray*}
      x^2 -y^2 & = & 5 \\ x^2-xy+y^2 & = & 7
    \end{eqnarray*}

  \item
    \begin{eqnarray*}
      x^3+y^3 & = & 1 \\ x^2y+2xy^2y^3 & = & 2
    \end{eqnarray*}
    \
  \item
    \begin{eqnarray*}
      (x^2+y^2)\frac{x}{y} & = & 6 \\ (x^2-y^2)\frac{y}{x} & = & 1
    \end{eqnarray*}
  \end{enumerate}

\end{questao}

\begin{questao}
  Resolva o sistema:
  \begin{eqnarray*}
    x+y+z & = & a \\ x^2+y^2+z^2 & = & b^2 \\ x^3+y^3+z^3 & = & c^3 \\
  \end{eqnarray*}
\end{questao}

\begin{questao}
  Prove que o sistema a seguir não possui soluções reais:

  \begin{eqnarray*}
    x+y+z & = & 1 \\ x^3+y^3+z^3+xyz & = & x^4+y^4+z^4+1
  \end{eqnarray*}
\end{questao}

\begin{questao}
  Resolva os sistemas, nos reais:

  \begin{enumerate}
    
  \item
    \begin{eqnarray*}
      x+y-z & = & -1 \\ x^2-y^2+z^2 & = & 1 \\ -x^3+y^3+z^3 & = & -1
    \end{eqnarray*}
    
  \item
    \begin{eqnarray*}
      x+y-z & = & 7 \\ x^2+y^2-z^2 & = & 37 \\ x^3+y^3-z^3 & = & 1
    \end{eqnarray*}
    
  \item
    \begin{eqnarray*}
      x^2+y^2+z & = & 2 \\ y^2+z^2+x & = & 2 \\ z^2+x^2+y & = & 2
    \end{eqnarray*}
    
  \item
    \begin{eqnarray*}
      \frac{xy}{x+y} & = & 1 \\ \frac{xz}{x+z} & = & 2 \\ \frac{yz}{y+z} & = & 3
    \end{eqnarray*}
    
  \item
    \begin{eqnarray*}
      y^3-6x^2+12x-8 & = & 0 \\ z^3-6y^2+12y-8 & = & 0 \\ x^3-6z^2+12z-8 & = & 0
      \\
    \end{eqnarray*}
    
  \item
    \begin{eqnarray*}
      (x+y)^3 & = & z \\ (y+z)^3 & = & x \\ (x+z)^3 & = & y
    \end{eqnarray*}
  \end{enumerate}

\end{questao}

\begin{questao}
  Encontre todas as soluções reais do sistema

  \begin{eqnarray*}
    ax+by & = & (x-y)^2 \\ by+cz & = & (y-z)^2 \\ cz+ax & = & (z-x)^2
  \end{eqnarray*}

  em que $a,b,c$ são constantes reais positivas.
\end{questao}

\begin{questao}
  Seja $a_1,a_2,\ldots,a_n$ uma permutação dos inteiros $1,2,\ldots,n$ e seja

  $$ S = \frac{a_1}{1} + \frac{a_2}{2} + \frac{a_3}{3} + \ldots +
  \frac{a_n}{n} $$

  Encontre um número natural $n$ tal que entre os valores que $S$ assume, nas
  diversas permutações $a_1,\ldots,a_n$, aparecem todos os inteiros entre $n$ e
  $n+100$.
\end{questao}

\begin{questao}
  Prove que se

  $$ \frac{x}{y+z+t} = \frac{y}{z+t+x} = \frac{z}{t+x+y} = \frac{t}{x+y+z} $$

  então o número

  $$ \frac{x+y}{z+t} + \frac{y+z}{t+x} + \frac{z+t}{x+y} + \frac{t+x}{z+y} $$

  é um inteiro.
\end{questao}

\begin{questao}
  Demonstre que se $a<b<c<d$ então $(a+b+c+d)^2 > 8(ac+bd)$.
\end{questao}

\begin{questao}
  Sejam $a,b,c$ reais positivos.

  \begin{enumerate}
    
  \item Mostre que $\frac{a^2+b^2}{a+b} \geq \frac{a+b}{2}$.
    
  \item Usando a identidade

    $$ \frac{a^2}{a+b} + \frac{b^2}{b+c} + \frac{c^2}{c+a} = \frac{b^2}{a+b} +
    \frac{c^2}{b+c} + \frac{a^2}{c+a} $$

    mostre que
    
    $$ \frac{a^2}{a+b} + \frac{b^2}{b+c} + \frac{c^2}{c+a} \geq \frac
    {a+b+c}{2} $$
  \end{enumerate}

\end{questao}

%%% Local Variables:
%%% mode: latex
%%% coding: utf-8-unix
%%% fill-column: 80
%%% TeX-master: "MASTER"
%%% End:
