\begin{questao}
  Sejam $AB$ uma corda em um círculo e $P$ um ponto deste. Seja
  $Q$ a projeção de $P$ sobre $AB$ e sejam $R,S$ as projeções
  de $P$ sobre as tangentes ao círculo em $A,B$
  respectivamente. Prove que $PQ$ é a média geométrica de $PR$ e
  $PS$.
\end{questao}

\begin{questao}
  Sejam $A,B$ as intersecções de dois círculos. Uma reta passando
  por $A$ intersecta os círculos em $C$ e $D$. Prove que $PQ$
  é tangente ao círculo de diâmetro $AB$.
\end{questao}

\begin{questao}
  Sejam $H$ o ortocentro do triângulo $ABC$, não retângulo, e
  $M$ o ponto médio do lado $BC$. A circunferência de diâmetro
  $AM$ encontra a circunferência circunscrita a $ABC$ em
  $P$. Mostre que $P,H,M$ são colineares.
\end{questao}

\begin{questao}
  Seja $ABC$ um triângulo com $\angle ACB = 120^\circ$. Sejam $O,G,H$
  o circuncentro, o baricentro e o ortocentro de $ABC$. Seja X o
  incentro de $AHB$.
  \begin{itemize}
    \item Prove que os pontos $A,B,O,H$ pertencem a uma mesma circunferência e
    mostre que seu centro é ponto médio do arco $ACB$.

    \item Prove que $O,G,X,H$ são colineares.
  \end{itemize}
\end{questao}

\begin{questao}
  Duas circunferências $\Gamma_1,\Gamma_2$ estão contidas no
  interior de uma circunferência $\Gamma$ e são tangentes a
  $\Gamma$ nos pontos distintos $M,N$ respectivamente. A
  circunferência $\Gamma_1$ passa pelo centro de $\Gamma_2$. A
  reta que passa pelos dois pontos de intersecção de $\Gamma_1$ e
  $\Gamma_2$ intersecta $\Gamma$ em $A$ e $B$. As retas $MA$
  e $MB$ intersectam $\Gamma_1$ respectivamente em $C$ e $D$.

  Prove que $CD$ é tangente a $\Gamma_2$.
\end{questao}

%%% Local Variables:
%%% mode: latex
%%% coding: utf-8-unix
%%% fill-column: 80
%%% TeX-master: "MASTER"
%%% End:
