\begin{questao}
  Mostre que
  \begin{enumerate}

    \item $2^{70}+3^{70}$ é divisível por $13$.

    \item ${2222}^{5555}+{5555}^{2222}$ é divisível por $7$.

    \item $11^{10}-1$ é divisível por $100$.

    \item Se $n>0$, $12$ divide $n^4-4n^3+5n^2-2n$.

    \item $n^5-5n^3+4n$ é divisível por $120$

    \item $n^2+3n+5$ não é divisível por $121$.

    \item Se $n>0$, $7$ não divide $2^n+1$.

    \item $3^{6n} - 2^{6n}$ é divisível por $35$.

    \item $2^{2x+1}+1$ é divisível por $3$.

    \item $2903^n-803^n-464^n+261^n$ é divisível por $1897$.
  \end{enumerate}
\end{questao}

\begin{questao}
  Determine os dois últimos dígitos de
  \begin{enumerate}

    \item $3^{1234}$;

    \item $2^{999}$;

    \item $9^{9^9}$;

    \item $14^{14^{14}}$;
  \end{enumerate}

\end{questao}

\begin{questao}
  Determine os três últimos dígitos de $7^{9999}$.
\end{questao}

\begin{questao}
  Determine o último dígito do número $ \left \lfloor
    \frac{10^{1992}}{10^{83}+7} \right \rfloor$.
\end{questao}

\begin{questao}
  Seja $p$ um número primo, e $w,n$ inteiros tais que
  $2^p+3^p=w^n$. Prove que $n=1$.
\end{questao}

\begin{questao}
  Quantos são os quadrados perfeitos módulo $2^n$?
\end{questao}

\begin{questao}
  Prove que todo número primo com $3$ é congruente a uma potência de
  $2$ módulo $3^n$.
\end{questao}

\begin{questao}
  Mostre que $ \frac{3^{77}-1}{2} $ é ímpar e composto.
\end{questao}

\begin{questao}
  Sejam $x,y,z$ inteiros tais que $x^3+y^3-z^3$ é um múltiplo de
  $7$. Mostre que um desses números é múltiplo de $7$.
\end{questao}

\begin{questao}
  Determine se a matriz abaixo é invertível ou não:
  $$ \left[
    \begin{array}{cccc}
      64809 & 91185 & 42391 & 44350 \\
      61372 & 26536 & 23165 & 71489 \\
      82561 & 39189 & 16596 & 46152 \\
      39177 & 55538 & 79922 & 51237
    \end{array}
  \right] $$

\end{questao}

\begin{questao}
  Prove que para inteiros positivos $a,m (a>1)$,
  $$\left(\frac{a^m-1}{a-1},a-1 \right) = (a-1,m)$$
\end{questao}

\begin{questao}
  Se $4^n+2^n+1$ é primo, prove que $n$ é uma potência de
  $3$.

\end{questao}

\begin{questao}
  Determine se a matriz abaixo é invertível ou não:

  $$ \left[ 
    \begin{array}[center]{cccc}
      64809 & 91185 & 42391 & 44350 \\
      61372 & 26536 & 23165 & 71489 \\
      82561 & 39189 & 16596 & 46152 \\
      39177 & 55538 & 79922 & 51237
    \end{array}
  \right] $$
\end{questao}

\begin{questao}
  \begin{enumerate}

    \item Se $n$ é um inteiro positivo maior que $1$ e $2^n+n^2$ é
    primo, prove que $n \equiv 3 \pmod{6}$.

    \item Seja $x \equiv 23 \pmod{24}$. Se $a,b$ são inteiros
    positivos tais que $ab=x$, então $a+b$ é um múltiplo de $24$.

    \item Se $n^2+m$ e $n^2-m$ são quadrados perfeitos, então $m$
    é divisível por $24$.

    \item Se $2n+1$ e $3n+1$ são quadrados perfeitos, então $n$ é
    divisível por $40$.
  \end{enumerate}
\end{questao}

\begin{questao}
  Seja $S$ um conjunto de números primos tais que se $a,b \in S$,
  então $ab+4 \in S$. Mostre que $S$ deve ser o conjunto vazio.
\end{questao}

\begin{questao}
  Mostre que a sequência $$11,111,\ldots$$ não contém quadrados.
\end{questao}

\begin{questao}
  {\it(Números de Fermat)} Seja $F_k = 2^{2^k}+1$.
  \begin{enumerate}

    \item Prove que $(F_i,F_j) = 1$ se, e somente se, $i \not = j$.

    \item Prove que os números primos da forma $2^k+1$ são números
    (primos) de Fermat.
  \end{enumerate}
\end{questao}

\begin{questao}
  Prove que um número de Fermat
  \begin{enumerate}

    \item Nunca será um quadrado perfeito;

    \item Nunca será um cubo perfeito;

    \item Se $n>1$, nunca será um número triangular, ou seja, um
    número da forma $\frac{m(m+1)}{2}, m \in \mathbb{N}$.
  \end{enumerate}
\end{questao}

\begin{questao}
  Prove que se $m,n$ são números naturais e $m$ é ímpar,
  então $(2^m-1,2^n+1) = 1$.
\end{questao}

\begin{questao}
  Mostre que $\widehat{111 \ldots 1}$ é divisível por $41$ se, e
  somente se, $n$ é divisível por $5$.
\end{questao}

\begin{questao}
  Seja $A$ a soma dos dígitos de $4444^{4444}$ e seja
  $B$ a soma dos dígitos de $A$. Encontre a soma dos dígitos de
  $B$.

\end{questao}

\begin{questao}
  Para quais inteiros positivos $n$ a soma $5^n+n^5$ é
  divisível por $n$? Qual é o menor $n$ satisfazendo esta condição?
\end{questao}

\begin{questao}
  Encontre um conjunto finito $\mathbb{A}$, com o menor número de
  elementos possível, para o qual existe uma função $f: N \rightarrow
  A$ tal que, se $|i-j|$ é primo, então $f(i) \not = f(j)$.
\end{questao}

\begin{questao}
  Seja $d$um inteiro positivo diferente de
  $2,5,13$. Mostre que podemos escolher $a,b$ distintos em
  $\{2,5,13,d\}$ tais que $ab-1$ não seja quadrado perfeito.

\end{questao}

\begin{questao}
  Determine todos os possíveis valores da soma dos algarismos
  de um quadrado perfeito.

\end{questao}

\begin{questao}
  Encontre todos os inteiros positivos $x,y$ que satisfazem
  $x^2-y!=2001$.

\end{questao}

\begin{questao}
  Sejam $d_1,d_2,\ldots,d_k$ todos os divisores positivos de
  um natural $n$, onde $1=d_1<d_2<\ldots<d_k=n$. Encontre todos os
  naturais $n$ tais que $k \geq 4$ e $d_1^2+d_2^2+d_3^2+d_4^2 = n$.

\end{questao}

\begin{questao}
  Seja $p$ um número primo e $w,n$ inteiros tais que
  $2^p+3^p=w^n$. Prove que $n=1$.

\end{questao}

\begin{questao}
  Quais são os quadrados perfeitos $\pmod{2^n}$?

\end{questao}

\begin{questao}
  Prove que todo número primo com $3$ é congruente a uma
  potência de $2$ $\pmod{3^n}$.

\end{questao}

\begin{questao}
  Prove que existe uma sucessão $a_0,a_1,a_2,\ldots$ onde
  $a_i\in\{0,1,2,3,4,5,6,7,8,9\}$ com $a_0 = 6$ tal que para cada
  inteiro positivo $n$, sendo
  $x_n=a_0+10a_1+100a_2+\ldots+10^{n-1}a_{n-1}$, $x_n^2-x_n$ é
  divisível por $10^n$.
\end{questao}

\begin{questao}
  \begin{enumerate}

    \item Seja $mdc(m,k) = 1$. Prove que existem inteiros
    $a_1,\ldots,a_m$ e $b_1,\ldots,b_k$ tais que cada um dos
    produtos $a_ib_j$ ($i=1,\ldots,m;j=1,\ldots,k$) deixa um resto
    diferente ao ser dividido por $m \cdot k$.

    \item Seja $mdc(m,k) > 1$. Prove que para quaisquer inteiros
    $a_1,\ldots,a_m$ e $b_1,\ldots,b_k$ existem dois produtos
    $a_ib_j,a_sb_t$ ($(i,j) \not = (s,t)$) que deixam o mesmo resto
    ao serem divididos por $m \cdot k$.
  \end{enumerate}
\end{questao}

\begin{questao}
  Seja $p_i$ o $i$-ésimo primo. Prove que, para todo $n \in N$,
  $p_1p_2\ldots p_n+1$ não é quadrado perfeito.
\end{questao}

%%% Local Variables:
%%% mode: latex
%%% coding: utf-8-unix
%%% fill-column: 80
%%% TeX-master: "MASTER"
%%% End:
