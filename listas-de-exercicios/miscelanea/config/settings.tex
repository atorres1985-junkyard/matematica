\usepackage{amsmath,amssymb}

\usepackage{unicode-math}
\usepackage{mathtools}
\usepackage{tabulary}
\usepackage{multirow}
\usepackage{verse}
\usepackage{moreenum}
\usepackage{enumitem}
\usepackage{pstricks}

\usepackage{polyglossia}
\setmainlanguage{portuges}

% Listas de números separados por vírgulas no modo matemático
% precisam de um tratamento diferenciado

\newcommand{\splitAtCommas}[1]{%
  \begingroup
  \begingroup\lccode`~=`, \lowercase{\endgroup
    \edef~{\mathchar\the\mathcode`, \penalty0 \noexpand\hspace{0pt plus 1em}}%
  }\mathcode`,="8000 #1%
  \endgroup
}

% Variáveis de controle de parágrafo

\setlength{\parindent}{2em}
\setlength{\parskip}{0.25em}

% Modo matemático em displaystyle por default

\everymath{\displaystyle}

% Numeração das questões

\newcounter{questoes}[subsection]

\newenvironment {questao}
{\stepcounter{questoes}
  \hrulefill
  
  \textbf{Questão \arabic{questoes}}:}
{}

%%% Local Variables:
%%% mode: latex
%%% coding: utf-8-unix
%%% fill-column: 80
%%% TeX-master: "MASTER"
%%% End:
